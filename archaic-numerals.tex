\documentclass[10pt, a4paper, twoside]{article}
\usepackage{fontspec}
\setmainfont[
  BoldFont=CambriaB,
  Mapping=tex-text, 
  Numbers={OldStyle, Proportional}, 
  Ligatures={TeX, Common}, 
  SmallCapsFeatures={Letters=SmallCaps},
  %Contextuals=WordFinal,
            ]{Cambria}
\setmonofont[Scale=MatchLowercase]{Consolas}

\newfontfamily\xsuxfont{NotoSansCuneiform-egg.ttf}
\newfontfamily\obfont{Santakku.ttf}
\newfontfamily\nafont{Assurbanipal.ttf}
\newfontfamily\nbfont{Esagil.ttf}
\newfontfamily\hantfont{NotoSerifTC-Regular.ttf}


\newfontfamily\anshufont{Uni12580ProtoCuneiformChartFont.ttf}
\newcommand\oneNone{\symbol{"12580}}
\newcommand\nineNone{\symbol{"126CB}}
\newcommand\oneNfourteen{\symbol{"12591}}
\newcommand\fiveNfourteen{\symbol{"12674}}
\newcommand\oneNthirtyFour{\symbol{"125BA}}
\newcommand\nineNthirtyFour{\symbol{"126D8}}
\newcommand\oneNfortyEight{\symbol{"125D0}}
\newcommand\fiveNfortyEight{\symbol{"12684}}
\newcommand\oneNfortyFive{\symbol{"125CA}}

\newcommand\oneNtwentyTwo{\symbol{"1259A}}

\newfontfamily\curviform{curviform.ttf}
\newcommand\oneAšC{\symbol{"E55D}} % 𒀸
\newcommand\twoAšC{\symbol{"E55E}} % 𒐀
\newcommand\threeAšC{\symbol{"E55F}} % 𒐁
\newcommand\fourAšC{\symbol{"E562}} % 𒐂
\newcommand\fiveAšC{\symbol{"E563}} % 𒐃
\newcommand\sixAšC{\symbol{"E564}} % 𒐄
\newcommand\sevenAšC{\symbol{"E565}} % 𒐅
\newcommand\eightAšC{\symbol{"E566}} % 𒐆
\newcommand\nineAšC{\symbol{"E567}} % 𒐇
\newcommand\oneUC{\symbol{"E568}}
\newcommand\twoUC{\symbol{"E569}}
\newcommand\threeUC{\symbol{"E56B}}
\newcommand\fourUC{\symbol{"E56D}}
\newcommand\fiveUC{\symbol{"E56E}}
\newcommand\sixUC{\symbol{"E56F}}
\newcommand\sevenUC{\symbol{"E570}}
\newcommand\eightUC{\symbol{"E571}}
\newcommand\nineUC{\symbol{"E572}}
\newcommand\oneŊešTwoC{\symbol{"E573}}
\newcommand\twoŊešTwoC{\symbol{"E574}}
\newcommand\threeŊešTwoC{\symbol{"E575}}
\newcommand\fourŊešTwoC{\symbol{"E578}}
\newcommand\fiveŊešTwoC{\symbol{"E579}}
\newcommand\sixŊešTwoC{\symbol{"E57A}}
\newcommand\sevenŊešTwoC{\symbol{"E57B}}
\newcommand\eightŊešTwoC{\symbol{"E57C}}
\newcommand\nineŊešTwoC{\symbol{"E57D}}
\newcommand\oneŊešʾuC{\symbol{"E57E}}
\newcommand\oneŠarʾuC{\symbol{"E58D}}
\newcommand\oneBurʾuC{\symbol{"E593}}
\newcommand\oneBanTwoC{\symbol{"E599}}
\newcommand\oneEšeThreeC{\symbol{"E5A3}}
\newcommand\oneDišC{\symbol{"E59E}}

\newfontfamily\cuneiformComposite{CuneiformComposite.ttf}
\newcommand\oneŠarTwoC{{\cuneiformComposite 𒊹}}

\usepackage{mathtools}

\usepackage{unicode-math}
\setmathfont[Scale=MatchLowercase, math-style=ISO]{Cambria Math}
\setmathfontface\unifrak{UnifrakturMaguntia.ttf}[Scale=MatchUppercase]

\usepackage[colorlinks,allcolors=Periwinkle]{hyperref}
\usepackage[backend=biber,giveninits=true,maxnames=100,style=alphabetic,maxalphanames=4,doi=true,url=false,eprint=true,labelalpha=true,dateusetime=true]{biblatex}
\addbibresource{bibliography.bib}

% Allow breaking in numbers and after lower and upper case letters in bibliography
% URLs, see https://tex.stackexchange.com/a/134281.
% We use fairly low values to avoid unsightly spacing:
% http : / / example . com is not an improvement over http://ex-
% ample.com.  We prefer breaking in numbers and uppercase letters, which are often
% IDs, rather than lowercase letters, which sometimes form meaningful words, or at
% least tokens that are not customarily broken, e.g. the protocol.
\setcounter{biburlnumpenalty}{100}
\setcounter{biburllcpenalty}{500}
\setcounter{biburlucpenalty}{100}
\renewcommand\UrlFont{\rmfamily}

\AtEveryBibitem{\clearlist{language}}  % TODO(egg): Why are we doing this?

\newcommand\UTCdoc[1]{\href{https://www.unicode.org/cgi-bin/GetMatchingDocs.pl?#1}{#1}}

\DeclareFieldFormat{doi}{%
  \newline
  \mkbibacro{DOI}\addcolon\space
    \ifhyperref
      {\href{https://doi.org/#1}{\nolinkurl{#1}}}
      {\nolinkurl{#1}}}
\DeclareFieldFormat{eprint:cnki}{%
  \newline
  \mkbibacro{CNKI}\addcolon\space
    \ifhyperref
      {\href{http://www.cnki.com.cn/Article/CJFDTOTAL-#1.htm}{\nolinkurl{#1}}}
      {\nolinkurl{#1}}}
      \DeclareFieldFormat{eprint:utc}{%
        \newline
        \mkbibacro{UTC}\addcolon\space
          \ifhyperref
            {\UTCdoc{#1}}
            {\nolinkurl{#1}}}
\DeclareFieldFormat{eprint:cdli}{%
  \newline
  \mkbibacro{CDLI}\addcolon\space
    \ifhyperref
      {\href{https://cdli.ucla.edu/#1}{\nolinkurl{#1}}}
      {\nolinkurl{#1}}}
      \DeclareFieldFormat{eprint:ebda}{%
        \newline
        EbDA\addcolon\space
          \ifhyperref
            {\href{http://ebda.cnr.it/tablet/view/#1}{\nolinkurl{#1}}}
            {\nolinkurl{#1}}}
\DeclareFieldFormat{eprint}{%
  \newline
  eprint\addcolon\space
    \ifhyperref
      {\href{#1}{\nolinkurl{#1}}}
      {\nolinkurl{#1}}}
\DeclareFieldFormat{isbn}{%
  \newline
  \mkbibacro{ISBN}\addcolon\space#1}
      
\newcommand{\idest}{\emph{i.e.}}
\newcommand{\exempligratia}{\emph{e.g.}}
\newcommand{\obverse}{obv.}
\newcommand{\reverse}{rev.}
\newcommand{\recto}{\emph{recto}}
\newcommand{\verso}{\emph{verso}}

\usepackage{multirow}

\usepackage{epigraph}
\renewcommand{\epigraphsize}{\footnotesize\itshape}
\setlength\epigraphwidth{9cm}
\setlength\epigraphrule{0pt}
\epigraphnoindent

\usepackage[dvipsnames]{xcolor}

\usepackage{enumitem}
\renewcommand\labelitemi{---}

\usepackage[style=iso]{datetime2}

\hyphenation{cunei-form}
\usepackage{microtype}

\title{Archaic cuneiform numbers}
\author{Robin Leroy, Anshuman Pandey, and Steve Tinney}

\begin{document}

\maketitle

\tableofcontents

\section{Summary}

This document proposes encoding some numerals used in the Uruk and Early Dynastic periods in conjunction
with the Sumero-Akkadian cuneiform script\footnote{ISO 15924: Xsux, Script property value long name: Cuneiform; encoded since Unicode Version 5.0.}
and the proto-cuneiform script\footnote{ISO 15924: Pcun, not yet encoded.}.
The proposed characters are listed in section~\ref{proposal}.

The non-numeric signs of proto-cuneiform will be the subject of a separate proposal;
we need only note here that the divergence between the approaches to character identity
in modern scholarship requires that proto-cuneiform be disunified from cuneiform:
proto-cuneiform is effectively treated as an undeciphered script.
In contrast, the cuneiform encoding model is semantic,
requiring an understanding of the text to correctly encode it.

However, the \emph{numerals} used in proto-cuneiform should be unified with
ones used in the Early Dynastic period, for the reasons set forth in
section~\ref{unificationRationale}.
The proposed ``curved'', or ``curviform'', numerals\footnote{{\anshufont \oneNone}–{\anshufont \nineNone} 1–9($\text{ašᶜ}=N_{1}$), {\anshufont \oneNfourteen}–{\anshufont \fiveNfourteen} 1–5($\text{uᶜ}=N_{14}$), {\anshufont \oneNthirtyFour}–{\anshufont \nineNthirtyFour} 1–9($\text{ŋeš₂ᶜ}=N_{34}$), {\anshufont \oneNfortyEight}–{\anshufont \fiveNfortyEight} 1–5($\text{ŋešʾuᶜ}=N_{48}$), etc.}
should however \emph{not} be unified with
the already-encoded cuneiform numerals\footnote{{\xsuxfont 𒀸}–{\xsuxfont 𒐇} 1–9(aš),
{\xsuxfont 𒌋}–{\xsuxfont 𒐐} 1–5(u),
{\xsuxfont 𒐕}–{\xsuxfont 𒐝} 1–9(ŋeš₂),
{\xsuxfont 𒐞}–{\xsuxfont 𒐢} 1–5(ŋešʾu), etc.}.
Since the encoding proposals for the cuneiform script
twenty years ago provisionally considered the curviform numerals
to be glyph variants of the cuneiform numerals,
a detailed rationale is provided in section~\ref{disunificationRationale},
including compatibility considerations in section \ref{compatibility}.

The overall picture of unifications and disunifications over time is illustrated in table~\ref{tableUnificationsDisunifications}.
The Script\_Extensions property assignments in section~\ref{properties} reflect the overlap.

[TODO(egg): Mention the other sections here too.]

\begin{table}
\begin{center}
\begin{tabular}{ l | l | l | l |} \cline{2-4}
                                                & Uruk III \& earlier & ED – Ur III                         & OB \& later    \\\hline
\multicolumn{1}{|c|}{\multirow{2}{*}{Numerals}} & \multicolumn{2}{|c|}{This proposal}                       &                \\\cline{2-4}
\multicolumn{1}{|c|}{}                          &                     & \multicolumn{2}{|c|}{\multirow{2}{*}{Existing Xsux}} \\\cline{1-2}
\multicolumn{1}{|c|}{Non-numeric signs}         & Future Pcun         & \multicolumn{2}{|c|}{}                               \\\hline
\end{tabular}
\caption{Usage of existing, proposed, and future characters across functions and time periods.}
\end{center}
\label{tableUnificationsDisunifications}
\end{table}

\section{Proposed changes to the Standard}
\label{proposal}
\subsection{Summary of proposed characters}
\subsection{Properties}
\label{properties}
\subsection{Character names list}
\subsection{Core specification text}

\section{Rationale for curviform--cuneiform disunification}
\label{disunificationRationale}
TODO(egg): blurb.

\subsection{The cuneiform encoding model}
\label{xsuxModel}
As outlined in, \exempligratia, \cite{UTR56}, the cuneiform encoding model is diachronic;
each character may have wildly different glyphs depending on time period and region.
For instance, the sign IM
may resemble {\cuneiformComposite 𒅎} in texts from Early Dynastic IIIa Šuruppag as in the character code
charts,
{\cuneiformComposite 𒉎} later in the third millenium\footnote{Merging with U+1224E {\xsuxfont 𒉎} NI₂.},
{\obfont 𒅎} in Old Babylonian cursive,
{\nafont 𒅎} in Neo-Assyrian, but is always encoded as U+1214E \textsc{cuneiform sign im}.

This encoding model allows for the interoperable representation of editions
of diachronic reference works such as sign lists\footnote{Notably \cite{OSL} and the online edition of \cite{Borger2010} in \cite[Signs]{eBL}.}
and dictionaries\footnote{Notably \cite{ePSD2} and the online edition of \cite{Schramm2010} in \cite[Dictionary]{eBL}.},
and of composite texts\footnote{For example, there are Neo-Assyrian and Neo-Babylonian
copies parts of the laws of {\xsuxfont 𒄩𒄠𒈬𒊏𒁉}, as well as Old Babylonian copies in both archaizing
and cursive styles.
Because of damage on the stele \cite{P249253},
some sections are known only from those copies. See \cite[110\psqq]{Oelsner2022}.}.
By being compatible with similarly diachronic transliteration practice
(that is, by avoiding distinctions finer than those made in transliteration),
the encoding model also allows for automated conversion of transliterated
corpora to cuneiform,
which has proven useful as a processing step in analyses such as
\cites{Romach24}{JauhiainenJauhiainen24}\footnote{Attendees
may recall the summary given on the third day of UTC \#180, as recorded in \cite{L2/24-159}.
Other readers may refer to \cite[242,148]{RAI69Abstracts}.}.
The diachronic approach is also useful for pedagogical applications\footnote{For instance,
Old Babylonian grammar may be taught in the Neo-Assyrian script, as in \cite{Caplice2002}.}.

\subsection{Arguments for curviform–cuneiform unification}
\label{oldArgumentsForUnification}
In this context, the argument was made in \cite{L2/04-099}, as part of discussion of the cuneiform
encoding\footnote{At that time scoped to the répertoire of the Ur III period and later, see \cite[1]{L2/03-162},
although many disunifications, such as $\text{\xsuxfont 𒅎}\ne\text{\xsuxfont 𒉎}$, were informed by Early Dynastic distinctions.}
that the curviform numerals, which occasionally appear in the Ur III period
and are used heavily in the Early Dynastic period,
were a stylistic distinction unifiable with the cuneiform digits, and that
an archaizing Ur III font or an Early Dynastic font could have curviform glyphs for
the appropriate characters.

Some co-occurrence of curviform and cuneiform digits was known and acknowledged.
\cite[3]{L2/04-099} cites \cite[62]{NissenDamerowEnglund1993}, which is a copy of \cite{P020054},
an Early Dynastic IIIb administrative tablet from Ŋirsu.
The excerpt cited, lines 1--3 of column 1 of the obverse, is as follows:
\begin{quote}
\begin{tabular}{l l l l l l l}
\xsuxfont 𒐕\footnotemark& \xsuxfont 𒌋 & \xsuxfont 𒈦&\xsuxfont 𒑍&\xsuxfont 𒄀&\xsuxfont 𒍑&\xsuxfont 𒁲\\
$1$(ŊEŠ₂) & $1$(U) & $1/2$(DIŠ) & $5$(DIŠ \emph{tenû}) & gi & us₂ & sa₂\\
\multicolumn{3}{c}{$7.5$ (ropes)} & $5$ & reed & side & equal
\end{tabular}\\
\footnotetext{As noted in \cite[466]{Powell1987}, this sign has a very short ``tail'' in this period,
so that it is wider than it is tall, and can at first seem like a large \text{\xsuxfont 𒀸} in copies.
The photos in CDLI clearly show that this is in fact a vertical wedge.}
\begin{tabular}{l l l l l}
\xsuxfont 𒌍\footnotemark& \xsuxfont 𒑎 &\xsuxfont 𒄀&\xsuxfont 𒊕&\xsuxfont 𒁲\\
$3$(U) & $6$(DIŠ \emph{tenû}) & gi & saŋ & sa₂\\
$3$(ropes) & $6$ & reed & front & equal
\end{tabular}\\
\footnotetext{Note that ED IIIb {\xsuxfont 𒌋} numerals have a somewhat
different appearance from those of the Ur III period used in this transcription;
the sign {\xsuxfont 𒌍} in \cite{P020054} looks more like Ur III {\xsuxfont 𒆳}.}
\begin{tabular}{l l l l l}
\xsuxfont 𒃷𒁉&
\curviform\oneUC&
\curviform\oneEšeThreeC&
\curviform\oneAšC&
\curviform\oneDišC\\
ašag-bi&1(BUR₃ᶜ)&1(EŠE₃ᶜ)&1(IKUᶜ)&1/2(IKUᶜ)\\
this field&
&
&
&
\end{tabular}\\\begin{flushright}
\begin{tabular}{l}
\xsuxfont 𒓺𒋛𒂵𒄰\\
tugₓ(LAK483)-si-ga-kam\footnotemark\\
deep ploughing
\end{tabular}
\end{flushright}
\footnotetext{Transliteration after \cite[8]{Lecompte2020}.}
\end{quote}
The argument made in \cite[4]{L2/04-099} is that this is comparable to a stylistic distinction such as\footnote{We
have taken the liberty of adjusting the analogy to use measures approximately equal to those in \cite{P020054},
instead of a field of five by twenty-five metres.}
\begin{quote}
$465$ metres, equal lengths\\
$198$ metres, equal widths\\
this field: $\unifrak{9,18}$ hectares, deeply ploughed
\end{quote}
where the numerals have the same structure (\cite{L2/04-099} contrasts this to the different structures
of ASCII digits and roman numerals).
That document further claims that ``the number signs do not normally carry in their individual signs the
meaning of what they are used to measure'', and that curviform and cuneiform numerals ``are not normally mixed together
in a single numerical expression'', noting the exceptions of \cites{P232278}{P232280}.
In addition, \cite[4]{L2/04-099} points out that the cuneiform numeric signs are descended from
the curviform ones (this is undisputed),
and claims there is only a small re-allocation of the function of signs
(from {\curviform \oneAšC}- to {\xsuxfont 𒁹}-numerals).
It therefore comes to the conclusion that the use of curviform
numerals should be seen as a formatting distinction,
rather than one that should be represented in plain text,
and insists that the encoding should capture the lineal
historical descent of those signs, presumably to take
advantage of the benefits of diachronic encoding described
in section~\ref{xsuxModel}.

Although they had been part of the preliminary proposal \cite{L2/03-393R},
the curviform numerals were therefore removed from \cite{L2/04-036} and \cite{L2/04-189},
which both state that ``The distinction between curved numerals and their cuneiform descendants
is treated as glyphic for the purposes of the present proposal;
this issue will need to be revisited in subsequent encoding phases.''

The time has come to revisit this issue.
As we will see in section~\ref{metrology},
numerals can only be interpreted in the context of what they measure \idest,
as part of a metrological system.
In section~\ref{earlyMetrology} we will see that in some periods:
\begin{itemize}[nosep]
  \item the functions and use of the numerals vary beyond the mere {\curviform \oneAšC}/{\xsuxfont 𒁹} switch;
  \item the contrast between curviform and cuneiform numerals is commonly used to distinguish metrological systems;
  \item some metrological systems commonly mix curviform and cuneiform in single numerical expressions.
\end{itemize}

\subsection{Metrology}
\label{metrology}
\epigraph{{\obfont 𒁾 𒊬𒊑𒉈 𒂵𒁺} \\ {\obfont 𒁾 𒀸 𒊺 𒄥𒋫 𒍠 {\nafont 𒐞} 𒄥𒂠} \\ {\obfont 𒁾 𒁹 𒂆𒋫 𒍠 𒆬 𒌋 𒈠𒈾𒂠} \\
I want to write tablets: the tablet of 1 \emph{gur} of barley to
600 \emph{gur}; the tablet of 1 shekel of silver to 10 minas […]}{Edubbaʾa D}

Before diving into the usage of the curviform numerals
in the Early Dynastic period to explain the constrast
with cuneiform numerals, it is useful to understand
the usage of the already-encoded characters in the
Ur III and Old Babylonian periods.

As is well known\footnote{See, \exempligratia, \cite[Section 22.3.3 ``Non-Decimal Radix Systems'',
\emph{sub} ``\href{https://www.unicode.org/versions/Unicode16.0.0/core-spec/chapter-22/\#G42894}{Cuneiform Numerals}'']{Unicode16}.}
a sexagesimal place value system (SPVS) was used in Meso\-potamia from the late third millenium onwards.
One should bear in mind, however, that other systems were used;
the SPVS was primarily used in calculations,
with results being expressed in non-positional systems \cites[76]{Robson2008}{Robson2022}.
The digits 1–59 of the SPVS have inner structure which is reflected in the encoding: the digits 1–9 are the individual
characters {\xsuxfont 𒁹}–{\xsuxfont 𒑆}, the multiples of ten (10–50) are {\xsuxfont 𒌋}–{\xsuxfont 𒐐},
but the other digits 11–59 are sequences {\xsuxfont 𒌋𒁹}–{\xsuxfont 𒐐𒑆};
in effect the base-sixty digits are themselves written in base ten, with a different set of symbols for the tens place.
This reflects the origin of the sexagesimal place value system;
it derives from a \emph{non-positional} system, hereafter the \emph{cuneiform discrete counting system} \ref{systemSOB},
which had different signs for the units {\xsuxfont 𒁹}–{\xsuxfont 𒑆},
tens {\xsuxfont 𒌋}–{\xsuxfont 𒐐}, sixties {\xsuxfont 𒐕}–{\xsuxfont 𒐝} (with larger wedges
than the units), six hundreds {\xsuxfont 𒐞}–{\xsuxfont 𒐢},
three thousand six hundreds {\xsuxfont 𒊹}–{\xsuxfont 𒐫}, and thirty-six thousands
 {\xsuxfont 𒐬}–{\xsuxfont 𒐱}.

 \subsubsection{The discrete counting system} 
The relations between the values of the signs in the cuneiform discrete counting system
may be summarized by the following factor diagram\footnote{These diagrams,
which have become standard in discussions of Mesopotamian metrology, originate with \cite[10]{Friberg1978},
where they are called \emph{step-diagrams}.},
where the number over arrow indicates the multiple
of the preceding sign (right of the arrow) corresponding to the following sign (left).
\begin{equation}
\text{\xsuxfont 𒐬} \xleftarrow{10}\text{\xsuxfont 𒊹} \xleftarrow{6}\text{\xsuxfont 𒐞} \xleftarrow{10}\text{\xsuxfont 𒐕} \xleftarrow{6}\text{\xsuxfont 𒌋}\xleftarrow{10}\text{\xsuxfont 𒁹}
\tag{$S_{\text{Ur III/OB}}$}
\label{systemSOB}
\end{equation}
For example, the number $1729=((\textcolor{BrickRed}{2}\times 10 + \textcolor{Dandelion}{8})\times 6 + \textcolor{NavyBlue}{4})\times 10 + \textcolor{Orchid}{9} = 28\times 60 + 49$
would be written {\xsuxfont \textcolor{BrickRed}{𒐟}\textcolor{Dandelion}{𒐜}\textcolor{NavyBlue}{𒐏}\textcolor{Orchid}{𒑄}} in the discrete counting system,
and {\xsuxfont 𒎙𒑄𒐏𒑆} in the sexagesimal place value system.

\subsubsection{The area system}
The discrete counting system was not the only non-positional system in use in the Ur III and Old Babylonian periods; different systems were in use depending on what was being counted or measured.
For instance, field areas were measured using the following system, where for the named
units we have provided the name of the unit in transliterated Sumerian, normalized Old Babylonian Akkadian,
and the approximate metric equivalent \cites[378]{Friberg2007}{Robson2019}:
\begin{equation}
\text{\xsuxfont 𒐬}
\xleftarrow{10}\text{\xsuxfont 𒊹}
\xleftarrow{6}\text{\xsuxfont 𒐴}
\xleftarrow{\ 10\ }\underset{\mathclap{\substack{1~\text{bur₃}\\1~\text{\emph{būrum}}\\6,48~\text{ha}}}}{\text{\xsuxfont 𒌋}}
\xleftarrow{\quad3\quad}\underset{\mathclap{\substack{1~\text{eše₃}\\1~\text{\emph{eblum}}\\2,16~\text{ha}}}}{\text{\xsuxfont 𒑘}}
\xleftarrow{\quad6\quad}\underset{\mathclap{\substack{1~\text{iku}\\1~\text{\emph{ikûm}}\\3600~\text{m}^2}}}{\text{\xsuxfont 𒀸}}
\xleftarrow{\quad2\quad}\underset{\mathclap{\substack{1~\text{\emph{ubûm}}\\1800~\text{m}^2}}}{\text{\xsuxfont 𒀹}}
\xleftarrow{\ 2\ }\text{\xsuxfont 𒑠}
\tag{$G_{\text{Ur III/OB}}$}
\label{systemGOB}
\end{equation}
Note that for the range of areas given above\footnote{For
areas smaller than a quarter \emph{ikûm}, an overt unit is used,
with $1~\text{\emph{mūšarum}}$ ($36~\text{m}^2$) written {\xsuxfont 𒁹𒊬}, equal to one hundredth of an \emph{ikûm},
then sexigesimally subdivided in $60~\text{\xsuxfont{𒂆}}$ (shekels).
For areas greater than $3600~\text{\emph{būrū}}$,
the {\xsuxfont 𒊹}- and {\xsuxfont 𒐬}-numerals are reused with a suffix {\xsuxfont 𒃲} (gal, Sumerian: big),
as follows \cites[\pno 295 with notes b and c]{Robson2008}[378]{Friberg2007}{Robson2019}: \[
\underbrace{\text{\xsuxfont 𒐬𒃲}
\xleftarrow{10}\text{\xsuxfont 𒊹𒃲}
\xleftarrow{6}\text{\xsuxfont 𒐬}
\xleftarrow{10}\text{\xsuxfont 𒊹}
\xleftarrow{6}\text{\xsuxfont 𒐴}
\xleftarrow{10}\text{\xsuxfont 𒌋}
\xleftarrow{3}\text{\xsuxfont 𒑘}
\xleftarrow{6}\text{\xsuxfont 𒀸}
\xleftarrow{2}\text{\xsuxfont 𒀹}
\xleftarrow{2}\text{\xsuxfont 𒑠}}_{\text{\footnotesize\xsuxfont 𒃷}}
\xleftarrow{2,5}\underbrace{\text{\xsuxfont 𒌋}
\xleftarrow{10}\text{\xsuxfont 𒁹}}_{\text{\footnotesize\xsuxfont 𒊬}}
\xleftarrow{6}\underbrace{\text{\xsuxfont 𒌋}
\xleftarrow{10}\text{\xsuxfont 𒁹}}_{\text{\footnotesize\xsuxfont 𒂆}}\text.
\]},
this system does not use any symbols separate from the numerals
for the individual units (\emph{ubûm}, \emph{ikûm}, \emph{eblum}, and \emph{būrum}).
As mentioned in \cite{Robson2019}, the whole numeric expression for the area would be followed by the sign {\xsuxfont 𒃷}
functioning as punctuation\footnote{TODO(egg): acknowledge Proust 2020
  but note that this is irrelevant to encoding concerns}, but the numerals are tied to the metrology; thus
a surface of $5~\text{\emph{būrū}}$ $1~\text{\emph{eblum}}$ $4~\text{\emph{ikû}}$ ($100~\text{\emph{ikû}}$, $36~\text{ha}$) would be written\footnote{As in the surface of the field of {\xsuxfont 𒀀𒅗𒋡𒆠} (the city of Apisal) reported on \cite[r.~1]{P102305}}
{\xsuxfont 𒐐𒑘𒐂𒃷}. Contrast this with systems
where the same numerals are used for different units,
and overt units are used, as in ``88 acres 3 roods 33 perches''.
Note also that the same signs are shared between multiple systems,
with different relations; the ŠAR₂ sign {\xsuxfont 𒊹} is equal to sixty times the U sign {\xsuxfont 𒌋}
in the area system, but to three hundred and sixty times {\xsuxfont 𒌋} in the discrete counting system.

\subsubsection{The capacity system}
Another such system of note is the one for capacities\footnote{Used
for volumes of grain, but also oil, dairy products, beer, etc., as well as to express the capacity of boats;
volumes of earthworks instead use system \ref{systemGOB} based on a height of one cubit, see\cites[488]{Powell1987}[294]{Robson2008}{Robson2019}.} \cites[376]{Friberg2007}{Robson2019},
\begin{equation}
\underbrace{
%\text{(as in \ref{systemSOB})}
\text{\xsuxfont 𒐬} \xleftarrow{10}\text{\xsuxfont 𒊹} \xleftarrow{6}\text{\xsuxfont 𒐞} \xleftarrow{10}\text{\xsuxfont 𒐕}
\xleftarrow{6}\text{\xsuxfont 𒌋}
\xleftarrow{10}\underset{\mathclap{\substack{1~\text{gur}\\1~\text{\emph{kurrum}}}}}{\text{\xsuxfont 𒀸}}}_{\text{\xsuxfont 𒄥}}
\xleftarrow{\quad5\quad}\underset{\mathclap{\substack{1~\text{bariga}\\1~\text{\emph{parsiktum}}}}}{\text{\xsuxfont 𒁹}}
\xleftarrow{\quad6\quad}\underset{\mathclap{\substack{1~\text{ban₂}\\1~\text{\emph{sūtum}}}}}{\text{\xsuxfont 𒑏}}
\xleftarrow{\ 10\ }\underset{\mathclap{\substack{1~\text{sila₃}\\1~\text{\emph{qûm}}\\1~\text{l}}}}{\text{\xsuxfont 𒁹𒋡}}\text,
\tag{$C_{\text{Ur III/OB}}$}
\label{systemC}
\end{equation}
where the numerals for ban₂ are {\xsuxfont 𒑏}, {\xsuxfont 𒑐}, {\xsuxfont 𒑑}, {\xsuxfont 𒑒},
and {\xsuxfont 𒑔}, and those for bariga are {\xsuxfont 𒁹}, {\xsuxfont 𒑖}, {\xsuxfont 𒑗}, and {\xsuxfont 𒐉} (contrast
ordinary {\xsuxfont 𒈫} and {\xsuxfont 𒐈} otherwise used with {\xsuxfont 𒁹}-numerals).
As described in \cite[\pno 585 with notes (b) and (f)]{Huehnergard2011},
the sign GUR {\xsuxfont 𒄥}, while it is used only with volumes in excess of one gur,
is written after the whole expression,
after the overt unit sign {\xsuxfont 𒋡} if present, and after the word for ``grain'' if present, as in
\[\begin{matrix}
\text{\xsuxfont 𒐢𒐝𒌋𒐂}&
\text{\xsuxfont 𒑑}&
\text{\xsuxfont 𒐋}&
\text{\xsuxfont 𒋡}&
\text{\xsuxfont 𒊺}&
\text{{\xsuxfont 𒄥}\footnotemark}\\
3554~\text{gur}&
3~\text{ban₂}&
6&\text{sila₃}&
\text{ of grain.}
\end{matrix}\]
\footnotetext{From \href{https://cdli.mpiwg-berlin.mpg.de/artifacts/309594}{P309594}.}
Observe that while large numbers of gur follow\footnote{A larger unit, the guru₇ (\emph{karûm}, grain heap), is sometimes used instead, with {\xsuxfont 𒀸𒄦}={\xsuxfont 𒊹𒄥} ($1~\text{\emph{karûm}}=3600~\text{kurrū}$). See \cites[415]{Friberg2007}{Robson2019}.}
system \ref{systemSOB},
the use of horizontal (AŠ) numerals for the gur disambiguates from the vertical bariga,
as {\xsuxfont 𒌋𒁹𒄥} would be $10$~gur $1$~bariga, and {\xsuxfont 𒌋𒀸𒄥} would be $11$~gur;
again even with some overt units, most of the numerals
that participate in a metrological system have an interpretation
dependent on that system. To quote \cite[78]{Robson2008}:
``The SPVS temporarily changed the status of numbers from properties of real-world objects to independent entities that could be manipulated without regard to […] metrological system. […] Once the calculation was done, the result was expressed in the most appropriate metrological units and thus re-entered the natural world as a concrete quantity.''

This intertwining of units and numerals explains the large number of already-encoded numeral series:
\begin{itemize}[nosep]
\item {\xsuxfont 𒁹}–{\xsuxfont 𒑆} used in \ref{systemSOB} and the SPVS as well as with overt units;
\item {\xsuxfont 𒌋}–{\xsuxfont 𒐔} used in \ref{systemGOB}, of which {\xsuxfont 𒌋}–{\xsuxfont 𒐐} are also used in \ref{systemSOB} and the SPVS as well as with overt units;
\item {\xsuxfont 𒐕}–{\xsuxfont 𒐝} used in \ref{systemSOB}, and sometimes with overt units;
\item {\xsuxfont 𒐞}–{\xsuxfont 𒐢} used in \ref{systemSOB};
\item {\xsuxfont 𒊹}–{\xsuxfont 𒐫} used in \ref{systemSOB} and \ref{systemGOB};
\item {\xsuxfont 𒐬}–{\xsuxfont 𒐱} used in \ref{systemSOB} and \ref{systemGOB};
\item {\xsuxfont 𒀸}–{\xsuxfont 𒐇} used in \ref{systemC} as well as with overt units of the weight system;
\item {\xsuxfont 𒑏}, {\xsuxfont 𒑐}, {\xsuxfont 𒑑}, {\xsuxfont 𒑒}, {\xsuxfont 𒑔} used in \ref{systemC};
\item {\xsuxfont 𒁹}, {\xsuxfont 𒑖}, {\xsuxfont 𒑗}, {\xsuxfont 𒐉} used in \ref{systemC}—note the overlap with {\xsuxfont 𒁹}–{\xsuxfont 𒑆};
\item {\xsuxfont 𒑘} and 	{\xsuxfont 𒑙} used in \ref{systemGOB}.
\end{itemize}

\subsubsection{The length system}
In the Ur III and Old Babylonian periods, lengths are expressed using overt units counted with
{\xsuxfont 𒁹}-  and {\xsuxfont 𒌋}-numerals with their system \ref{systemSOB} values\footnote{Adjacent
units are no more than a factor of $60$ apart, so higher numerals such as {\xsuxfont 𒐞} or {\xsuxfont 𒊹} are not
used.}.
Since it does not have any unusual numerals,
this system would not in itself be of much relevance to character encoding,
but we present it here as background for
its Early Dynastic counterpart presented in section~\ref{earlyMetrology}.
Metrological tables use the following units \cites[118]{Friberg2007}{Robson2019}:
\begin{equation}
  \underset{\substack{\text{danna}\\\text{\emph{bērum}}\\\text{league}\\10,8~\text{km}}}{\text{\xsuxfont 𒁹𒆜𒁍}}
  \xleftarrow{60}\underset{\substack{\text{UŠ\footnotemark}\\\\\text{cable}\\360~\text{m}}}{\text{\xsuxfont 𒁹𒍑}}
  \xleftarrow{10}\underset{\substack{\text{nindan}\\\text{\emph{nindanum}}\\\text{rod}\\6~\text{m}}}{\text{\xsuxfont 𒁹𒃻}}
  \xleftarrow{12}\underset{\substack{\text{kuš₃}\\\text{\emph{ammatum}}\\\text{cubit}\\50~\text{cm}}}{\text{\xsuxfont 𒁹𒌑}}
  \xleftarrow{30}\underset{\substack{\text{šu-si}\\\text{\emph{ubānum}}\\\text{finger}\\17~\text{mm}}}{\text{\xsuxfont 𒁹𒋗𒋛}}\text.
  \tag{$L_{\text{Ur III/OB}}$}
  \label{systemLOB}
\end{equation}
\footnotetext{TODO}
Two more units appear occasionally \cites[459]{Powell1987}[118]{Friberg2007}{Robson2019}:
\begin{equation}
  \text{\xsuxfont 𒁹𒆜𒁍}
  \xleftarrow{30}\text{\xsuxfont 𒁹𒍑}
  \xleftarrow{6}\underset{\substack{\text{eše₂}\\\text{\emph{ašlum}}\\\text{rope}\\60~\text{m}}}{\text{\xsuxfont 𒁹𒂠}}
  \xleftarrow{10}{\text{\xsuxfont 𒁹𒃻}}
  \xleftarrow{2}\underset{\substack{\text{gi}\\\text{\emph{qânum}}\\\text{reed}\\3~\text{m}}}{\text{\xsuxfont 𒁹𒄀}}
  \xleftarrow{6}{\text{\xsuxfont 𒁹𒌑}}
  \xleftarrow{30}{\text{\xsuxfont 𒁹𒋗𒋛}}\text.
  \tag{$\bar{L}_{\text{Ur III/OB}}$}
  \label{systemLbarOB}
\end{equation}
In addition, there are Akkadian names for the half-rope and half-reed,
see \cites[463\psq]{Powell1987}.
\subsubsection{Fractions}
TODO

\subsection{Early metrology}
\label{earlyMetrology}

At first sight, the metrological systems from the Early Dynastic period match the
ones previously mentioned.
In particular, the discrete counting system used in the Early Dynastic period
(and earlier in the Uruk period) clearly mirrors system \ref{systemSOB}
\cites[374]{Friberg2007}[127,165]{DamerowEnglund1987}:
\begin{equation}
\text{\curviform \oneŠarʾuC}
\xleftarrow{10}\text{\oneŠarTwoC}
\xleftarrow{6}\text{\curviform \oneŊešʾuC}
\xleftarrow{10}\text{\curviform \oneŊešTwoC}
\xleftarrow{6}{\text{\curviform \oneUC}}
\xleftarrow{10}{\text{\curviform \oneAšC}}\text.
\tag{$S$}
\label{systemS}
\end{equation}
Likewise the area system used in the Early Dynastic IIIb period mirrors system \ref{systemGOB}
\cites[72]{Deimel1922}[63]{NissenDamerowEnglund1993}[378]{Friberg2007}{Gombert2016}:
\begin{equation}
\text{\curviform \oneŠarʾuC}
\xleftarrow{10}\text{\oneŠarTwoC}
\xleftarrow{6}\text{\curviform \oneBurʾuC}
\xleftarrow{10}{\text{\curviform \oneUC}}
\xleftarrow{3}{\text{\curviform \oneEšeThreeC}}
\xleftarrow{6}\text{\curviform \oneAšC}\text,
\tag{$G_{\text{ED IIIb}}$}
\label{systemGED}
\end{equation}

As noted in \cite[4]{L2/04-099} (see section~\ref{oldArgumentsForUnification}), the vertical
{\xsuxfont 𒁹} from \ref{systemSOB} becomes a horizontal {\curviform \oneAšC} in system \ref{systemS}.
It is however far from the only case of such a reallocation of function.
The earlier form of System G was \cites[141,165]{DamerowEnglund1987}[378]{Friberg2007}:
\begin{equation}
\text{\oneŠarTwoC}
\xleftarrow{6}\text{\curviform \oneŠarʾuC}
\xleftarrow{10}{\text{\curviform \oneUC}}
\xleftarrow{3}{\text{\curviform \oneEšeThreeC}}
\xleftarrow{6}\text{\curviform \oneAšC}\text,
\tag{$G$}
\label{systemG}
\end{equation}
Observe that, as noted in \cites[142]{DamerowEnglund1987}, {\curviform \oneŠarʾuC} changes meaning from $10 \text{\curviform \oneUC}$ in system \ref{systemG} to $10 \text{\curviform \oneUC}$ in system \ref{systemGED}.
System \ref{systemG} is used in the Uruk period, but also
in the ED I–II period (it is the ``area 2'' system in \cite{Chambon2003},
whereas \ref{systemGED} is the ``area 1'' system).

\subsubsection{Field lengths in Ŋirsu}
The length system Early Dynastic IIIb of the state of Lagaš is of particular interest.
As described in \cites[466]{Powell1987}[289\psq]{Lecompte2020}, lengths are expressed in rods,
but the unit sign {\xsuxfont 𒃻} is generally omitted; in addition, only tens of rods
are used; these are equal to one rope, but the sign {\xsuxfont 𒂠} is not written either.
Length shorter than one rope are expressed in half-ropes
using the $1/2$ sign {\xsuxfont 𒈦} (again with no {\xsuxfont 𒂠}),
and then in reeds, \emph{with} the sign {\xsuxfont 𒄀}.
Effectively, this yields the following factor diagram:
\begin{equation}
  \text{\xsuxfont 𒐕}
  \xleftarrow{\quad6\quad}\underset{\mathclap{\substack{1~\text{eše₂}=10~\text{nindan}\\1~\text{rope}=10~\text{rods}\\60~\text{m}}}}{\text{\xsuxfont 𒌋}}
  \xleftarrow{\quad2\quad}\text{\xsuxfont 𒈦}
  \xleftarrow{10}\underset{\substack{\text{gi}\\\text{reed}\\3~\text{m}}}{\text{\xsuxfont 𒀹𒄀}}\text.
  \tag{$L_{\text{ED IIIb}}$}
  \label{systemLED}
\end{equation}
This is the system that was used to express the sides of the field in
\cite{P020054} discussed in section~\ref{oldArgumentsForUnification}.
In that tablet and others from the same period, such as the ones, areas are expressed in
system \ref{systemGED}, with curviform numerals\footnote{TODO(egg):
Note the handful of late Urukagina tablets that start to have cuneiform areas.};
in the absence of overt units, such as when dealing with length that are
integer multiples of a half-rope\footnote{This is the case of the sides of the
field in \cite[\obverse~ii~2--3]{P020054}.},
the use of curviform or cuneiform numerals therefore disambiguates
a numeric expression between an area and a length,
and therefore the interpretation of its
numerals between systems \ref{systemGED} and \ref{systemLED}.
The sign GAN₂ {\xsuxfont 𒃷},
which would also disambiguate the interpretation as an area,
is sometimes used after areas in ED IIIb Lagaš, but not systematically;
in particular the area of the first field in \cite{P020054} does not use this suffix.
See \cite{Lecompte2020} for many examples with and without {\xsuxfont 𒃷}.

\subsubsection{Dyke lengths in Ŋirsu}
\cites[466]{Powell1987} notes that reeds ``are regularly written with the normal,
cuneiform end of the stylus. Higher units are usually written with the reversed
(round) end of the stylus.''
Powell does not elaborate on the specifics of this mixed use of numerals,
but a cursory search in CDLI finds many occurrences\footnote{A search for
curviform numerals followed by some number of reeds counted in (tilted)
cuneiform numerals currently finds 125 occurrences
across 47 tablets.}, such as:
\begin{itemize}[nosep]
  \item \cite[\obverse~1, 4]{P221305} {\curviform\twoAšC\xsuxfont 𒂠𒑍𒄀𒌑𒑌𒋗𒆕𒀀𒑊}% 2(asz@c) esz2 5(disz@t) gi kusz3 4(disz@t) szu-du3-a 2(disz@t)
  \item \cite[\reverse~1, 1]{P221305} {\curviform\oneŊešTwoC\fourUC\oneBanTwoC\xsuxfont 𒂠𒑍𒄀} % 1(gesz2@c) 4(u@c) 1/2(asz@c) esz2 5(disz@t) gi
  \item \cite[\reverse~2, 1]{P020129} {\curviform\sixŊešTwoC\threeUC\xsuxfont 𒃻𒁺\curviform\oneBanTwoC\xsuxfont 𒑌𒄀} % 6(gesz2@c) 3(u@c) {ninda}nindax(DU) 1/2(asz@c) 4(disz@t) gi
  \item \cite[\reverse~5, 1]{P221291} {\curviform\oneŊešʾuC\twoŊešTwoC\fourUC\oneBanTwoC\xsuxfont 𒂠𒃻𒁺𒑊𒄀𒌑𒑋} % 1(gesz'u@c) 2(gesz2@c) 4(u@c) 1/2(asz@c) esz2 {ninda}nindax(DU) 2(disz@t) gi kusz3 3(disz@t)
  \item \cite[\reverse~2, 1]{P221266} {\curviform\oneAšC\oneBanTwoC\xsuxfont 𒂠𒑋𒄀} % 1(asz@c) 1/2(asz@c) esz2 3(disz@t) gi
\end{itemize}
These expressions use an
explicit sign {\xsuxfont 𒃻𒁺} (counted in multiples of ten) or
{\xsuxfont 𒂠}.
This notation—but not its use of curviform numerals—is
remarked on in \cite[\pno 290 with note 27]{Lecompte2020},
which cites several of the instances listed above.
It seems to be typical of texts about dykes.
These match the following factor diagrams:
\begin{equation}
  \underbrace{\text{\curviform\oneAšC}
  \xleftarrow{2}\text{\curviform \oneBanTwoC}}_{\text{\xsuxfont 𒂠}}
  \xleftarrow{10}\text{\xsuxfont 𒀹𒄀}
  \xleftarrow{6}\text{\xsuxfont 𒌑𒀹}
  \xleftarrow{3}\text{\xsuxfont 𒋗𒆕𒀀𒀹}\text.
  \tag{$L'_{\text{ED IIIb}}$}
  \label{systemLpED}
\end{equation}
\begin{equation}
  \underbrace{
  \text{\curviform \oneŊešʾuC}
  \xleftarrow{10}\text{\curviform \oneŊešTwoC}
  \xleftarrow{6}\text{\curviform\oneUC}}_{\text{\xsuxfont 𒃻𒁺}}
  \xleftarrow{2}\text{\curviform \oneBanTwoC\xsuxfont 𒂠\footnotemark}
  \xleftarrow{10}\text{\xsuxfont 𒀹𒄀}\text.
  \tag{$L''_{\text{ED IIIb}}$}
  \label{systemLppED}
\end{equation}
\footnotetext{TODO(egg): Note that one unit may be omitted if the other is present}

\subsubsection{Grain in Ebla}
Lengths of Early Dynastic IIIb dykes from Ŋirsu are far from the
only numeric expressions that mix curviform and cuneiform numerals.

The system of grain\footnote{Liquid capacities use a different system \cite[\pno~229 with note 12]{Archi2015}:\begin{equation*}
  \underset{\text{la-ḫa}}{\text{\xsuxfont 𒆷𒄩}}
  \xleftarrow{30}\underset{\text{sila₃}}{\text{\xsuxfont 𒋡}}
  \xleftarrow{6}\underset{\text{an-zamₓ}}{\text{\xsuxfont 𒀭𒍡}}\text.
\end{equation*}
At a glance it seems that {\xsuxfont 𒋡} are counted with cuneiform numerals and higher units
with curviform ones, thus
\begin{equation*}
  \underbrace{\text{\curviform \oneAšC\xsuxfont 𒈪𒀜}
  \xleftarrow{\frac{5}{3}}\text{\curviform \oneŊešTwoC}
  \xleftarrow{6}\text{\curviform \oneUC}
  \xleftarrow{10}\text{\curviform\oneAšC}}_{\text{\xsuxfont 𒆷𒄩}}
  \xleftarrow{3}\underbrace{\text{\xsuxfont 𒌋}
  \xleftarrow{10}\text{\xsuxfont 𒁹}}_{\text{\xsuxfont 𒋡}}
  \xleftarrow{6}\text{\xsuxfont 𒁹𒀭𒍡}\text,
\end{equation*}
but we have not investigated this thoroughly.} capacities in Ebla uses the following units\footnote{TODO mention the other one citing Chambon and the footnote in Archi}:
\begin{equation*}
  \underset{\text{gu₂-bar}}{\text{\xsuxfont 𒄘𒁇}}
  \xleftarrow{2}\underset{\text{ba-ri₂-zu}}{\text{\xsuxfont 𒁀𒌷𒍪}}
  \xleftarrow{\frac{5}{2}}\underset{\text{ŋin₄}}{\text{\xsuxfont 𒂆}}
  \xleftarrow{4}\underset{\text{niŋ₂-sagšu}}{\text{\xsuxfont 𒃻𒌋𒊕}}
  \xleftarrow{6}\underset{\text{an-zamₓ}}{\text{\xsuxfont 𒀭𒍡}}\text.
\end{equation*}
The {\xsuxfont 𒄘𒁇} and {\xsuxfont 𒁀𒌷𒍪} are generally counted using curviform numerals,
and the smaller units using cuneiform {\xsuxfont 𒁹} numerals.
Indeed, a search on \cite{EbDA} for co-occurrences of {\xsuxfont 𒀭𒍡} with either of {\xsuxfont 𒄘𒁇} or {\xsuxfont 𒁀𒌷𒍪}
finds the following expressions\footnote{We cite here only one attestation per tablet;
most tablets contain several expressions mixing curviform {\xsuxfont 𒁀𒌷𒍪} and larger with cuneiform {\xsuxfont 𒂆} and smaller.
In all cases the transcriptions given here are based on the EbDA transliterations, but the
shape and orientation of the numerals was checked\footnotemark on a photograph (from EbDA unless noted otherwise).}:
\footnotetext{As we will see in
Section~\ref{transliteration}, CDLI transliterations indicate numeral shape; however,
as of this writing, they do so incorrectly on the Ebla corpus, claiming that all numerals are curviform,
so we were not able to rely on them in this specific case.}
\begin{enumerate}[nosep]
  \item \cite[\href{http://ebda.cnr.it/tablet/view/1324\#58045}{\verso~4, 9}]{P240532} {\curviform\oneAšC\xsuxfont 𒊺 𒁀𒊑𒅗\footnote{ba-ri₂-zu₂, a variant spelling.}𒐊𒃻𒌋𒊕}% v.4,9 1 še ba-ri₂-zu₂ 5 nig₂-sagšu % ARET 09 0010 
  \item \cite[\href{http://ebda.cnr.it/tablet/view/1350\#59630}{\verso~1, 1}]{P240548} {\curviform\oneAšC\xsuxfont 𒁀𒊑𒍪 𒐈𒀭𒍡} % ARET 09 0036 1 ba-ri₂-zu 3 an-zamᵪ
  \item \cite[\href{http://ebda.cnr.it/tablet/view/1358\#60317}{\recto~7, 9}]{P240655} {\curviform\twoAšC\xsuxfont 𒊺𒁇\footnote{Short for {\xsuxfont 𒄘𒁇}.}𒐊𒃻𒌋𒊕} % r.7,9	2 še bar 5 nig₂-sagšu ARET 09 0044
  \item \cite[\href{http://ebda.cnr.it/tablet/view/1364\#60805}{\verso~4, 3}]{P240579} {\curviform\fiveAšC\oneDišC\xsuxfont 𒄘𒁇 𒈫𒃻𒌋𒊕}% 5-1/2 gu₂-bar 2 nig₂-sagšu ARET 09 0050
  \item \cite[\href{http://ebda.cnr.it/tablet/view/1371\#61227}{\verso~2, 2}]{P240675} {\curviform\oneAšC\xsuxfont 𒊺 𒁀𒌷𒍪 𒐊𒃻𒌋𒊕}% v.2,2  1 še ba-ri₂-zu 5 nig₂-sagšu ARET 09 0057
  \item \cite[\href{http://ebda.cnr.it/tablet/view/1378\#61605}{\verso~3, 1}]{P240609} {\curviform\oneAšC\xsuxfont 𒁀𒌷𒍪 𒐈𒀭𒍡}% ARET 09 0064 1 ba-ri₂-zu 3 an-zamᵪ
  %\item % r.3,3	1 ba-ri₂-zu 8 nig₂-sagšu ninda ARET 09 0064 http://ebda.cnr.it/tablet/view/1378#61586
  \item \cite[\href{http://ebda.cnr.it/tablet/view/1379\#61615}{\recto~3, 3}]{P240533} {\curviform\twoUC\oneAšC\oneDišC\xsuxfont 𒄘𒁇 𒐌𒃻𒌋𒊕 𒈫𒈦𒀭𒍡𒍝}  % ARET 09 0065 #61615 20-1-1/2 gu₂-bar 7 nig₂-sagšu 2-1/2 an-zamᵪ za
  \item \cite[\href{http://ebda.cnr.it/tablet/view/1381\#61931}{\recto~1, 5}]{P240697} {\curviform\oneAšC\oneDišC\footnote{Note the omitted {\xsuxfont 𒄘𒁇}.}\xsuxfont 𒈫𒂆 𒐈𒀭𒍡}% ARET 09 0067 r.1,5	1-1/2 ❮gu₂-bar❯ 2 GIN₂ 3 an-zamᵪ
  \item \cite[\href{http://ebda.cnr.it/tablet/view/1382\#62088}{\recto~6, 2}]{P240653} {\curviform\twoUC\threeAšC\oneDišC\xsuxfont 𒄘𒁇\curviform\xsuxfont 𒁹𒃻𒌋𒊕 𒈫𒈦𒀭𒍡} % ARET 09 0068  20-3-1/2 gu₂-bar 1 nig₂-sagšu 2-1/2 an-zamᵪ  (also has 1 ba-ri₂-zu₂ 1 GIN₂ elsewhere)
  \item \cite[\href{http://ebda.cnr.it/tablet/view/1383\#62337}{\recto~2, 6}]{P240654} {\curviform\oneAšC\xsuxfont 𒁀𒌷𒍪 𒐋𒋡\footnote{Instead of the expected {\xsuxfont 𒃻𒌋𒊕}.} 𒐈𒀭𒍡\footnote{{\xsuxfont 𒐈𒀭𒍡} not legible on the EbDA photo.}}% 1 ba-ri₂-zu 6 sila₃ 3 an-zamᵪ % ARET 09 0069
  \item \cite[\href{http://ebda.cnr.it/tablet/view/1415\#63792}{\recto~1, 8}]{P240531} {\curviform\oneAšC\xsuxfont 𒁀𒌷𒍪 𒐋𒃻𒌋𒊕 𒊺𒍥𒄖}% r.1,8	1 ba-ri₂-zu 6 nig₂-sagšu še-zid₂-gu ARET 09 101 
  \item \cite[\href{http://ebda.cnr.it/tablet/view/3173\#154945}{\recto~1, 1}]{P241708}\footnote{From CDLI photo.} {\curviform\twoAšC\oneDišC\xsuxfont 𒆗 𒄘𒁇 𒐌𒃻𒌋𒊕}% 2 1/2 sig₁₅ gu₂-bar 7 nig₂-sagšu  EST 50 https://cdli.mpiwg-berlin.mpg.de/artifacts/241708/reader/133004
  \item \cite[\href{http://ebda.cnr.it/tablet/view/3183\#155340}{\recto~1, 1}]{P241904}\footnote{From photo in \cite[6]{Archi1989}.} {\curviform\fourAšC\xsuxfont 𒄘𒁇 𒐉\footnote{Laid out as {\xsuxfont 𒁹𒁹𒁹𒁹}; on stacking patterns see Section~\ref{stackingPatterns}.}𒃻𒌋𒊕}% EST 60 = RA 083, 001, fig. 1
\end{enumerate}
  %\item \cite[\obverse iii 2]{P240964} {\curviform\sixAšC\xsuxfont 𒇷\curviform\sixŊešTwoC\oneDišC\xsuxfont 𒄘𒁇\rotatebox[origin=c]{270}{\curviform\twoAšC}\xsuxfont 𒋙𒃻𒈫𒀭𒍡} [TODO comment or stick it in its own paragraph] % EbDA http://ebda.cnr.it/tablet/view/3184#155353, EST 061.
Note that higher numbers of {\text{\xsuxfont 𒄘𒁇}} are expressed in hundreds (\emph{mi-at} {\xsuxfont 𒈪𒀜})
and then thousands (\emph{li-im} {\xsuxfont 𒇷𒅎}),
as is typical in Ebla \cite[33]{Archi2015},
\exempligratia, in \cite[\href{https://ebda.cnr.it/tablet/view/1324\#58019}{\verso~2, 3}]{P240532},
{\curviform\oneAšC\xsuxfont 𒈪𒀜\curviform\oneŊešTwoC\threeUC\fiveAšC\xsuxfont 𒊺 𒄘𒁇}
($100+60+30+5=195$~{\xsuxfont 𒄘𒁇} of grain). %1 mi-at 60-30-5 še gu₂-bar
These expressions match the following factor diagram:
\begin{equation}
  \underbrace{\text{\curviform \oneAšC\xsuxfont 𒈪𒀜}
  \xleftarrow{\frac{5}{3}}\text{\curviform \oneŊešTwoC}
  \xleftarrow{6}\text{\curviform \oneUC}
  \xleftarrow{10}\text{\curviform\oneAšC}
  \xleftarrow{2}\text{\curviform\oneDišC}}_{\text{\xsuxfont 𒄘𒁇}}=\text{\curviform\oneAšC\xsuxfont 𒁀𒊑𒍪}
  \xleftarrow{\frac{5}{2}}\text{\xsuxfont 𒁹𒂆}
  \xleftarrow{4}\text{\xsuxfont 𒁹𒃻𒌋𒊕}
  \xleftarrow{6}\text{\xsuxfont 𒁹𒀭𒍡}
  \tag{$C_{\text{Ebla}}$}
  \label{systemCEbla}
\end{equation}

[TODO(egg): {\xsuxfont 𒌋𒊕} to {\xsuxfont 𒋙𒊕} above?]

\subsubsection{Use in modern publications}
\subsection{Non-numeric usage}
\epigraph{
{\nafont 𒊕 𒉆𒁾𒊬 𒁹 𒀸𒁉 𒅗𒁉 𒐋𒀀𒀭 𒐕𒀀𒀭 𒁀𒁺 𒅗 𒌤𒁉 𒄿𒍪𒌋}\\
{\nafont 𒊑𒌍 𒁾𒊬𒊒𒋾 𒊓𒀭𒋳𒆪 𒅖𒋼 𒋗𒌋 𒊑𒁶𒋗 𒋀𒅆𒋗 𒋗𒍑 𒄿𒆲 𒉌𒂍𒋢 𒋾𒁲𒂊}\\
{The beginning of the scribal art is a single wedge. That one has six pronunciations; it also stands for `sixty'. Do you know its reading?}}{Examenstext A}

[TODO(egg): In a footnote, comment on the {\xsuxfont 𒃻𒌋𒊕} situation.]

\subsection{Limited benefits of diachronic encoding for numerals}
\label{limitedBenefitsOfDiachrony}

[Composite texts dating back to the period where curved numerals are in use
tend to be limited to lexical texts, which do not usually have numbers.
When they do, diachronic encoding is prevented
by diš-aš distincitons anyway.
Administrative texts, which are where numbers are most prominent,
are not composite.]

[Diachronic reference works tend to not include numbers, or when they do,
to treat them specially (for intance, they are shown at the
end of sign lists such as TODO).]

[The overarching goal of having consistent representation for
equivalent numeric expressions from different periods
is quickly foiled by changes in metrology.]

Note that in \cite{Romach24} [TODO(egg): Cite the GitHub repository],
as in many other such analyses, numbers are removed as an early step
in processing; these therefore would not benefit from diachrony in
the encoding of numeric expressions.

\subsubsection{Compatibility with transliteration}
\label{transliteration}
%the area {\xsuxfont 𒐐𒑘𒐂𒃷} could be transliterated as 5.1.4, \cite[295]{Robson2008}.

\subsection{Compatibility considerations}
\label{compatibility}

\subsubsection{The case of ŠAR₂}


\section{Rationale for ED–Uruk numeral unification}
\label{unificationRationale}

\section{Considerations on individual numeral series}

[TODO Document to the extent possible the metrological systems in which each sign is used.
Note the disunification of N9 and N10 from 4(ban₂@c) and 5(ban₂@c).]

\section{Characters not included in this proposal}
\subsection{Missing numerals}
($N_{17}$, $12N_{14}$, etc.) 7(diš \emph{tenû})
\subsection{Stacking patterns}
\label{stackingPatterns}
(… are a mess, vary within Uruk, and are not transliterated/documented by Englund, so let’s not go there for now.)
\subsection{Matters for higher-level protocols}
Rotated bits:
https://cdli.mpiwg-berlin.mpg.de/artifacts/101087

\section{Acknowledgements}

TODO(egg): Something about the Vanséveren fonts

\printbibliography
\end{document}