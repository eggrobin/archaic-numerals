\documentclass[10pt, a4paper, twoside]{article}
\usepackage{fontspec}
\usepackage{realscripts}
\setmainfont[
  BoldFont=CambriaB,
  Numbers={OldStyle, Proportional}, 
  Ligatures={TeX, Common}, 
  SmallCapsFeatures={Letters=SmallCaps},
  %Contextuals=WordFinal,
            ]{Cambria}
\setmonofont[Scale=MatchLowercase]{Consolas}
\newcommand{\textcsc}[1]{{\addfontfeature{Letters=UppercaseSmallCaps}#1}}

\usepackage[dvipsnames]{xcolor}
\usepackage{accsupp}
\newfontfamily\cuneiformComposite{CuneiformComposite.ttf}
\newfontfamily\originalNoto{NotoSansCuneiform-UN-Swapped.ttf}
\newfontfamily\xsuxfont{NotoSansCuneiform-egg.ttf}
\newfontfamily\obfont{Santakku.ttf}
\newfontfamily\nafont{Assurbanipal.ttf}
\newfontfamily\nbfont{Esagil.ttf}
\newfontfamily\hantfont{NotoSerifTC-Regular.ttf}
%\usepackage{arabluatex}

\newfontfamily\arabicfont[Script=Arabic]{ScheherazadeNew-Regular.ttf}
%\newcommand{\textarabic}[1]{%\BeginAccSupp{method=pdfstringdef,unicode,ActualText={#1}}%
%\bgroup\textdir TRT%
%\arabfont #1%
%\egroup%\EndAccSupp{}
%}

\newfontfamily\proposalfont{Archaic-Cuneiform-Numerals.ttf}

\newcommand\oneAšC{{\proposalfont\symbol{"12550}}} % 𒀸
\newcommand\twoAšC{{\proposalfont\symbol{"12551}}} % 𒐀
\newcommand\threeAšC{{\proposalfont\symbol{"12552}}} % 𒐁
\newcommand\fourAšC{{\proposalfont\symbol{"12553}}} % 𒐂
\newcommand\fiveAšC{{\proposalfont\symbol{"12554}}} % 𒐃
\newcommand\sixAšC{{\proposalfont\symbol{"12555}}} % 𒐄
\newcommand\sevenAšC{{\proposalfont\symbol{"12556}}} % 𒐅
\newcommand\eightAšC{{\proposalfont\symbol{"12557}}} % 𒐆
\newcommand\nineAšC{{\proposalfont\symbol{"12558}}} % 𒐇
\newcommand\oneDišC{{\proposalfont\symbol{"12559}}}
\newcommand\twoDišC{{\proposalfont\symbol{"1255A}}}
\newcommand\threeDišC{{\proposalfont\symbol{"1255B}}}
\newcommand\fourDišC{{\proposalfont\symbol{"1255C}}}
\newcommand\fiveDišC{{\proposalfont\symbol{"1255D}}}
\newcommand\sixDišC{{\proposalfont\symbol{"1255E}}}
\newcommand\sevenDišC{{\proposalfont\symbol{"1255F}}}
\newcommand\eightDišC{{\proposalfont\symbol{"12560}}}
\newcommand\nineDišC{{\proposalfont\symbol{"12561}}}
\newcommand\oneUC{{\proposalfont\symbol{"12562}}}
\newcommand\twoUC{{\proposalfont\symbol{"12563}}}
\newcommand\threeUC{{\proposalfont\symbol{"12564}}}
\newcommand\fourUC{{\proposalfont\symbol{"12565}}}
\newcommand\fiveUC{{\proposalfont\symbol{"12566}}}
\newcommand\sixUC{{\proposalfont\symbol{"12567}}}
\newcommand\sevenUC{{\proposalfont\symbol{"12568}}}
\newcommand\eightUC{{\proposalfont\symbol{"12569}}}
\newcommand\nineUC{{\proposalfont\symbol{"1256A}}}
\newcommand\oneŊešTwoC{{\proposalfont\symbol{"1256B}}}
\newcommand\twoŊešTwoC{{\proposalfont\symbol{"1256C}}}
\newcommand\threeŊešTwoC{{\proposalfont\symbol{"1256D}}}
\newcommand\fourŊešTwoC{{\proposalfont\symbol{"1256E}}}
\newcommand\fiveŊešTwoC{{\proposalfont\symbol{"1256F}}}
\newcommand\sixŊešTwoC{{\proposalfont\symbol{"12570}}}
\newcommand\sevenŊešTwoC{{\proposalfont\symbol{"12571}}}
\newcommand\eightŊešTwoC{{\proposalfont\symbol{"12572}}}
\newcommand\nineŊešTwoC{{\proposalfont\symbol{"12573}}}
\newcommand\oneŊešʾuC{{\proposalfont\symbol{"12574}}}
\newcommand\twoŊešʾuC{{\proposalfont\symbol{"12575}}}
\newcommand\threeŊešʾuC{{\proposalfont\symbol{"12576}}}
\newcommand\fourŊešʾuC{{\proposalfont\symbol{"12577}}}
\newcommand\fiveŊešʾuC{{\proposalfont\symbol{"12578}}}
\newcommand\oneŠarTwoC{{\proposalfont\symbol{"12579}}}
\newcommand\twoŠarTwoC{{\proposalfont\symbol{"1257A}}}
\newcommand\threeŠarTwoC{{\proposalfont\symbol{"1257B}}}
\newcommand\fourŠarTwoC{{\proposalfont\symbol{"1257C}}}
\newcommand\fiveŠarTwoC{{\proposalfont\symbol{"1257D}}}
\newcommand\sixŠarTwoC{{\proposalfont\symbol{"1257E}}}
\newcommand\seveŠarTwoC{{\proposalfont\symbol{"1257F}}}
\newcommand\eightŠarTwoC{{\proposalfont\symbol{"12580}}}
\newcommand\nineŠarTwoC{{\proposalfont\symbol{"12581}}}
\newcommand\oneŠarʾuC{{\proposalfont\symbol{"12582}}}
\newcommand\twoŠarʾuC{{\proposalfont\symbol{"12583}}}
\newcommand\threeŠarʾuC{{\proposalfont\symbol{"12584}}}
\newcommand\fourŠarʾuC{{\proposalfont\symbol{"12585}}}
\newcommand\fiveŠarʾuC{{\proposalfont\symbol{"12586}}}
\newcommand\oneEšeThreeC{{\proposalfont\symbol{"1258C}}}
\newcommand\oneBurʾuC{{\proposalfont\symbol{"1258E}}}
\newcommand\oneBanTwoC{{\proposalfont\symbol{"12593}}}
\newcommand\oneNThirtyA{{\proposalfont\symbol{"125AF}}}
\newcommand\oneNThirtyC{{\proposalfont\symbol{"125B0}}}
\newcommand\oneNTwentyNineA{{\proposalfont\symbol{"125B6}}}

\usepackage{tikz-cd}
\usetikzlibrary{babel}

\usepackage{polyglossia}
\setdefaultlanguage[variant=british]{english}
\setotherlanguages{greek,german,russian,french,arabic}
\usepackage{luabidi}
% We use the english/american quote style, i.e., outer double quotes and inner
% single quotes, but british typographic rules (punctuation after the quotation
% marks).
\usepackage[style=english/american]{csquotes}

\usepackage{mathtools}

\usepackage{unicode-math}
\setmathfont[Scale=MatchLowercase, math-style=ISO]{Cambria Math}
\setmathfontface\unifrak{UnifrakturMaguntia.ttf}[Scale=MatchLowercase]

\usepackage[colorlinks,allcolors=Periwinkle]{hyperref}
\usepackage[backend=biber,giveninits=true,maxnames=100,style=alphabetic,maxalphanames=4,doi=true,url=false,eprint=true,labelalpha=true,dateusetime=true]{biblatex}
\addbibresource{bibliography.bib}
\DeclareSourcemap{
  \maps{
    \map{
      \step[fieldsource=keywords, match=\regexp{reference}, final]
      \step[fieldsource=entrykey, final]
      \step[fieldset=shorthand, origfieldval]
    }
    \map{
      \step[fieldsource=keywords, match=\regexp{unicode}, final]
      \step[fieldsource=eprint, final]
      \step[fieldset=shorthand, origfieldval]
    }
    \map{
      \pertype{artwork}
      \step[fieldsource=eprint, final]
      \step[fieldset=shorthand, origfieldval]
    }
  }
}

% Allow breaking in numbers and after lower and upper case letters in bibliography
% URLs, see https://tex.stackexchange.com/a/134281.
% We use fairly low values to avoid unsightly spacing:
% http : / / example . com is not an improvement over http://ex-
% ample.com.  We prefer breaking in numbers and uppercase letters, which are often
% IDs, rather than lowercase letters, which sometimes form meaningful words, or at
% least tokens that are not customarily broken, e.g. the protocol.
\setcounter{biburlnumpenalty}{100}
\setcounter{biburllcpenalty}{500}
\setcounter{biburlucpenalty}{100}
\renewcommand\UrlFont{\rmfamily}

\AtEveryBibitem{\clearlist{language}}  % TODO(egg): Why are we doing this?

\newcommand\UTCdoc[1]{\href{https://www.unicode.org/cgi-bin/GetMatchingDocs.pl?#1}{#1}}

\DeclareFieldFormat{doi}{%
  \newline
  \mkbibacro{DOI}\addcolon\space
    \ifhyperref
      {\href{https://doi.org/#1}{#1}}
      {#1}}
\DeclareFieldFormat{eprint:cnki}{%
  \newline
  \mkbibacro{CNKI}\addcolon\space
    \ifhyperref
      {\href{http://www.cnki.com.cn/Article/CJFDTOTAL-#1.htm}{#1}}
      {#1}}
      \DeclareFieldFormat{eprint:utc}{%
        \newline
        \mkbibacro{UTC}\addcolon\space
          \ifhyperref
            {\UTCdoc{#1}}
            {#1}}
\DeclareFieldFormat{eprint:cdli}{%
  \newline
  \mkbibacro{CDLI}\addcolon\space
    \ifhyperref
      {\href{https://cdli.ucla.edu/#1}{#1}}
      {#1}}
\DeclareFieldFormat{eprint:ebda}{%
  \newline
  Eb\textsc{da}\addcolon\space
    \ifhyperref
      {\href{http://ebda.cnr.it/tablet/view/#1}{#1}}
      {#1}}
\DeclareFieldFormat{eprint:oracc}{%
  \newline
  \mkbibacro{ORACC}\addcolon\space
    \ifhyperref
      {\href{http://oracc.org/#1}{#1}}
      {#1}}
\DeclareFieldFormat{eprint:louvre}{%
  Louvre Collections\addcolon\space
    \ifhyperref
      {\href{https://collections.louvre.fr/#1}{#1}}
      {#1}}
\DeclareFieldFormat{eprint}{%
  \newline
  eprint\addcolon\space
    \ifhyperref
      {\href{#1}{\nolinkurl{#1}}}
      {\nolinkurl{#1}}}
\DeclareFieldFormat{isbn}{%
  \newline
  \mkbibacro{ISBN}\addcolon\space#1}
      
\newcommand{\idest}{\emph{i.e.}}
\newcommand{\exempligratia}{\emph{e.g.}}
\newcommand{\obverse}{obv.}
\newcommand{\reverse}{rev.}
\newcommand{\recto}{\emph{recto}}
\newcommand{\verso}{\emph{verso}}
\newcommand{\withnote}{n.}
\newcommand{\withnotes}{nn.}

\usepackage{multirow}

\usepackage{epigraph}
\renewcommand{\epigraphsize}{\footnotesize}
\renewcommand{\textflush}{flushepinormal}
%\setlength\epigraphrule{0pt}
\epigraphnoindent

\usepackage{enumitem}
\renewcommand\labelitemi{---}
\renewcommand\labelitemii{---}

\usepackage[style=iso]{datetime2}

\hyphenation{cunei-form}
\usepackage{microtype}
\tikzcdset{
arrow style=math font,
}

\title{Archaic cuneiform numbers}
\author{Robin Leroy, Anshuman Pandey, and Steve Tinney}

\begin{document}

\maketitle

\tableofcontents

\section{Summary}

This document proposes encoding some numerals used in the Uruk and Early Dynastic periods in conjunction
with the Sumero-Akkadian cuneiform script\footnote{ISO 15924: Xsux, Script property value long name: Cuneiform; encoded since Unicode Version 5.0.}
and the proto-cuneiform script\footnote{ISO 15924: Pcun, not yet encoded.}.
The proposed characters are listed in section~\ref{proposal}.

The non-numeric signs of proto-cuneiform will be the subject of a separate proposal;
we need only note here that the divergence between the approaches to character identity
in modern scholarship requires that proto-cuneiform be disunified from cuneiform:
proto-cuneiform is effectively treated as an undeciphered script.
In contrast, the cuneiform encoding model is semantic,
requiring an understanding of the text to correctly encode it.

However, the \emph{numerals} used in proto-cuneiform should be unified with
ones used in the Early Dynastic period, for the reasons set forth in
section~\ref{unificationRationale}.
The proposed ``curved'', or ``curviform'', numerals\footnote{%
Impressed into clay using cylindrical styli,
held either perpendicular to the tablet, yielding
\oneUC{} (small stylus) or
\oneŠarTwoC{} (large stylus),
or at a shallower angle:
\oneAšC, \oneDišC{} (small stylus),
\oneŊešTwoC{} (large stylus).
Some numerals are composed of multiple such impressions,
\exempligratia, \oneŊešʾuC.
The terms ``curved'', ``curviform'', and ``round''
can be found in the literature.
We avoid the term ``round'' here as it has other meanings
in the context of numbers.
We use ``curviform'' in this document as, being the least
common term, it is least likely to lead to confusion,
and ``CURVED'' in the character names for consistency
with documentation about the modifier @c used in machine
readable ATF transliterations \cite{inlineATF}.}
should however \emph{not} be unified with
the already-encoded cuneiform numerals\footnote{%
Impressed into clay using a stylus with a trihedral end:
{\xsuxfont 𒀸} (stylus held horizontally),
{\xsuxfont 𒁹} (vertically),
{\xsuxfont 𒀹} (diagonally)
{\xsuxfont 𒌋} (with the head of the stylus),
{\xsuxfont 𒐕} (stylus pressed deeper, forming a larger wedge),
{\xsuxfont 𒐞} (combining {\xsuxfont 𒐕} and {\xsuxfont 𒌋}), etc.}.
Since the encoding proposals for the cuneiform script
twenty years ago provisionally considered the curviform numerals
to be glyph variants of the cuneiform numerals,
a detailed rationale is provided in section~\ref{disunificationRationale},
including compatibility considerations in section \ref{compatibility}.

The overall picture of unifications and disunifications over time is illustrated in table~\ref{tableUnificationsDisunifications}.
The Script\_Extensions property assignments in section~\ref{properties} reflect the overlap.

[TODO(egg): Mention the other sections here too.]

\begin{table}[htbp]
\begin{center}
\begin{tabular}{ l | l | l | l |} \cline{2-4}
                                                & Uruk III \& earlier & ED – Ur III                         & OB \& later    \\\hline
\multicolumn{1}{|c|}{\multirow{2}{*}{Numerals}} & \multicolumn{2}{|c|}{This proposal}                       &                \\\cline{2-4}
\multicolumn{1}{|c|}{}                          &                     & \multicolumn{2}{|c|}{\multirow{2}{*}{Existing Xsux}} \\\cline{1-2}
\multicolumn{1}{|c|}{Non-numeric signs}         & Future Pcun         & \multicolumn{2}{|c|}{}                               \\\hline
\end{tabular}
\caption{Usage of existing, proposed, and future characters across functions and time periods.}
\end{center}
\label{tableUnificationsDisunifications}
\end{table}

\section{Proposed changes to the Standard}
\label{proposal}
\subsection{Summary of proposed characters}
% The following characters are proposed;
% their names follow archaic conventions where possible, see section~\ref{unificationRationale}.
% they are ordered roughly according to their function in various metrological systems,
% so as to facilitate annotation and readability in the character names list.
\subsection{Properties}
\label{properties}
\subsection{Character names list}
\subsection{Core specification text}

\section{Rationale for curviform--cuneiform disunification}
\label{disunificationRationale}
TODO(egg): blurb.

\subsection{The cuneiform encoding model}
\label{xsuxModel}
As outlined in, \exempligratia, \cite{UTR56}, the cuneiform encoding model is diachronic;
each character may have wildly different glyphs depending on time period and region.
For instance, the sign IM
may resemble {\cuneiformComposite 𒅎} in texts from Early Dynastic IIIa Šuruppag as in the character code
charts,
{\cuneiformComposite 𒉎} % TODO(egg): Do something clever so the underlying text is 𒅎.
later in the third millenium\footnote{Merging with U+\textcsc{1224E} \textcsc{cuneiform sign ni2}.},
{\obfont 𒅎} in Old Babylonian cursive,
{\nafont 𒅎} in Neo-Assyrian, but is always encoded as U+\textcsc{1214E} \textsc{cuneiform sign im}.

This encoding model allows for the interoperable representation of editions
of diachronic reference works such as sign lists\footnote{Notably \cite{OSL} and the online edition of \cite{MZL} in \cite[Signs]{eBL}.}
and dictionaries\footnote{Notably \cite{ePSD2} and the online edition of \cite{Schramm2010} in \cite[Dictionary]{eBL}.},
and of composite texts\footnote{For example, there are Neo-Assyrian and Neo-Babylonian
copies parts of the laws of {\xsuxfont 𒄩𒄠𒈬𒊏𒁉}, as well as Old Babylonian copies in both archaizing
and cursive styles.
Because of damage on the stele \cite{P249253},
some sections are known only from those copies. See \cite[110\psqq]{Oelsner2022}.}.
By being compatible with similarly diachronic transliteration practice,
\idest, by avoiding distinctions finer than those made in transliteration,
the encoding model also allows for automated conversion of transliterated
corpora to cuneiform,
which has proven useful as a processing step in analyses such as
\cites{Romach24}{JauhiainenJauhiainen24}\footnote{Attendees
may recall the summary given on the third day of UTC \#180, as recorded in \cite{L2/24-159}.
Other readers may refer to \cite[242,148]{RAI69Abstracts}.}.
The diachronic approach is also useful for pedagogic applications\footnote{For instance,
Old Babylonian grammar may be taught in the Neo-Assyrian script, as in \cite{Caplice2002}.}.

\subsection{Arguments for curviform–cuneiform unification}
\label{oldArgumentsForUnification}
In this context, the argument was made in \cite{L2/04-099}, as part of discussion of the cuneiform
encoding\footnote{At that time scoped to the répertoire of the Ur III period and later, see \cite[1]{L2/03-162},
although many disunifications, such as $\text{\xsuxfont 𒅎}\ne\text{\xsuxfont 𒉎}$, were informed by Early Dynastic distinctions.}
that the curviform numerals, which occasionally appear in the Ur III period
and are used heavily in the Early Dynastic period,
were a stylistic distinction unifiable with the cuneiform digits, and that
an archaizing Ur III font or an Early Dynastic font could have curviform glyphs for
the appropriate characters.

Some co-occurrence of curviform and cuneiform digits was known and acknowledged.
\cite[3]{L2/04-099} cites \cite[62]{NissenDamerowEnglund1993}, which is a copy of \cite{P020054},
an Early Dynastic IIIb administrative tablet from Ŋirsu.
The excerpt cited, lines 1--3 of column 1 of the obverse, is as follows:
\begin{quote}
\begin{tabular}{l l l l l l l}
\xsuxfont 𒐕\footnotemark& \xsuxfont 𒌋 & \xsuxfont 𒈦&\xsuxfont 𒑍&\xsuxfont 𒄀&\xsuxfont 𒍑&\xsuxfont 𒁲\\
$1$(ŋeš₂) & $1$(u) & $1/2$(diš) & $5$(diš \emph{tenû}) & gi & us₂ & sa₂\\
\multicolumn{3}{c}{$7.5$ (ropes)} & $5$ & reed & side & equal
\end{tabular}\\
\footnotetext{As noted in \cite[466]{Powell1987}, this sign has a very short ``tail'' in this period,
so that it is wider than it is tall, and can at first seem like a large \text{\xsuxfont 𒀸} in copies.
The photos in CDLI clearly show that this is in fact a vertical wedge.}
\begin{tabular}{l l l l l}
\xsuxfont 𒌍\footnotemark& \xsuxfont 𒑎 &\xsuxfont 𒄀&\xsuxfont 𒊕&\xsuxfont 𒁲\\
$3$(u) & $6$(diš \emph{tenû}) & gi & saŋ & sa₂\\
$3$(ropes) & $6$ & reed & front & equal
\end{tabular}\\
\footnotetext{Note that ED IIIb {\xsuxfont 𒌋} numerals have a somewhat
different appearance from those of the Ur III period used in this transcription;
the sign {\xsuxfont 𒌍} in \cite{P020054} looks more like Ur III {\xsuxfont 𒆳}.}
\begin{tabular}{l l l l l}
\xsuxfont 𒃷𒁉&
\oneUC&
\oneEšeThreeC&
\oneAšC&
\oneDišC\\
ašag-bi&1(bur₃ᶜ)&1(eše₃ᶜ)&1(ikuᶜ)&1/2(ikuᶜ)\\
field-this&
&
&
&
\end{tabular}\\\begin{flushright}
\begin{tabular}{lll}
\multicolumn{3}{l}{\xsuxfont 𒓺𒋛𒂵𒄰}\\
\multicolumn{3}{l}{tugₓ(LAK483)-si-ga-kam}\\
tugsiga&=ak&=am\\
ploughed&=GEN&=COP
\end{tabular}
\end{flushright}
\end{quote}
The argument made in \cite[4]{L2/04-099} is that this is comparable to a stylistic distinction such as\footnote{We
have taken the liberty of adjusting the analogy to use measures approximately equal to those in \cite{P020054},
instead of a field of five by twenty-five metres.}
\begin{quote}
$465$ metres, equal lengths\\
$198$ metres, equal widths\\
this field is $\unifrak{9,18}$ hectares of ploughed land
\end{quote}
where the numerals have the same structure (\cite{L2/04-099} contrasts this to the different structures
of ASCII digits and roman numerals).
That document further claims that ``the number signs do not normally carry in their individual signs the
meaning of what they are used to measure'', and that curviform and cuneiform numerals ``are not normally mixed together
in a single numerical expression'', noting the exceptions of \cites{P232278}{P232280}.
In addition, \cite[4]{L2/04-099} points out that the cuneiform numeric signs are descended from
the curviform ones (this is undisputed),
and claims there is only a small re-allocation of the function of signs
(from \oneAšC{} to {\xsuxfont 𒁹} numerals).
It therefore comes to the conclusion that the use of curviform
numerals should be seen as a formatting distinction,
rather than one that should be represented in plain text,
and insists that the encoding should capture the lineal
historical descent of those signs, presumably to take
advantage of the benefits of diachronic encoding described
in section~\ref{xsuxModel}.

Although they had been part of the preliminary proposal \cite{L2/03-393R},
the curviform numerals were therefore removed from \cite{L2/04-036} and \cite{L2/04-189},
which both state that ``The distinction between curved numerals and their cuneiform descendants
is treated as glyphic for the purposes of the present proposal;
this issue will need to be revisited in subsequent encoding phases.''

The time has come to revisit this issue.
As we will see in section~\ref{metrology},
numerals can only be interpreted in the context of what they measure \idest,
as part of a metrological system.
In section~\ref{earlyMetrology} we will see that in some periods:
\begin{itemize}[nosep]
  \item the functions and use of the numerals vary beyond the mere \oneAšC/{\xsuxfont 𒁹} switch;
  \item the contrast between curviform and cuneiform numerals is commonly used to distinguish metrological systems;
  \item some metrological systems commonly mix curviform and cuneiform in single numerical expressions.
\end{itemize}

\subsection{A primer on classic Ur III and Old Babylonian metrologies}
\label{metrology}
{\settowidth{\epigraphwidth}{\epigraphsize\obfont 𒁾 𒁹 𒂆𒋫 𒍠 𒆬 𒌋 𒈠𒈾𒂠 \hspace{1.5em}}
\epigraph{
{\obfont 𒁾 𒊬𒊑𒉈\hfill 𒂵𒁺} \\
{\obfont 𒁾 𒀸 𒊺 𒄥𒋫 𒍠 \hfill {\nafont 𒐞} 𒄥𒂠} \\
{\obfont 𒁾 𒁹 𒂆𒋫 𒍠 \hfill 𒆬 \hfill 𒌋 𒈠𒈾𒂠} \\
I want to write tablets: the tablet of 1 cor of barley to
600 cor; the tablet of 1 shekel of silver to 10 minas […]}{Edubbaʾa D}}

Before diving into the usage of the curviform numerals
in the Early Dynastic period to explain the constrast
with cuneiform numerals, it is useful to understand
the usage of the already-encoded characters in the
Ur III and Old Babylonian periods.

As is well known\footnote{See, \exempligratia, \cite[§22.3.3,
\emph{sub} ``\href{https://www.unicode.org/versions/Unicode16.0.0/core-spec/chapter-22/\#G42894}{Cuneiform Numerals}'']{Unicode16}.}
a sexagesimal place value system (SPVS) was used in Meso\-potamia from the late third millenium onwards.
One should bear in mind, however, that other systems were used;
the SPVS was primarily used in calculations,
with results being expressed in non-positional systems \cites[76]{Robson2008}{Robson2022}.
The digits $1$–$59$ of the SPVS have inner structure which is reflected in the encoding: the digits 1–9 are the individual
characters {\xsuxfont 𒁹}–{\xsuxfont 𒑆}, the multiples of ten ($10$–$50$) are {\xsuxfont 𒌋}–{\xsuxfont 𒐐},
but the other digits $11$–$59$ are sequences {\xsuxfont 𒌋𒁹}–{\xsuxfont 𒐐𒑆};
in effect the base-sixty digits are themselves written in base ten, with a different set of symbols for the tens place.
This reflects the origin of the sexagesimal place value system;
it derives from a \emph{non-positional} system, hereafter the \emph{cuneiform discrete counting system} \ref{systemSOB},
which had different signs for the units {\xsuxfont 𒁹}–{\xsuxfont 𒑆},
tens {\xsuxfont 𒌋}–{\xsuxfont 𒐐}, sixties {\xsuxfont 𒐕}–{\xsuxfont 𒐝} (with larger wedges
than the units), multiples of six hundred {\xsuxfont 𒐞}–{\xsuxfont 𒐢},
multiples of three thousand six hundreds {\xsuxfont 𒊹}–\vphantom{\scalebox{0.8}{\xsuxfont 𒐫}}\smash{\xsuxfont 𒐫}, and
multiples of thirty-six thousand {\xsuxfont 𒐬}–\vphantom{\scalebox{0.8}{\xsuxfont 𒐱}}\smash{\xsuxfont 𒐱}.

 \subsubsection{The discrete counting system} 
The relations between the values of the signs in the cuneiform discrete counting system
may be summarized by the following factor diagram\footnote{These diagrams,
which have become standard in discussions of Mesopotamian metrology, originate with \cite[10]{Friberg1978},
where they are called \emph{step-diagrams}.},
where the number over arrow indicates the multiple
of the preceding sign (right of the arrow) corresponding to the following sign (left).
\begin{equation}
\text{\xsuxfont 𒐬} \xleftarrow{10}\text{\xsuxfont 𒊹} \xleftarrow{6}\text{\xsuxfont 𒐞} \xleftarrow{10}\text{\xsuxfont 𒐕} \xleftarrow{6}\text{\xsuxfont 𒌋}\xleftarrow{10}\text{\xsuxfont 𒁹}
\tag{$S_{\text{Ur III/OB}}$}
\label{systemSOB}
\end{equation}
For example, the number $1729=((\textcolor{BrickRed}{2}\times 10 + \textcolor{Dandelion}{8})\times 6 + \textcolor{NavyBlue}{4})\times 10 + \textcolor{Orchid}{9} = 28\times 60 + 49$
would be written {\xsuxfont \textcolor{BrickRed}{𒐟}\textcolor{Dandelion}{𒐜}\textcolor{NavyBlue}{𒐏}\textcolor{Orchid}{𒐎}} in the discrete counting system,
and {\xsuxfont 𒎙𒐍𒐏𒐎} in the sexagesimal place value system.

\subsubsection{The area system}
The discrete counting system was not the only non-positional system in use in the Ur III and Old Babylonian periods; different systems were in use depending on what was being counted or measured.
For instance, field areas were measured using the following system, where for the named
units we have provided the name of the unit in transliterated Sumerian, normalized Old Babylonian Akkadian,
and the approximate metric equivalent \cites[378]{Friberg2007}{Robson2019}:
\begin{equation}
\text{\xsuxfont 𒐬}
\xleftarrow{10}\text{\xsuxfont 𒊹}
\xleftarrow{6}\text{\xsuxfont 𒐴}
\xleftarrow{\ 10\ }\underset{\mathclap{\substack{\text{bur₃}\\\text{\emph{būrum}}\\6,48~\text{ha}}}}{\text{\xsuxfont 𒌋}}
\xleftarrow{\quad3\quad}\underset{\mathclap{\substack{\text{eše₃}\\\text{\emph{eblum}}\\2,16~\text{ha}}}}{\text{\xsuxfont 𒑘}}
\xleftarrow{\quad6\quad}\underset{\mathclap{\substack{\text{iku}\\\text{\emph{ikûm}}\\3600~\text{m}^2}}}{\text{\xsuxfont 𒀸}}
\xleftarrow{\quad2\quad}\underset{\mathclap{\substack{\text{\emph{ubûm}}\\1800~\text{m}^2}}}{\text{\xsuxfont 𒀹}}
\xleftarrow{\ 2\ }\text{\xsuxfont 𒑠}
\tag{$G_{\text{Ur III/OB}}$}
\label{systemGOB}
\end{equation}
Note that for the range of areas given above\footnote{For
areas smaller than a quarter \emph{ikûm}, an overt unit is used,
with $1~\text{\emph{mūšarum}}$ ($36~\text{m}^2$) written {\xsuxfont 𒁹𒊬}, equal to one hundredth of an \emph{ikûm},
then sexigesimally subdivided in $60~\text{\xsuxfont{𒂆}}$ (shekels).
For areas greater than $3600~\text{\emph{būrū}}$,
the {\xsuxfont 𒊹} and {\xsuxfont 𒐬} numerals are reused with a suffix {\xsuxfont 𒃲} (gal, Sumerian: big),
as follows \cites[\pno~295 \withnote~b and c]{Robson2008}[378]{Friberg2007}{Robson2019}: \[
\underbrace{\text{\xsuxfont 𒐬𒃲}
\xleftarrow{10}\text{\xsuxfont 𒊹𒃲}
\xleftarrow{6}\text{\xsuxfont 𒐬}
\xleftarrow{10}\text{\xsuxfont 𒊹}
\xleftarrow{6}\text{\xsuxfont 𒐴}
\xleftarrow{10}\text{\xsuxfont 𒌋}
\xleftarrow{3}\text{\xsuxfont 𒑘}
\xleftarrow{6}\text{\xsuxfont 𒀸}
\xleftarrow{2}\text{\xsuxfont 𒀹}
\xleftarrow{2}\text{\xsuxfont 𒑠}}_{\text{\footnotesize\xsuxfont 𒃷}}
\xleftarrow{2,5}\underbrace{\text{\xsuxfont 𒌋}
\xleftarrow{10}\text{\xsuxfont 𒁹}}_{\text{\footnotesize\xsuxfont 𒊬}}
\xleftarrow{6}\underbrace{\text{\xsuxfont 𒌋}
\xleftarrow{10}\text{\xsuxfont 𒁹}}_{\text{\footnotesize\xsuxfont 𒂆}}\text.
\]},
this system does not use any symbols separate from the numerals
for the individual units (\emph{ubûm}, \emph{ikûm}, \emph{eblum}, and \emph{būrum}).
As mentioned in \cite{Robson2019}, the whole numeric expression for the area would be followed by the sign {\xsuxfont 𒃷}
functioning as punctuation\footnote{TODO(egg): acknowledge Proust 2020
  but note that this is irrelevant to encoding concerns}, but the numerals are tied to the metrology; thus
a surface of $5~\text{\emph{būr}}$ $1~\text{\emph{ebel}}$ $4~\text{\emph{ikû}}$ ($100~\text{\emph{ikû}}$, $36~\text{ha}$) would be written\footnote{As in the surface of the field of {\xsuxfont 𒀀𒅗𒋡𒆠} (the city of Apisal) reported on \cite[r.~1]{P102305}}
{\xsuxfont 𒐐𒑘𒐂𒃷}. Contrast this with systems
where the same numerals are used for different units,
and overt units are used, as in ``$88$ acres $3$ roods $33$ perches'' or {\hantfont 五頃八畝五分九厘}.
Note also that the same signs are shared between multiple systems,
with different relations; the sign {\xsuxfont 𒊹} is equal to sixty times {\xsuxfont 𒌋}
in the area system, but to three hundred and sixty times {\xsuxfont 𒌋} in the discrete counting system.

\subsubsection{The capacity system}\label{capacity}
Another such system of note is the one for capacities\footnote{Used
for volumes of grain, but also oil, dairy products, beer, etc., as well as to express the capacity of boats;
volumes of earthworks instead use system \ref{systemGOB} based on a height of one cubit, see\cites[488]{Powell1987}[294]{Robson2008}{Robson2019}.} \cites[376]{Friberg2007}{Robson2019},
\begin{equation}
\underbrace{
%\text{(as in \ref{systemSOB})}
\text{\xsuxfont 𒐬} \xleftarrow{10}\text{\xsuxfont 𒊹} \xleftarrow{6}\text{\xsuxfont 𒐞} \xleftarrow{10}\text{\xsuxfont 𒐕}
\xleftarrow{6}\text{\xsuxfont 𒌋}
\xleftarrow{10}\underset{\mathclap{\substack{\text{gur}\\\text{\emph{kurrum}}}}}{\text{\xsuxfont 𒀸}}}_{\text{\xsuxfont 𒄥}}
\xleftarrow{\quad5\quad}\underset{\mathclap{\substack{\text{bariga}\\\text{\emph{parsiktum}}}}}{\text{\xsuxfont 𒁹}}
\xleftarrow{\quad6\quad}\underset{\mathclap{\substack{\text{ban₂}\\\text{\emph{sūtum}}}}}{\text{\xsuxfont 𒑏}}
\xleftarrow{\ 10\ }\underset{\mathclap{\substack{\text{sila₃}\\\text{\emph{qûm}}\\1~\text{l}}}}{\text{\xsuxfont 𒁹𒋡}}\text,
\tag{$C_{\text{Ur III/OB}}$}
\label{systemC}
\end{equation}
where the numerals for ban₂ are {\xsuxfont 𒑏}, {\xsuxfont 𒑐}, {\xsuxfont 𒑑}, {\xsuxfont 𒑒},
and {\xsuxfont 𒑔}, and those for bariga are {\xsuxfont 𒁹}, {\xsuxfont 𒑖}, {\xsuxfont 𒑗}, and {\xsuxfont 𒐉} (contrast
ordinary {\xsuxfont 𒈫} and {\xsuxfont 𒐈} otherwise used with {\xsuxfont 𒁹} numerals).
As described in \cite[\pno~585 \withnote~(b) and (f)]{Huehnergard2011},
the sign {\xsuxfont 𒄥} GUR, while it is used only with volumes in excess of one gur,
is written after the whole expression,
after the overt unit sign {\xsuxfont 𒋡} if present, and after the word for ``grain'' if present, as in
\[\begin{matrix}
\text{\xsuxfont 𒐢𒐝𒌋𒐂}&
\text{\xsuxfont 𒑑}&
\text{\xsuxfont 𒐋}&
\text{\xsuxfont 𒋡}&
\text{\xsuxfont 𒊺}&
\text{{\xsuxfont 𒄥}\footnotemark}\\
3554~\text{gur}&
3~\text{ban₂}&
6&\text{sila₃}&
\text{ of grain.}
\end{matrix}\]
\footnotetext{From \cite[\obverse~1, 1]{P309594}.}
Observe that while large numbers of gur follow\footnote{A larger unit, the guru₇ (\emph{karûm}, grain heap), is sometimes used instead, with {\xsuxfont 𒀸𒄦}={\xsuxfont 𒊹𒄥} ($1~\text{\emph{karûm}}=3600~\text{kurrū}$). See \cites[415]{Friberg2007}{Robson2019}.}
system \ref{systemSOB},
the use of horizontal (AŠ) numerals for the gur disambiguates from the vertical bariga,
as {\xsuxfont 𒌋𒁹𒄥} would be $10$~gur $1$~bariga, and {\xsuxfont 𒌋𒀸𒄥} would be $11$~gur;
again even with some overt units, most of the numerals
that participate in a metrological system have an interpretation
dependent on that system.

This intertwining of units and numerals explains the large number of already-encoded numeral series:
\begin{itemize}[nosep]
\item {\xsuxfont 𒁹}–{\xsuxfont 𒐎} used in \ref{systemSOB} and the SPVS as well as with overt units;
\item {\xsuxfont 𒌋}–{\xsuxfont 𒐔} used in \ref{systemGOB}, of which {\xsuxfont 𒌋}–{\xsuxfont 𒐐} are also used in \ref{systemSOB} and the SPVS as well as with overt units;
\item {\xsuxfont 𒐕}–{\xsuxfont 𒐝} used in \ref{systemSOB}, and sometimes with overt units;
\item {\xsuxfont 𒐞}–{\xsuxfont 𒐢} used in \ref{systemSOB};
\item {\xsuxfont 𒊹}–{\xsuxfont 𒐫} used in \ref{systemSOB} and \ref{systemGOB};
\item {\xsuxfont 𒐬}–{\xsuxfont 𒐱} used in \ref{systemSOB} and \ref{systemGOB};
\item {\xsuxfont 𒀸}–{\xsuxfont 𒐇} used in \ref{systemC} as well as with overt units of the weight system;
\item {\xsuxfont 𒑏}, {\xsuxfont 𒑐}, {\xsuxfont 𒑑}, {\xsuxfont 𒑒}, {\xsuxfont 𒑔} used in \ref{systemC};
\item {\xsuxfont 𒁹}, {\xsuxfont 𒑖}, {\xsuxfont 𒑗}, {\xsuxfont 𒐉} used in \ref{systemC}—note the overlap with {\xsuxfont 𒁹}–{\xsuxfont 𒑆};
\item {\xsuxfont 𒑘} and 	{\xsuxfont 𒑙} used in \ref{systemGOB}.
\end{itemize}
Only in the SPVS did numerals exist truly independently of metrology; to quote \cite[78]{Robson2008}:
``The SPVS temporarily changed the status of numbers from properties of real-world objects to independent entities that could be manipulated without regard to […] metrological system. […] Once the calculation was done, the result was expressed in the most appropriate metrological units and thus re-entered the natural world as a concrete quantity.''

\subsubsection{The length system}
In the Ur III and Old Babylonian periods, lengths are expressed using overt units counted with
{\xsuxfont 𒁹}-  and {\xsuxfont 𒌋} numerals with their system \ref{systemSOB} values\footnote{Adjacent
units are no more than a factor of $60$ apart, so higher numerals such as {\xsuxfont 𒐞} or {\xsuxfont 𒊹} are not
used.}.
Since it does not have any unusual numerals,
this system would not in itself be of much relevance to character encoding,
but we present it here as background for
its Early Dynastic counterpart presented in section~\ref{earlyMetrology}.
Metrological tables use the following units \cites[118]{Friberg2007}{Robson2019}:
\begin{equation}
  \underset{\substack{\text{danna}\\\text{\emph{bērum}}\\\text{league}\\10,8~\text{km}}}{\text{\xsuxfont 𒁹𒆜𒁍}}
  \xleftarrow{60}\underset{\substack{\text{UŠ\footnotemark}\\\\\text{cable}\\360~\text{m}}}{\text{\xsuxfont 𒁹𒍑}}
  \xleftarrow{10}\underset{\substack{\text{nindan}\\\text{\emph{nindanum}}\\\text{rod}\\6~\text{m}}}{\text{\xsuxfont 𒁹𒃻}}
  \xleftarrow{12}\underset{\substack{\text{kuš₃}\\\text{\emph{ammatum}}\\\text{cubit}\\50~\text{cm}}}{\text{\xsuxfont 𒁹𒌑}}
  \xleftarrow{30}\underset{\substack{\text{šu-si}\\\text{\emph{ubānum}}\\\text{finger}\\17~\text{mm}}}{\text{\xsuxfont 𒁹𒋗𒋛}}\text.
  \tag{$L_{\text{Ur III/OB}}$}
  \label{systemLOB}
\end{equation}
\footnotetext{TODO}
Two more units appear occasionally \cites[459]{Powell1987}[118]{Friberg2007}{Robson2019}:
\begin{equation}
  \text{\xsuxfont 𒁹𒆜𒁍}
  \xleftarrow{30}\text{\xsuxfont 𒁹𒍑}
  \xleftarrow{6}\underset{\substack{\text{eše₂}\\\text{\emph{ašlum}}\\\text{rope}\\60~\text{m}}}{\text{\xsuxfont 𒁹𒂠}}
  \xleftarrow{10}{\text{\xsuxfont 𒁹𒃻}}
  \xleftarrow{2}\underset{\substack{\text{gi}\\\text{\emph{qânum}}\\\text{reed}\\3~\text{m}}}{\text{\xsuxfont 𒁹𒄀}}
  \xleftarrow{6}{\text{\xsuxfont 𒁹𒌑}}
  \xleftarrow{30}{\text{\xsuxfont 𒁹𒋗𒋛}}\text.
  \tag{$\bar{L}_{\text{Ur III/OB}}$}
  \label{systemLbarOB}
\end{equation}
In addition, there are Akkadian names for the half-rope and half-reed,
see \cites[463\psq]{Powell1987}.
\subsubsection{Fractions}
TODO

\subsection{Curviform numerals in early metrologies}
\label{earlyMetrology}

At first sight, the metrological systems from the Early Dynastic period match the
ones previously mentioned.
In particular, the discrete counting system used in the Early Dynastic period
(and earlier in the Uruk period) clearly mirrors system \ref{systemSOB}
\cites[374]{Friberg2007}[127,165]{DamerowEnglund1987}:
\begin{equation}
\text{\oneŠarʾuC}
\xleftarrow{10}\text{\oneŠarTwoC}
\xleftarrow{6}\text{\oneŊešʾuC}
\xleftarrow{10}\text{\oneŊešTwoC}
\xleftarrow{6}{\text{\oneUC}}
\xleftarrow{10}{\text{\oneAšC}}\text.
\tag{$S$}
\label{systemS}
\end{equation}
Likewise the area system used in the Early Dynastic IIIb period mirrors system \ref{systemGOB}
\cites[72]{LAK}[63]{NissenDamerowEnglund1993}[378]{Friberg2007}{Lecompte2016}:
\begin{equation}
\text{\oneŠarʾuC}
\xleftarrow{10}\text{\oneŠarTwoC}
\xleftarrow{6}\text{\oneBurʾuC}
\xleftarrow{10}{\text{\oneUC}}
\xleftarrow{3}{\text{\oneEšeThreeC}}
\xleftarrow{6}\text{\oneAšC}\text,
\tag{$G_{\text{ED IIIb}}$}
\label{systemGED}
\end{equation}

As noted in \cite[4]{L2/04-099} (see section~\ref{oldArgumentsForUnification}), the vertical
{\xsuxfont 𒁹} from \ref{systemSOB} becomes a horizontal {\oneAšC} in system \ref{systemS}.
It is however far from the only case of such a reallocation of function.
The earlier form of System G is \cites[141,165]{DamerowEnglund1987}[378]{Friberg2007}:
\begin{equation}
\text{\oneŠarTwoC}
\xleftarrow{6}\text{\oneŠarʾuC}
\xleftarrow{10}{\text{\oneUC}}
\xleftarrow{3}{\text{\oneEšeThreeC}}
\xleftarrow{6}\text{\oneAšC}\text,
\tag{$G$}
\label{systemG}
\end{equation}
Observe that, as noted in \cites[142]{DamerowEnglund1987}, {\oneŠarʾuC} changes meaning from $10 \text{\oneUC}$ in system \ref{systemG} to $600 \text{\oneUC}$ in system \ref{systemGED}.
System \ref{systemG} is used in the Uruk period, but also
in the ED I–II period (it is the ``area 2'' system in \cite{Chambon2003},
whereas \ref{systemGED} is the ``area 1'' system).

\subsubsection{Field lengths in Ŋirsu}
The length system of the Early Dynastic IIIb state of Lagaš is of particular interest.
As described in \cites[466]{Powell1987}[289\psq]{Lecompte2020}, lengths are expressed in rods,
but the unit sign {\xsuxfont 𒃻} is generally omitted; in addition, only tens of rods
are used; these are equal to one rope, but the sign {\xsuxfont 𒂠} is not written either.
Length shorter than one rope are expressed in half-ropes
using the $1/2$ sign {\xsuxfont 𒈦} (again with no {\xsuxfont 𒂠}),
and then in reeds, \emph{with} the sign {\xsuxfont 𒄀}.
Effectively, this yields the following factor diagram:
\begin{equation}
  \text{\xsuxfont 𒐕}
  \xleftarrow{\quad6\quad}\underset{\mathclap{\substack{1~\text{eše₂}=10~\text{nindan}\\1~\text{rope}=10~\text{rods}\\60~\text{m}}}}{\text{\xsuxfont 𒌋}}
  \xleftarrow{\quad2\quad}\text{\xsuxfont 𒈦}
  \xleftarrow{10}\underset{\substack{\text{gi}\\\text{reed}\\3~\text{m}}}{\text{\xsuxfont 𒀹𒄀\footnotemark}}\text.
  \tag{$L_{\text{ED IIIb}}$}
  \label{systemLED}
\end{equation}
\footnotetext{Note that the reeds are counted using \emph{tenû} numerals, {\xsuxfont 𒀹}, {\xsuxfont 𒑊}, {\xsuxfont 𒑋}, etc.}
This is the system that was used to express the sides of the field in
\cite{P020054} discussed in section~\ref{oldArgumentsForUnification}.
In that tablet and others from the same period, such as the ones
discussed in \cite{Lecompte2020}, areas are expressed in
system \ref{systemGED}, with curviform numerals\footnote{TODO(egg):
Note the handful of late Urukagina tablets that start to have cuneiform areas.};
in the absence of overt units, such as when dealing with length that are
integer multiples of a half-rope\footnote{This is the case of the sides of the
field in \cite[\obverse~ii~2--3]{P020054}.},
the use of curviform or cuneiform numerals therefore disambiguates
a numeric expression between an area and a length,
and therefore the interpretation of its
numerals between systems \ref{systemGED} and \ref{systemLED}.
The sign {\xsuxfont 𒃷},
which would also disambiguate the interpretation as an area,
is sometimes used after areas in ED IIIb Lagaš, but not systematically;
in particular the area of the first field in \cite{P020054} does not use this suffix.
See \cite{Lecompte2020} for many examples with and without {\xsuxfont 𒃷}.

\subsubsection{Dyke lengths in Ŋirsu}
\cites[466]{Powell1987} notes that reeds ``are regularly written with the normal,
cuneiform end of the stylus. Higher units are usually written with the reversed
(round) end of the stylus.'' [TODO(egg): also mention Krebernik 1998 p. 303 \withnote~686.]
Powell does not elaborate on the specifics of this mixed use of numerals,
but a cursory search in CDLI finds many occurrences\footnote{A search for
curviform numerals followed by some number of reeds counted in (\emph{tenû})
cuneiform numerals currently finds 125 occurrences
across 47 tablets.}, such as:
\begin{itemize}[nosep] % TODO(egg): Check the photos while adding them to the bibliography.
  \item \cite[\obverse~1, 4]{P221305}\footnote{CDLI only has a copy, but a photo may be found in
  \cite[82]{Lecompte2012}. On that photo the {\xsuxfont 𒋗𒆕𒀀𒑊} is not visible.
  Lecompte notes that the copy is faithful;
  indeed another {\xsuxfont 𒋗𒆕𒀀𒑊} can be seen both on the copy and the photo on \obverse~2, 2.} {\twoAšC\xsuxfont 𒂠𒑍𒄀𒌑𒑌𒋗𒆕𒀀𒑊}% 2(asz@c) esz2 5(disz@t) gi kusz3 4(disz@t) szu-du3-a 2(disz@t)
%  \item \cite[\reverse~1, 1]{P221305} {\oneŊešTwoC\fourUC\oneBanTwoC\xsuxfont 𒂠𒑍𒄀} % 1(gesz2@c) 4(u@c) 1/2(asz@c) esz2 5(disz@t) gi
  \item \cite[\reverse~2, 1]{P020129} {\sixŊešTwoC\threeUC\xsuxfont 𒃻𒁺\oneBanTwoC\xsuxfont 𒑌𒄀} % 6(gesz2@c) 3(u@c) {ninda}nindax(DU) 1/2(asz@c) 4(disz@t) gi
  \item \cite[\reverse~5, 1]{P221291}\footnote{From copy.} {\oneŊešʾuC\twoŊešTwoC\fourUC\oneBanTwoC\xsuxfont 𒂠𒃻𒁺𒑊𒄀𒌑𒑋} % 1(gesz'u@c) 2(gesz2@c) 4(u@c) 1/2(asz@c) esz2 {ninda}nindax(DU) 2(disz@t) gi kusz3 3(disz@t)
  \item \cite[\reverse~2, 1]{P221266} {\oneAšC\oneBanTwoC\xsuxfont 𒂠𒑋𒄀} % 1(asz@c) 1/2(asz@c) esz2 3(disz@t) gi
\end{itemize}
These expressions use an
explicit sign {\xsuxfont 𒃻𒁺} (counted in multiples of ten) or
{\xsuxfont 𒂠}.
This notation—but not its use of curviform numerals—is
remarked on in \cite[\pno~290 \withnote~27]{Lecompte2020},
which cites several of the instances listed above.
It seems to be typical of texts about dykes.
These\footnote{TODO Cite also DP 568, the one with \oneŊešTwoC{} and \oneAšC{\xsuxfont 𒂠} even though it has no reeds.}
can be summarized by the following factor diagram:
\begin{equation}
  \underbrace{
  \text{\oneŊešʾuC}
  \xleftarrow{10}\text{\oneŊešTwoC}
  \xleftarrow{6}\text{\oneUC}}_{\text{\xsuxfont 𒃻𒁺}}=
  \underbrace{\text{\oneAšC}
  \xleftarrow{2}\text{\oneBanTwoC}}_{\text{\xsuxfont 𒂠\footnotemark}}
  \xleftarrow{10} \text{\xsuxfont 𒀹𒄀}
  \xleftarrow{6}\text{\xsuxfont 𒌑𒀹}
  \xleftarrow{3}\text{\xsuxfont 𒋗𒆕𒀀𒀹}
  \tag{$L'_{\text{ED IIIb}}$}
  \label{systemLpED}
\end{equation}
\footnotetext{TODO(egg): Note that one unit may be omitted if the other is present}

\subsubsection{Grain in Ŋirsu}

TODO note the {\xsuxfont 𒄥𒑊𒌌}.
Also cite Friberg1978 p.~43 and reproduce its factor diagram.

\subsubsection{Grain in Ebla}
Lengths of Early Dynastic IIIb dykes from Ŋirsu are far from the
only numeric expressions that mix curviform and cuneiform numerals.

The system of grain\footnote{Liquid capacities use a different system \cite[\pno~229 \withnote~12]{Archi2015}:\begin{equation*}
  \underset{\text{la-ḫa}}{\text{\xsuxfont 𒆷𒄩}}
  \xleftarrow{30}\underset{\text{sila₃}}{\text{\xsuxfont 𒋡}}
  \xleftarrow{6}\underset{\text{an-zamₓ}}{\text{\xsuxfont 𒀭𒍡}}\text.
\end{equation*}
At a glance it seems that {\xsuxfont 𒋡} are counted with cuneiform numerals and higher units
with curviform ones, thus
\begin{equation*}
  \underbrace{\text{\oneAšC\xsuxfont 𒈪𒀜}
  \xleftarrow{\frac{5}{3}}\text{\oneŊešTwoC}
  \xleftarrow{6}\text{\oneUC}
  \xleftarrow{10}\text{\oneAšC}}_{\text{\xsuxfont 𒆷𒄩}}
  \xleftarrow{3}\underbrace{\text{\xsuxfont 𒌋}
  \xleftarrow{10}\text{\xsuxfont 𒁹}}_{\text{\xsuxfont 𒋡}}
  \xleftarrow{6}\text{\xsuxfont 𒁹𒀭𒍡}\text,
\end{equation*}
but we have not investigated this thoroughly.} capacities in Ebla uses the following units\footnote{TODO mention the other one citing Chambon and the footnote in Archi}:
\begin{equation*}
  \underset{\text{gu₂-bar}}{\text{\xsuxfont 𒄘𒁇}}
  \xleftarrow{2}\underset{\text{ba-ri₂-zu}}{\text{\xsuxfont 𒁀𒌷𒍪}}
  \xleftarrow{\frac{5}{2}}\underset{\text{ŋin₄}}{\text{\xsuxfont 𒂆}}
  \xleftarrow{4}\underset{\text{niŋ₂-sagšu}}{\text{\xsuxfont 𒃻𒋙𒊕}}
  \xleftarrow{6}\underset{\text{an-zamₓ}}{\text{\xsuxfont 𒀭𒍡}}\text.
\end{equation*}
The {\xsuxfont 𒄘𒁇} and {\xsuxfont 𒁀𒌷𒍪} are generally counted using curviform numerals,
and the smaller units using cuneiform {\xsuxfont 𒁹} numerals.
Indeed, a search on \cite{EbDA} for co-occurrences of either {\xsuxfont 𒀭𒍡} or {\xsuxfont 𒃻𒋙𒊕} with
either of {\xsuxfont 𒄘𒁇} or {\xsuxfont 𒁀𒌷𒍪}
finds the following expressions\footnote{We cite here only one attestation per tablet;
most tablets contain several expressions mixing curviform {\xsuxfont 𒁀𒌷𒍪} and larger with cuneiform {\xsuxfont 𒂆} and smaller.
In all cases the transcriptions given here are based on the EbDA transliterations, but the
shape and orientation of the numerals was checked\footnotemark on a photograph (from EbDA unless noted otherwise).}:
\footnotetext{As we will see in
Section~\ref{transliteration}, CDLI transliterations indicate numeral shape; however,
as of this writing, they do so incorrectly on the Ebla corpus, claiming that all numerals are curviform,
so we were not able to rely on them in this specific case.}
\begin{enumerate}[nosep]
  \item \cite[\href{http://ebda.cnr.it/tablet/view/1324\#58045}{\verso~4, 9}]{P240532} {\oneAšC\xsuxfont 𒊺 𒁀𒊑𒅗\footnote{ba-ri₂-zu₂, a variant spelling.}𒐊𒃻𒋙𒊕}% v.4,9 1 še ba-ri₂-zu₂ 5 nig₂-sagšu % ARET 09 0010 
  \item \cite[\href{http://ebda.cnr.it/tablet/view/1350\#59630}{\verso~1, 1}]{P240548} {\oneAšC\xsuxfont 𒁀𒊑𒍪 𒐈𒀭𒍡} % ARET 09 0036 1 ba-ri₂-zu 3 an-zamᵪ
  \item \cite[\href{http://ebda.cnr.it/tablet/view/1358\#60317}{\recto~7, 9}]{P240655} {\twoAšC\xsuxfont 𒊺𒁇\footnote{Short for {\xsuxfont 𒄘𒁇}.}𒐊𒃻𒋙𒊕} % r.7,9	2 še bar 5 nig₂-sagšu ARET 09 0044
  \item \cite[\href{http://ebda.cnr.it/tablet/view/1364\#60805}{\verso~4, 3}]{P240579} {\fiveAšC\oneDišC\xsuxfont 𒄘𒁇 𒈫𒃻𒋙𒊕}% 5-1/2 gu₂-bar 2 nig₂-sagšu ARET 09 0050
  \item \cite[\href{http://ebda.cnr.it/tablet/view/1371\#61227}{\verso~2, 2}]{P240675} {\oneAšC\xsuxfont 𒊺 𒁀𒌷𒍪 𒐊𒃻𒋙𒊕}% v.2,2  1 še ba-ri₂-zu 5 nig₂-sagšu ARET 09 0057
  \item \cite[\href{http://ebda.cnr.it/tablet/view/1378\#61605}{\verso~3, 1}]{P240609} {\oneAšC\xsuxfont 𒁀𒌷𒍪 𒐈𒀭𒍡}% ARET 09 0064 1 ba-ri₂-zu 3 an-zamᵪ
  %\item % r.3,3	1 ba-ri₂-zu 8 nig₂-sagšu ninda ARET 09 0064 http://ebda.cnr.it/tablet/view/1378#61586
  \item \cite[\href{http://ebda.cnr.it/tablet/view/1379\#61615}{\recto~3, 3}]{P240533} {\twoUC\oneAšC\oneDišC\xsuxfont 𒄘𒁇 𒐌𒃻𒋙𒊕 𒈫𒈦𒀭𒍡𒍝}  % ARET 09 0065 #61615 20-1-1/2 gu₂-bar 7 nig₂-sagšu 2-1/2 an-zamᵪ za
  \item \cite[\href{http://ebda.cnr.it/tablet/view/1381\#61931}{\recto~1, 5}]{P240697} {\oneAšC\oneDišC\footnote{Note the omitted {\xsuxfont 𒄘𒁇}.}\xsuxfont 𒈫𒂆 𒐈𒀭𒍡}% ARET 09 0067 r.1,5	1-1/2 ❮gu₂-bar❯ 2 GIN₂ 3 an-zamᵪ
  \item \cite[\href{http://ebda.cnr.it/tablet/view/1382\#62088}{\recto~6, 2}]{P240653} {\twoUC\threeAšC\oneDišC\xsuxfont 𒄘𒁇\xsuxfont 𒁹𒃻𒋙𒊕 𒈫𒈦𒀭𒍡} % ARET 09 0068  20-3-1/2 gu₂-bar 1 nig₂-sagšu 2-1/2 an-zamᵪ  (also has 1 ba-ri₂-zu₂ 1 GIN₂ elsewhere)
  \item \cite[\href{http://ebda.cnr.it/tablet/view/1383\#62337}{\recto~2, 6}]{P240654} {\oneAšC\xsuxfont 𒁀𒌷𒍪 𒐋𒋡\footnote{Instead of the expected {\xsuxfont 𒃻𒋙𒊕}.} 𒐈𒀭𒍡\footnote{{\xsuxfont 𒐈𒀭𒍡} not legible on the EbDA photo.}}% 1 ba-ri₂-zu 6 sila₃ 3 an-zamᵪ % ARET 09 0069
  \item \cite[\href{http://ebda.cnr.it/tablet/view/1415\#63792}{\recto~1, 8}]{P240531} {\oneAšC\xsuxfont 𒁀𒌷𒍪 𒐋𒃻𒋙𒊕 𒊺𒍥𒄖}% r.1,8	1 ba-ri₂-zu 6 nig₂-sagšu še-zid₂-gu ARET 09 101 
  \item \cite[\href{http://ebda.cnr.it/tablet/view/3173\#154945}{\recto~1, 1}]{P241708}\footnote{From CDLI photo.} {\twoAšC\oneDišC\xsuxfont 𒆗 𒄘𒁇 𒐌𒃻𒋙𒊕}% 2 1/2 sig₁₅ gu₂-bar 7 nig₂-sagšu  EST 50 https://cdli.mpiwg-berlin.mpg.de/artifacts/241708/reader/133004
  \item \cite[\href{http://ebda.cnr.it/tablet/view/3183\#155340}{\recto~1, 1}]{P241904}\footnote{From photo in \cite[6]{Archi1989}.} {\fourAšC\xsuxfont 𒄘𒁇 𒐉\footnote{Laid out as {\xsuxfont 𒁹𒁹𒁹𒁹}; on stacking patterns see Section~\ref{stackingPatterns}.}𒃻𒋙𒊕}% EST 60 = RA 083, 001, fig. 1
\end{enumerate}
  %\item \cite[\obverse iii 2]{P240964} {\sixAšC\xsuxfont 𒇷\sixŊešTwoC\oneDišC\xsuxfont 𒄘𒁇\rotatebox[origin=c]{270}{\twoAšC}\xsuxfont 𒋙𒃻𒈫𒀭𒍡} [TODO comment or stick it in its own paragraph] % EbDA http://ebda.cnr.it/tablet/view/3184#155353, EST 061.
Note that higher numbers of {\text{\xsuxfont 𒄘𒁇}} are expressed in hundreds (\emph{mi-at} {\xsuxfont 𒈪𒀜})
and then thousands (\emph{li-im} {\xsuxfont 𒇷𒅎}),
as is typical in Ebla \cite[33]{Archi2015},
\exempligratia, in \cite[\href{https://ebda.cnr.it/tablet/view/1324\#58019}{\verso~2, 3}]{P240532},
{\oneAšC\xsuxfont 𒈪𒀜\oneŊešTwoC\threeUC\fiveAšC\xsuxfont 𒊺 𒄘𒁇}
($100+60+30+5=195$~{\xsuxfont 𒄘𒁇} of grain). %1 mi-at 60-30-5 še gu₂-bar
These expressions correspond to the following factor diagram:
\begin{equation}
\begin{tikzcd}[column sep=small]
  \smash{\underbrace{\text{\oneAšC\xsuxfont 𒈪𒀜}
  \xleftarrow{\frac{5}{3}}\text{\oneŊešTwoC}
  \xleftarrow{6}\text{\oneUC}
  \xleftarrow{10}\text{\oneAšC}
  \xleftarrow{2}\text{\oneDišC}}_{\text{\xsuxfont 𒄘𒁇}}}=\text{\oneAšC\xsuxfont 𒁀𒊑𒍪}
  &&
  \arrow[ll, "10"]\arrow[ld, "4", start anchor={south west}]\text{\xsuxfont 𒁹𒃻𒋙𒊕}
  \xleftarrow{6}\text{\xsuxfont 𒁹𒀭𒍡}\\
  &\arrow[lu, "\frac{5}{2}", end anchor={south east}]\text{\xsuxfont 𒁹𒂆}
\end{tikzcd}
  \tag{$C_{\text{Ebla}}$}
  \label{systemCEbla}
\end{equation}

\subsubsection{Use in modern publications}
\label{modernUsage}
Because of their prevalence in the Uruk and Early Dynastic periods,
the proposed numerals are widely used in modern publications discussing metrology
in those periods, as illustrated in Figures \ref{EnglundTwoUC}--\ref{GoriTransliteration}.

Since they contrast with the cuneiform numerals, they
likewise appear contrastively in such publications.
A remarkable example of that is found in Figure~\ref{GoriTransliteration}.
The partial\footnote{The untransliterated text would be {\fourAšC\xsuxfont 𒂍𒁕𒌝𒌆𒑊\fourDišC\xsuxfont 𒀀𒋢\fourAšC\xsuxfont 𒔻𒌆𒊷𒁯}; note the atomically encoded $\text{ib₂}\times3\text{\xsuxfont 𒁹}=\text{\xsuxfont 𒌈}\times\text{\xsuxfont 𒐈}=\text{\xsuxfont 𒔻}$.} transliteration ``$4\text{\oneAšC}$
\emph{ʾa₃-da-um} $4\text{\oneDišC}$ \emph{aktum}
$4\text{\oneAšC}$
ib₂\textsuperscript*{tu₉}×$3\text{\xsuxfont 𒁹}$ sa₆ gunu₃''
is used to illustrate a discussion of the interpretation of
the contrast between {\oneAšC} and {\oneDišC} numerals.
More conventional transliterations\footnote{TODO cite the EbDA one.} might omit the numeral shapes entirely, \exempligratia,
$4$ \emph{ʾa₃-da-um} $4$ \emph{aktum} $4$ ib₂\textsuperscript*{tu₉}×$3$ sa₆ gunu₃,
which would obviously be inadequate in this context.
There are transliteration conventions that are more explicit about numeral shape, \exempligratia,
$4(\text{ašᶜ})$ \emph{ʾa₃-da-um} $4(\text{dišᶜ})$ \emph{aktum} $4(\text{ašᶜ})$ ib₂\textsuperscript*{tu₉}×$3(\text{diš})$ sa₆ gunu₃,
but the result would be less readable.
See Section~\ref{transliteration} for a discussion of transliteration conventions for numerals.

\begin{figure}
  \begin{center}
  \includegraphics[width=0.75\textwidth]{friberg-15.png}
  \caption{TODO \cite[15]{Friberg1978}\label{FribergEquations}}
  \end{center}
\end{figure}
\begin{figure}
  \begin{center}
  \includegraphics[width=0.75\textwidth]{friberg-49.png}
  \caption{TODO \cite[49]{Friberg1978}\label{FribergFractions}}
  \end{center}
\end{figure}
\begin{figure}
  \begin{center}
  \includegraphics[width=0.75\textwidth]{englund-132.png}
  \caption{TODO \cite[132]{Englund1988}\label{EnglundTwoUC}}
  \end{center}
\end{figure}
\begin{figure}
  \begin{center}
  \includegraphics[width=0.75\textwidth]{englund-113.png}
  \caption{TODO \cite[113]{Englund1998}\label{englund113}}
  \end{center}
\end{figure}
\begin{figure}
  \begin{center}
  \includegraphics[width=0.75\textwidth]{englund-116.png}
  \caption{TODO \cite[116]{Englund1998}\label{englund116}}
  \end{center}
\end{figure}
\begin{figure}
  \begin{center}
  \includegraphics[width=0.75\textwidth]{krebernik-303.png}
  \caption{TODO \cite[303]{Krebernik1998}\label{krebernik303}}
  \end{center}
\end{figure}
\begin{figure}
  \begin{center}
  \includegraphics[width=0.75\textwidth]{englund-15.png}
  \caption{TODO \cite[132]{Englund2001}\label{englund15}}
  \end{center}
\end{figure}
\begin{figure}
  \begin{center}
  \includegraphics[width=0.75\textwidth]{chambon-6.png}
  \caption{TODO \cite[6]{Chambon2003}\label{chambon6}}
  \end{center}
\end{figure}

\begin{figure}
  \begin{center}
  \includegraphics[width=0.75\textwidth]{chambon-58.png}
  \caption{TODO \cite[58]{Chambon2012}\label{chambon58}}
  \end{center}
\end{figure}
\begin{figure}
  \begin{center}
  \includegraphics[width=0.75\textwidth]{chambon-59.png}
  \caption[TODO]{TODO \cite[59]{Chambon2012}\label{chambon59}\footnotemark}
  \end{center}
\end{figure}
\footnotetext{TODO(egg): On the order cite TSS 188, Friberg2007 p. 148 and any of the usual suspects on the haphazard order of signs in early texts; contrast P274845, P241764.}
\begin{figure}
  \begin{center}
  \includegraphics[width=0.75\textwidth]{chambon-61.png}
  \caption{TODO \cite[61]{Chambon2012}\label{chambon61}}
  \end{center}
\end{figure}
\begin{figure}
  \begin{center}
  \includegraphics[width=0.75\textwidth]{chambon-63.png}
  \caption{TODO \cite[64]{Chambon2012}\label{chambon64}}
  \end{center}
\end{figure}

\begin{figure}
  \begin{center}
  \includegraphics[width=0.75\textwidth]{proust-350.png}
  \caption{TODO\label{proust350}}
  \end{center}
\end{figure}

\begin{figure}
  \begin{center}
  \includegraphics[width=0.75\textwidth]{gori-162.png}
  \caption{Discussion of the contrast between {\oneAšC} and {\oneDišC} numerals in \cite[162]{Gori2023}.\label{gori162}}
  \end{center}
\end{figure}

\begin{figure}
  \begin{center}
  \includegraphics[width=0.75\textwidth]{gori-163.png}
  \caption{Transliteration in \cite[163]{Gori2023} of \cite[\href{http://ebda.cnr.it/tablet/view/217\#13705}{\recto~4, 1}]{P242293} incorporating untransliterated numerals.\label{GoriTransliteration}}
  \end{center}
\end{figure}

\subsection{Non-numeric usage}\label{non-numeric}
\epigraph{
{\nafont 𒊕 𒉆𒁾𒊬 𒁹 𒀸𒁉 \hfill 𒅗𒁉 𒐋𒀀𒀭\\
    \hspace*{2em} 𒐕𒀀𒀭 𒁀𒁺 \hfill 𒅗 𒌤𒁉 𒄿𒍪𒌋}\\
{\nafont 𒊑𒌍 𒁾𒊬𒊒𒋾 𒊓𒀭𒋳𒆪 𒅖𒋼 \hfill 𒋗𒌋 \\
    \hspace*{2em} 𒊑𒁶𒋗 𒋀𒅆𒋗 \hfill 𒁹 𒋗𒅆 𒄿𒆲 \\
    \hspace*{2em} 𒉌𒂍𒋢 \hfill 𒋾𒁲𒂊}\\
{The beginning of the scribal art is a single wedge.
That one has six pronunciations;
it also stands for `sixty'\footnotemark. Do you know its reading\footnotemark?}}{Examenstext A}
\addtocounter{footnote}{-1}
\footnotetext{The reader
will recall that ŋeš₂ is written {\xsuxfont 𒐕}, with a larger wedge than {\xsuxfont 𒁹};
however, these signs have merged by the time Examenstext A is composed.}
\stepcounter{footnote}
\footnotetext{
Besides ŋeš₂, a look at \cite{OSL} shows that the values
diš, ge₃, makkaš, saŋtak₄, and tal₄ are attested both in
\cite{ePSD2} and in lexical lists.
The sign is also used for the Akkadian word \emph{ana} in the Neo-Assyrian period.}

Many of the cuneiform numerals are used with a logographic or phonetic value.
For example, the sign {\xsuxfont 𒀸} has, \emph{inter alia}, the values aš, rum, and dili.
While the horizontal numerals are most frequently
written {\oneAšC} in the Early Dynastic period\footnote{A
CDLI search for \texttt{"(asz@c)"} finds 3296 ED texts,
while a search for \texttt{"(asz)"} finds 81 ED texts,
of which 46 also contain \texttt{"(asz@c)"}.},
such non-numeric usage is almost\footnote{Exceptions are discussed in section~\ref{šar₂}.} always written
{\xsuxfont 𒀸}, for instance:
\begin{itemize}[nosep]
  \item in personal names in administrative texts,
  such as the following, which all contain {\oneAšC} numerals:\begin{itemize}[nosep]
  \item {\xsuxfont 𒀸𒊕} in
  \cites[\reverse~1, 5]{P010424}[\obverse~1, 5]{P010458}[\obverse~2, 5′]{P010459}
  from ED IIIa \smash{\textarabic{أبو صلابيخ}},
  \item {\xsuxfont 𒀸𒉌} in \cite[\obverse~2, 5]{P010960} from ED IIIa Šuruppag,
  % \item {\xsuxfont 𒀸𒁯𒌦𒄷} in \cite[\obverse~7, 18]{P010964} from ED IIIa Šuruppag, % FIXCDLI Nope, 𒀾
  \item {\xsuxfont 𒆷𒈠𒀸𒁯} in \cite[\obverse~4, 3]{P251641} from ED IIIb Adab,
  \item {\xsuxfont 𒅤𒊭𒀸𒁯𒈨} in \cite[\reverse~2, 3]{P252866} from ED IIIb Adab,
  \item {\xsuxfont 𒈗𒋗𒀸𒉌} in \cite[\reverse~2, 4]{P298637} from ED IIIb Umma;
  \end{itemize}
  \item in the Sumerian word {\xsuxfont 𒌑𒀸} u₂-rum, ``property'' in ED IIIb Ŋirsu administrative texts which contain
    {\oneAšC} numerals, such as
    \cites[\obverse~2, 3]{P020006}[\reverse~1, 2]{P020008}[\reverse~1, 2]{P020018}[\obverse~1, 4]{P020024}[\obverse~3, 1]{P020030};
  \item in lexical texts:
  \begin{itemize}[nosep]
    \item in the divine name {\xsuxfont 𒀭𒎏𒉾𒀸𒁇} in the lexical texts \cites[TODO]{P010570}[\obverse~3, 6]{P010572},
      where the entries are prefixed with {\oneAšC}.
    \item in the word {\xsuxfont 𒀸} dili, ``small fish'' in
    \cite[\obverse~2, 5]{P010578}, witness to Early Dynastic Fish,
    \item in the same word with a determinative, {\xsuxfont 𒀸𒄩} dili\textsuperscript*{ku₆},
    in \cite[\obverse~4, 4 \& 6]{P010586},
    witness to Early Dynastic Food, which starts with {\oneAšC} numerals.
  \end{itemize}
\end{itemize}

This is a clear contrast between {\xsuxfont 𒀸} and {\oneAšC} in this period,
and genuine ambiguity can arise if it is lost; for instance, the personal name
{\xsuxfont 𒀸𒊕} occurs on its own line in the aforementioned administrative texts;
a line {\oneAšC\xsuxfont 𒊕} would instead be read as ``one slave''.

\subsection{Limited benefits of diachronic encoding for numerals}
\label{limitedBenefitsOfDiachrony}
The argument in favour of diachronic encoding is that it facilitates
interoperability in a variety of use cases,
as we have outlined in section~\ref{xsuxModel}.
While these benefits are real and now visible for cuneiform signs,
similar considerations are not generally applicable to curviform numerals.

Diachronic reference works such as sign lists and dictionaries tend to not include numbers,
or when they do, they treat them separately, and include signs such as {\xsuxfont 𒀸}
that have both numeric and non-numeric values in both the main list and the section on numbers.
For instance, \cite[123 \psqq]{KWU} lists all of {\xsuxfont 𒀸}--{\xsuxfont 𒐇} together with
{\oneAšC}--{\nineAšC},
while {\xsuxfont 𒀸}, {\xsuxfont 𒐀}, and {\xsuxfont 𒐁}, and only those, appear at the beginning of the sign list,
since they have non-numeric values\footnote{Non-numeric values of {\xsuxfont 𒀸} were discussed in
section~\ref{non-numeric};
{\xsuxfont 𒐀} has the values man₃ and min₅, and is used for the word didli, ``several, various'';
{\xsuxfont 𒐁} has the value eš₆.}.
\cite[58]{PTACE} has the numeric signs {\oneAšC}, {\xsuxfont 𒀹}, {\xsuxfont 𒁹},
whereas non-numeric {\xsuxfont 𒀸} is at the beginning of the sign list,
where its values \emph{aš} and \emph{rum} are listed.
For signs with both non-numeric and numeric usage, \cite{LAK} writes \emph{\textgerman{s. die Zahlz.}}
throughout the main list; LAK~1 {\xsuxfont 𒀸} thus reappears at LAK~829 together with
{\oneAšC}, {\xsuxfont 𒀹}, and {\xsuxfont 𒁹}.
One should note \cite{MZL}, which has numbers throughout the sign list;
but that sign list does not show glyphs predating the Old Babylonian period,
nor does it comprehensively cover the numerals used in the Ur III and Old Babylonian
periods, as, for instance, it does not have {\xsuxfont 𒐒}--{\xsuxfont 𒐔} used in
system~\ref{systemGOB}.

Composite texts rarely have witnesses both from the Early Dynastic period
and later; the kinds of texts that do, chiefly lexical and literary texts, do not
contain numbers to the extent that administrative texts do.
Further, there tend to be changes\footnote{TODO comment on the ED witnesses to the instructions of Šuruppag}
to the text between Early Dynastic
and later witnesses that prevent a diachronic encoding of such composites.
For numerals, the switch from {\oneAšC} to {\xsuxfont 𒁹} numerals prevents
diachronic encoding even if {\oneAšC} were unified with {\xsuxfont 𒀸}.
For instance, the lexical list Early Dynastic Food, already mentioned in section~\ref{non-numeric},
contains some numbers, and has a witness from the Old Akkadian period covering
these numbers: \cite[\href{https://oracc.museum.upenn.edu/dcclt/P215653.3}{a~1′--6′}]{P215653};
however, they are written with {\xsuxfont 𒁹} numerals, whereas they are written with
{\oneAšC} numerals in the Early Dynastic witnesses; since {\xsuxfont 𒁹}
and {\xsuxfont 𒀸} are distinct\footnote{Besides the contrasts in numeric usage
mentioned in section~\ref{capacity}, these characters are clearly
not unifiable because of the many contrasts in non-numeric usage between them;
several values of {\xsuxfont 𒀸} which are not shared with {\xsuxfont 𒁹}
have already been mentioned, but perhaps most striking is the fact that,
in the Neo-Assyrian period, {\xsuxfont 𒀸} is used for the preposition \emph{ina}, ``in'', and
{\xsuxfont 𒁹} for the preposition \emph{ana}, ``to''.}
characters, the {\oneAšC}-{\xsuxfont 𒀸} unification does not help. 

% Maybe cite https://cdli.mpiwg-berlin.mpg.de/artifacts/522091 OB witness to ED Plants?

More generally, since numbers are so deeply tied to metrology, and since
metrological systems change between the Early Dynastic and later
periods\footnote{TODO cite a few things here.}, there is little opportunity
for a diachronic representation of numeric quantities.

In the case of analyses such as
\cite[\emph{sub} ``\href{https://github.com/ARomach/Cuneiform-Stylometry/tree/06278f9f9747a897d00fb0418b6eaa14fa573e83\#adding-corpora}{Adding Corpora}'']{Romach2023},
it is interesting to note that numeric expressions are removed prior
to the conversion of the corpus to Unicode cuneiform for further analysis.

\subsection{Compatibility considerations}
\label{compatibility}
A disunification twenty years after the fact,
affecting all numerals, would ordinarily be a serious
compatibility issue.
Fortunately, with the exception of one character discussed below,
we are not aware of any font using curviform glyphs for the
already-encoded numerals.
In fact we are not aware of any font designed for a style earlier than Old Babylonian,
except for fonts mimicking the representative glyphs from the code charts,
which are primarily Ur III, but sometimes earlier or later,
as described in \cite[\href{https://www.unicode.org/reports/tr56/\#Representative_Glyphs}{§2.4}]{UTR56}.
The lack of dedicated Ur III fonts may be explainable by the chart-like fonts\footnote{Most
prominently Noto Sans Cuneiform, a system font on both Windows—as part of Segoe UI Historic—and macOS.}
being good enough for most purposes;
the lack of Early Dynastic fonts, by the aforementioned issues with numeral unification
making the representation of any text with numerals intractable.

\subsubsection{The case of ŠAR₂}\label{šar₂}
The character U+\textcsc{122B9} {\cuneiformComposite 𒊹} \textsc{cuneiform sign shar2}
has a circular reference glyph.

In most texts from the Early Dynastic IIIb and Old Akkadian period\footnote{For example, in personal names:\begin{itemize}[nosep]
%\item \cite[\reverse~5, 3]{P010263}, ED IIIa literary; % NOTE(egg): Lisman reads that ḫe, so let’s not go there.
\item {\xsuxfont 𒂗𒆠𒊹𒊏} in \cite[\reverse~1, 2]{P020019} from ED IIIb Ŋirsu;
\item {\xsuxfont 𒂄𒈨𒊹𒊏𒁺} in \cite[\obverse~2, 9]{P020182}, also from ED IIIb Ŋirsu; 
\item {\xsuxfont 𒊩𒀭𒊹} in \cite[\obverse~3, 3]{P222186} from ED IIIb Umma;
\item {\xsuxfont 𒌨𒀭𒀀𒊹} in \cite[\obverse~16]{P235312} from Old Akkadian Umma.
\end{itemize}},
a contrast between non-numeric šar₂ written {\xsuxfont 𒊹}
and numeric $1$(šar₂ᶜ) written \oneŠarTwoC{} can be observed,
similar to the contrast between {\xsuxfont 𒀸} and {\oneAšC}
previously discussed in section~\ref{non-numeric}.
However, in lexical lists from Šuruppag and Ebla\footnote{TODO Mention other ways in which
these are archaizing},
as well as in the \emph{\textfrench{Stèle des vautours}},
non-numeric šar₂ is curviform:\begin{itemize}[nosep]
  \item {\originalNoto 𒀭𒂗𒆠𒊹} and {\originalNoto 𒀭𒂗𒊹𒉡𒄄} in \cite[\obverse~10, 10 \& 11]{P010566};
  \item {\originalNoto 𒊹𒄈} and {\originalNoto 𒀭𒊹𒄈} in \cite[\reverse~3, 16 \& 17]{P010576};
  \item {\originalNoto 𒊹𒄷} in \cite[\href{https://oracc.museum.upenn.edu/dcclt/P240986.44}{\recto~3, 3}]{P240986}\footnote{From copy in \cite[No.~397]{ELLes}.};
  \item {\originalNoto 𒊹𒊏𒈾} in \cite[\obverse~17, 9; 18, 11; 22, 12]{P222399}\footnote{Note
  however {\xsuxfont 𒀭𒉌 𒊹𒊏} on \cite[\obverse~6, 17]{P222399}.
  Curviform non-numeric šar₂ is clearly archaizing in ED IIIb Ŋirsu;
  one might suppose that the scribe slipped into their modern ways here.
  TODO add a photo.}.
\end{itemize}

It \emph{would} be disruptive to the diachronic representation of text
if non-numeric šar₂ were to have two different representations.
The character U+\textcsc{122B9} \textsc{cuneiform sign shar2} should therefore be used
in those cases, with its curviform glyph {\xsuxfont 𒊹},
identical to the glyph of the proposed U+\textcsc{12579} \oneŠarTwoC{} \textcsc{CUNEIFORM
NUMERIC SIGN ONE N45}.
Since the archaizing style of texts wherein non-numeric šar₂ is curviform
solidly predates the transition from \oneŠarTwoC{} to {\xsuxfont 𒊹}
in the relevant metrological systems, there is no need to represent
a {\xsuxfont 𒊹}-\oneŠarTwoC{} contrast,
so these characters can have the same glyph in specialist archaizing
Early Dynastic fonts.

Since cuneiform U+\textcsc{122B9} \textsc{cuneiform sign shar2}
effectively merges with U+\textcsc{1212D} {\cuneiformComposite 𒄭} \textsc{cuneiform sign hi},
the reference glyph should remain as it is, \idest, curviform, so that the contrast
between reference glyphs within the Cuneiform block remains clear; see
\cite[\href{https://www.unicode.org/reports/tr56/\#Representative_Glyphs}{§2.4}]{UTR56}.
Since system fonts follow the reference glyphs,
and since extant specialist fonts target styles where U+\textcsc{122B9} is unambiguously cuneiform,
there are no compatibility issues.

Note that in rare cases, such as \cite[\obverse~2, 7]{P222243} from ED IIIa Adab,
non-numeric {\xsuxfont 𒀸} (here with the value rum) is written {\oneAšC}.
It is out of scope for this proposal to decide whether
such occurrences should be treated as anomalous spellings, encoded as
U+\textcsc{12550} \oneAšC{} \textcsc{cuneiform numeric sign one N01},
or as stylistic distinctions, encoded as
U+\textcsc{12038} \textsc{cuneiform sign ash} with a curviform glyph.
in practice this would often be determined by the transliteration from which
the cuneiform text is generated; it is noteworthy that as of this writing,
the CDLI transliteration (UR2-$1$(aš@c)) and the ePSD2 one (uru₈\textsuperscript*{rum})
of this word disagree on that aspect.
Since {\xsuxfont 𒀸} has a cuneiform reference glyph,
this does not pose any compatibility concerns.

\subsubsection{Transliteration}
\label{transliteration}
An important feature of the encoding is that, in order to
support input and bulk conversion of transliterated corpora to
Unicode cuneiform, it should not represent distinctions that are finer
than those recorded in typical transliterations; 
thus, while some older forms of BIL₂ can be described as
{\cuneiformComposite 𒉈}×{\cuneiformComposite 𒆜}
NE×KASKAL or {\cuneiformComposite 𒉈}×{\cuneiformComposite 𒉽}
NE×PAP\footnote{As on \cite{P249253}.},
they are typically all transliterated bil₂,
and therefore are all represented by the character U+\textcsc{1224B} {\cuneiformComposite 𒉋}
\textcsc{CUNEIFORM SIGN NE SHESHIG}, its name notwithstanting,
as described in \cite[\href{https://www.unicode.org/reports/tr56/\#Sign_Names}{§2.5}]{UTR56}.

The situation is more complicated for numbers.
Many transliterations do not represent the type of numeral used,
instead interpreting the whole numeric expression and transcribing
it with delimiters or units as needed to disambiguate.
For instance, {\xsuxfont 𒐕𒌍𒐃𒄥} from \cite[\reverse~21]{P305639}
may be transliterated as $95$~gur, as in \volcite{2}[62]{Feuerherm2004}.
The numerals may also be transliterated separately,
but solely by their values in terms of the overt unit,
as in EbDA transliterations:
the aforementioned {\twoUC\oneAšC\oneDišC\xsuxfont 𒄘𒁇 𒐌𒃻𒋙𒊕 𒈫𒈦𒀭𒍡𒍝}
from \cite[\href{http://ebda.cnr.it/tablet/view/1379\#61615}{\recto~3, 3}]{P240533}
is transliterated ``$20$-$1$-$1/2$~\emph{gu}₂-\emph{bar} $7$~nig₂-sagšu $2$-$1/2$~an-zamₓ\footnote{As
of this writing, EbDA actually has an-zamᵪ, with U+\textcsc{1D6A}
\textcsc{GREEK SUBSCRIPT SMALL LETTER CHI}.} \emph{za}'', reading both
\oneDišC{} and {\xsuxfont 𒈦} as $1/2$,
but not distinguishing them.

In particular, these transliterations do not differentiate between
{\xsuxfont 𒀸} and {\xsuxfont 𒁹} numerals, nor between
\oneAšC{} and \oneDišC{} numerals.
For instance, the aforementioned
{\fourAšC~\xsuxfont 𒂍𒁕𒌝𒌆𒑊 \fourDišC~\xsuxfont 𒀀𒋢 \fourAšC~\xsuxfont 𒔻
𒌆 𒊷𒁯}
from \cite[\href{http://ebda.cnr.it/tablet/view/217\#13705}{\recto~4, 1}]{P242293} is
transliterated ``$4$~\emph{ʾa}₃-\emph{da}-\emph{um}\textsuperscript*{tug₂}-II $4$~aktum\textsuperscript*{tug₂} $4$~ib₂-III gun₃ sa₆ \textsuperscript*{tug₂}''
in EbDA, with no distinction between the \fourAšC{} and \fourDišC.
Since {\xsuxfont 𒀸} and {\xsuxfont 𒁹} numerals are separately encoded,
the numeric expressions in such transliterations cannot be transformed
into Unicode cuneiform without additional context, regardless of
curviform--cuneiform unification.

In metrological systems such as systems~\ref{systemGOB} and~\ref{systemC}
where some units are indicated by the type of numeral
rather than an overt unit sign, it is common practice to add the unit in parentheses
in transliteration; for instance, {\xsuxfont 𒑘𒐃𒀹𒃷𒐌𒊬} from \cite[\obverse~1]{P386847}
is transliterated ``$1$(eše₃) $5\text{½}$~iku\footnote{TODO say something about this reading} $7$~sar'' in \volcite{2}[176]{Feuerherm2004},
and {\xsuxfont 𒑗𒑐𒐌𒈦𒋡} from \cite[\obverse~12]{P307255}
is transliterated ``$1$(n\footnote{TODO comment on nigida.}) $2$(b) $7~\text{½}$~sila₃'' in \volcite{2}[151]{Feuerherm2004}.

This practice has been generalized to systematically indicate numeral shape;
this is in particular the case in CDLI,
where the transliterations of some the above examples are
``$1$(gesz2) $3$(u) $5$(asz) gur'' for {\xsuxfont 𒐕𒌍𒐃𒄥},
``$1$(esze3) $5$(iku) $1/2$(iku) GAN2 $7$(disz) sar'' for {\xsuxfont 𒑘𒐃𒀹𒃷𒐌𒊬},
and ``$3$(barig) $2$(ban2) $7$(disz) $1/2$(disz) sila3'' for {\xsuxfont 𒑗𒑐𒐌𒈦𒋡}.
CDLI and ePSD2 both distinguish curviform from cuneiform numerals in
transliteration: the length
\vphantom{\scalebox{0.85}{\sixŊešTwoC}}{\smash{\sixŊešTwoC}\threeUC\xsuxfont 𒃻𒁺\oneBanTwoC\xsuxfont 𒑌𒄀}
from \cite[\reverse~2, 1]{P020129} is transliterated
``$6$(gesz2@c) $3$(u@c) \{ninda\}nindax(DU) $1/2$(asz@c) $4$(disz@t) gi''
in CDLI, and ``$6$(geš₂ᶜ) $3$(uᶜ) \textsuperscript*{ninda}nindaₓ(DU) $1/2$(ašᶜ) $4$(dišᵗ) gi''
in ePSD2.
Another example is \cite[39]{Molina2014}, which uses $1a$ for {\xsuxfont 𒀸},
$1d$ for {\xsuxfont 𒁹}, $1ac$ for \oneAšC, $1dc$ or $\text{½}dc$ for \oneDišC{}
depending on reading, etc.
The literature on the Uruk and Early Dynastic I--II periods uses
a different set of transliteration conventions that also
disambiguate numeral shapes,
as will be discussed in section~\ref{unificationRationale}.

While there exist transliterations that distinguish {\xsuxfont 𒀸} from {\xsuxfont 𒁹}
but not \fourAšC{} from {\xsuxfont 𒀸}, such as the ones used in \cite{DCCMT},
the trend, especially in more recent works in third millenium studies, seems to be to
represent numeral shape; for example, \cite{Maiocchi2024} gave an example of
the input syntax used by the new ``Urban Economy Begins'' project as
``$10+5\text{c}(\text{GUR})+2(\text{BARIGA})+1(\text{BAN2})$'' for
{\oneUC\fiveAšC\xsuxfont 𒑖𒑏}, with a c
indicating that the GUR numerals are curviform, and the parenthetical
GUR indicating that these are \oneAšC{} rather than \oneDišC{} numerals.
% TODO cite the rant in Englund 2004.

\subsection{Conclusions}
Co-occurences of curviform and cuneiform numerals are not anecdotal
in the Early Dynastic period, nor are they the result of scribal idosyncrasy.
Instead, they represent systematic contrasts between metrological
systems, between individual units within metrological system, and
between numeric usage and phonetic or logographic usage.
This contrastive usage is reflected in modern publications.

While it would be technically possible to handle this contrast as
a stylistic distinction, this approach has no real benefit, and
is highly inconvenient, as it would require any treatment of Early
Dynastic administrative texts to use multiple cuneiform fonts, often within
single numeric expressions.
Further, if that contrast is lost in plain-text interchange, the text can
be misinterpreted:
{\xsuxfont 𒌍} is a length of three ropes, but \threeUC{} is an area of three bur₃;
{\oneAšC\xsuxfont 𒁹} could be read as one {\xsuxfont 𒄘𒁇} and one {\xsuxfont 𒃻𒋙𒊕},
where {\oneAšC\oneDišC} would be one and a half {\xsuxfont 𒄘𒁇};
{\xsuxfont 𒀸𒊕} is a personal name, but {\oneAšC\xsuxfont 𒊕} would be ``one slave''.

At the same time, contrary to most disunifications,
the separate encoding of curviform numerals poses no serious compatibility issues
for existing fonts or encoded corpora,
nor does it, in general,
introduce new issues with transliterated third millenium corpora.
The oddity of {\originalNoto 𒊹} requires some explanation,
but does not pose any architectural issues,
and is not fundamentally different from the other mergers and splits
encountered in the cuneiform script.

\section{Rationale for ED–Uruk numeral unification}
\label{unificationRationale}
A complete rationale for disunification between the non-numeric signs used
in the fourth millenium and the already-encoded cuneiform signs will
be given in the forthcoming proto-cuneiform encoding proposal.
The core issue with extending the cuneiform script further back in time
is that, since 1987, fourth millenium studies have used a different model
of character identity and associated transliteration conventions,
with names being given to structurally different glyphs, and no attempt being made
at assigning phonetic values to them.

This is not a mere classification of glyph variants, as contrastive meanings
of these systematic variants can often be reconstructed,
with, \exempligratia, signs KAŠ\textsubscript{a}, KAŠ\textsubscript{b}, and
KAŠ\textsubscript{c}, depicting filled jars with a spout (a), a handle (c),
or neither (b), being understood as referring to containers of different substances,
see \cite[34\psq]{Englund2001}.
However, not all identified systematic variants are understood, and the general
approach to character identity is closer to that used for undeciphered or
partially deciphered script.

As part of the development of these conventions,
a classification of fourth millenium numeric signs was developed; see \cite{DamerowEnglund1987}.
This classification assigns to each unit numerals an identifier formed by the letter $N$ with a numeric subscript
(sometimes with an additional alphabetic subscript):
$N_1$ is \oneAšC, $N_{14}$ is \oneUC, $N_{34}$ is \oneŊešTwoC, etc.
Transliterations of numeric expression then use those to identify the type of number used,
thus $5N_1$ is \fiveAšC, and $5N_{14}$ is \fiveUC.

In contrast with the use of parenthetical unit names,
this approach does not require interpreting the quantity being counted.
This is valuable in contexts where the exact meaning of the choice of numeral type
is not fully understood, as the transliterations can otherwise force a dubious interpretation.
For instance, the CDLI transliteration of \threeAšC\threeDišC{\xsuxfont 𒌆}  or \sixAšC\sixDišC{} in
\cite[\reverse~1, 6;2, 2]{P283802} currently uses (barig@c) for the vertical numerals,
suggesting a capacity measure; but \cite{Gori2023} interprets these instead
as counting linen textiles.
As a result, the fourth millenium conventions for numeral transliteration
are used in Early Dynastic texts, especially those from the ED I--II period,
even though the Sumerian text uses classical assyriological transliteration conventions.

While the non-numeric signs are treated as undeciphered,
the metrological systems used in the fourth millenium are well understood,
as can be seen in \cite[165]{DamerowEnglund1987}.
As a result, contrary to the non-numeric proto-cuneiform conventions,
these numeric transliteration conventions are compatible with the classical ones described in
section~\ref{transliteration}; they are indeed used interchangeably, as in \cite{P011104} % https://oracc.museum.upenn.edu/epsd2/admin/ed12/P011104
which uses the notation u@f in \cite{ePSD2}, but N14@f in CDLI.
Indeed, the numerals are used similarly in Early Dynastic metrological systems,
and are visually identical.

A disunification of numerals between the third and fourth millenium would
therefore induce confusion as to which numerals should be used in third millenium
studies, and would needlessly duplicate the encoding of at least seventy characters;
by splitting the attestations, these separate encoding proposals would run into additional
difficulties to supply evidence for encoding.

Note that the structural variants designated by letters in fourth millenium notation have systematically
been encoded, as they have occasionally be found to carry distinct numeric meaning.
For instance, \oneNThirtyC{} $N_{30c}$ is listed as a variant of \oneNThirtyA{} $N_{30a}$ in
\cite[166]{DamerowEnglund1987}, where the numeric value of either in relation to \oneNTwentyNineA{} $N_{29a}$  is still unknown,
but their values are found in \cite[33]{Englund2004} to be $\text{\oneNThirtyC}=\frac{1}{10}\text{\oneNTwentyNineA}$, whereas
$\text{\oneNThirtyA}=\frac{1}{6}\text{\oneNTwentyNineA}$.
\section{Considerations on individual numeral series}
[TODO Document to the extent possible the metrological systems in which each sign is used.
Note the disunification of N9 and N10 from 4(ban₂@c) and 5(ban₂@c).]

\section{Characters not included in this proposal}
\subsection{The fractions $\frac13$ and $\frac23$}
% FIXCDLI P271238 o ii 6 is 1/3(asz@c), not @v.
TODO {\xsuxfont 𒑝}, {\xsuxfont 𒑞}. Note the occasional omission of {\xsuxfont 𒋙},
see citations in email to steve plus https://cdli.mpiwg-berlin.mpg.de/artifacts/274845/reader/51537.
\subsection{Missing numerals}
TODO N13 not attested in CDLI 
TODO ($N_{17}$ not usefully numeric, $12N_{14}$ not encodable, etc.). Cite \cite[147]{DamerowEnglund1987}
7 and 8(diš \emph{tenû}) encodable, but not today; want to go into the Cuneiform Numbers and Punctuation block for sanity.
\subsection{Stacking patterns}
\label{stackingPatterns}
The already-encoded numerals in the Cuneiform Numbers and Punctuation block
distinguish some \emph{stacking patterns};
for instance $9\text{\xsuxfont 𒁹}$ is encoded both as U+\textcsc{12446} {\xsuxfont 𒑆} and
as U+\textcsc{1240E} {\xsuxfont 𒐎}.
This is in part due to contrastive usage of stacking patterns.
For instance, besides {\xsuxfont 𒑖} and {\xsuxfont 𒑗} which are characteristic of bariga measures,
four bariga is written {\xsuxfont 𒐉} even where $4${\xsuxfont 𒁹} is written {\xsuxfont 𒐼}, as in
\cites[\obverse~2, 3;\reverse~1, 17]{P255010}[\obverse~4;\reverse~5]{P292843}.
Another contrast is that between the stacking patterns used in scratch calculations in the SPVS,
often {\xsuxfont 𒁹 𒈫 𒐈 𒐼 𒐊 𒐋 𒑂 𒑄 𒑆 𒌋 𒎙 𒌍 𒑩 𒑪}, and results
in metrological systems,
typically {\xsuxfont 𒁹 𒈫 𒐈 𒐉 𒐊 𒐋 𒐌 𒐍 𒐎 𒌋 𒎙 𒌍 𒐏 𒐐},
occasionally co-occurring as in \cites{P142827}{P142357}.
This separate encoding is also for compatibility with distinctions made in
reference works and in some non-numeric transliterations;
for instance, {\xsuxfont 𒐼} is \cite[No.~860]{MZL} and has the value limmu,
and {\xsuxfont 𒐉} is \cite[No.~852]{MZL} and  has the value limmu₅.
Numeric\footnote{The Sumerian word
limmu means ``four'', so limmu and limmu₅ are still numbers. The distinction here is between
usage in transliterations of phrases such as {\xsuxfont 𒈗 𒀭𒌒𒁕 𒐉𒁀𒆤}
lugal an-ub-da limmu₅-ba-ke₄ (king of the four quarters) or of names,
and of numeric expressions such as {\xsuxfont 𒐉𒋡} $4(\text{diš})$ sila₃.}
transliterations occasionally distinguish the stacking patterns {\xsuxfont 𒐼 𒑂 𒑄 𒑆 𒑩 𒑪},
as in the CDLI transliterations of the aforementioned tablets,
although this is rare; often $4(\text{diš})$ is {\xsuxfont 𒐉} in Ur III,
but {\xsuxfont 𒐼} in the Neo-Assyrian period.

However, the stacking patterns from earlier periods
are not separately encoded; for instance,
in ED IIIb Ŋirsu, {\xsuxfont 𒎙} $2$(u) often has one {\xsuxfont 𒌋} atop another.
These older stacking patterns do not appear to be contrastive,
are not marked in transliteration,
and are not listed separately in sign lists
nor assigned any different values.
There is therefore no evidence of a need to encode them;
instead, they should be considered style variants, and an ED IIIb Ŋirsu font should have
an appropriate glyph for U+\textcsc{12399} {\xsuxfont 𒎙} \textcsc{CUNEIFORM SIGN U U}.

Likewise, many stacking patterns are attested for the curviform numerals proposed in this document,
and it is not proposed to separately encode them;
these distinctions would be incompatible with the state of the art in numeric transliterations,
%TODO(egg): Including those by Englund
and are not needed to represent reference works.
Idiosyncratic stacking patterns are in fact particularly common in Early Dynastic and earlier tablets,
as they are structured in rectangular cases rather than lines,
so that numerals may be laid out across the case in whichever way fits the available space;
this is illustrated in Figure~\ref{cheese}.
Note also that the numerals need to be considerably enlarged in order to reproduce the layout of the
tablets, so that \twoUC{} often spans two lines of cuneiform signs, as shown in Figure~\ref{twentyTwoOxen}.
This is impractical when these numerals are set in text that
contrasts them with the larger \oneŊešTwoC{},
and inconsistent with actual practice when
typesetting these numerals, as illustrated in Figure~\ref{EnglundTwoUC}:
reproducing the layout of tablets is not within the scope of plain text.
% TODO(egg): Consider using the : layout as the default and switching the example to P010018.
% Or P010588?
% P010843?
%P011099, last case is probably best.
\begin{figure}
  \newlength{\cheeseWidth}
  \settowidth{\cheeseWidth}{\rotatebox[origin=c]{270}{\LARGE\twoUC}{\LARGE\oneDišC}{\xsuxfont\scriptsize 𒓻}}
\begin{center}
  \parbox{\cheeseWidth}{%
  \linespread{0}\selectfont{}
  \smash{\rotatebox[origin=c]{270}{\LARGE\twoUC}{\LARGE\oneDišC}{\xsuxfont \rlap{\raisebox{-0.2em}{\scriptsize 𒓻}}\raisebox{0.5em}{\hspace*{1pt}\small 𒋗}}}\vphantom{\scalebox{0.6}{\LARGE\twoAšC}}\\
  \vphantom{\scalebox{0.6}{\LARGE\twoAšC}}\hfill \smash{\xsuxfont\scriptsize 𒋳}\\
  \vphantom{\scalebox{0.6}{\LARGE\twoAšC}}\hfill \xsuxfont \smash{\raisebox{0.5em}{𒂵}}\smash{\small 𒉌}\hspace*{1pt}}
  \caption{The layout of case \cites[\reverse~2, 3]{P011099}; the numeral \twoUC{} is rotated to fit the rounded corner of the tablet.}
  \label{cheese}
\end{center}
\end{figure}
\begin{figure}
\newlength{\twentyTwoOxenWidth}
\settowidth{\twentyTwoOxenWidth}{\scalebox{0.5}[1]{\LARGE\twoAšC}{\xsuxfont 𒄞}}
\begin{center}
  {\LARGE\twoUC}\parbox{\twentyTwoOxenWidth}{%
  \linespread{0}\selectfont{}\vphantom{\scalebox{0.6}{\LARGE\twoAšC}}\smash{\scalebox{0.5}[1]{\LARGE\twoAšC}\raisebox{2pt}{\xsuxfont 𒄞}}\\
  \vphantom{\scalebox{0.6}{\LARGE\twoAšC}}{\hfill\smash{\xsuxfont 𒈬}\smash{\xsuxfont 𒀹}}}
  \caption{The layout of case \cites[\obverse~1, 1]{P020066}; the numeral \twoUC{} is spread across two lines.
  The text is read in the order {\twoUC\twoAšC\xsuxfont 𒄞 𒈬𒀹}, ``twenty-two oxen, one year old''.}
  \label{twentyTwoOxen}
\end{center}
\end{figure}
\begin{figure}
\begin{center}
\[
  \text{\nineŊešTwoC}=
  \text{\proposalfont\addfontfeature{Alternate=0}\symbol{"12573}}=
  \text{\proposalfont\addfontfeature{Alternate=1}\symbol{"12573}}
\]
  \caption{Three stacking patterns for U+\textcsc{12573} \textcsc{CUNEIFORM NUMERIC SIGN NINE N34}.
  The one on the left is the reference glyph, used in Uruk III
  \cites[\obverse~1, 1b]{P003499}[\reverse~1, 2]{P004430}, and widely afterwards, \exempligratia,
  ED~IIIa Šuruppag \cite[\obverse~2]{P010678},
  ED~IIIb Ŋirsu \cite[\obverse~1, 3]{P020057},
  Old Akkadian Umma \cite[\obverse~11]{P212464}.
  The ones in the middle and right are used in two Uruk IV tablets
  \cites[\reverse]{P001243}[\reverse~2]{P004500}. All three Uruk examples are transliterated
  $9$(N34) in CDLI.}
  \label{9N34StackingPatterns}
\end{center}
\end{figure}

The reference glyphs use stacking patterns that are common in the Early Dynastic period,
but that are also attested in the Uruk period;
the Uruk period also frequently features numerals that use a more vertical layout,
as illustrated in Figure~\ref{9N34StackingPatterns}.
The later, more horizontal styles were chosen for two reasons:
for the numerals used in the third and fourth millenium,
usage in third millenium scholarship will be more frequent;
and the horizontal layout poses fewer layout difficulties when set
in lines of non-cuneiform text, as most modern scholarship is.
Indeed, the absolute size of the indents \oneAšC, \oneŊešTwoC, \oneUC, and \oneŠarTwoC{}
must remain consistent across the numeral series, lest a \oneŊešTwoC{} numeral be confused with
an \oneAšC{} numeral. Since the single indents are frequently used in running text,
as illustrated in section~\ref{modernUsage}, they need to be large enough that
the vertical stacking patterns are impractical.

Variant stacking patterns, if needed, may be handled
at a higher level as stylistic distinctions;
Figure~\ref{9N34StackingPatterns} uses OpenType stylistic alternates,
and Figure~\ref{cheese} rotates the character \twoUC{},
in both cases preserving the plain text backing.

% FIXCDLI bad transliteration 4(barig) for 4(ban2) in P286521
%\subsection{Matters for higher-level protocols}
%Rotated bits:
%https://cdli.mpiwg-berlin.mpg.de/artifacts/101087

\subsection{Other glyph variants not reflected in transliteration}
TODO Comment on the nameless variant glyphs from L2/23-190 and note that they are illustrating
an even wider glyphic range as shown in \cite{Englund2001}.

\section*{Acknowledgements}
\addcontentsline{toc}{section}{Acknowledgements}

TODO(egg): Something about the Vanséveren fonts
\printbibheading[heading=bibintoc]
\printbibliography[heading=subbibintoc,title={Artefacts},type=artwork]
\printbibliography[heading=subbibintoc,title={Unicode documents},nottype=artwork,keyword=unicode]
\printbibliography[heading=subbibintoc,title={Major reference works and online projects},nottype=artwork,keyword=reference]
\printbibliography[heading=subbibintoc,title={Other documents},nottype=artwork,notkeyword=unicode,notkeyword=reference]
\end{document}