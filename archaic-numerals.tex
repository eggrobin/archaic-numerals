\documentclass[10pt, a4paper, twoside]{article}
\usepackage{fontspec}
\setmainfont[
  Mapping=tex-text, 
  Numbers={OldStyle, Proportional}, 
  Ligatures={TeX, Common}, 
  SmallCapsFeatures={Letters=SmallCaps},
  Contextuals=WordFinal,
            ]{Cambria}
\setmonofont[Scale=MatchLowercase]{Consolas}

\newfontfamily\xsuxfont{NotoSansCuneiform-egg.ttf}
\newfontfamily\obfont{Santakku.ttf}
\newfontfamily\nafont{Assurbanipal.ttf}
\newfontfamily\nbfont{Esagil.ttf}
\newfontfamily\hantfont{NotoSerifTC-Regular.ttf}


\newfontfamily\anshufont{Uni12580ProtoCuneiformChartFont.ttf}
\newcommand\oneNone{\symbol{"12580}}
\newcommand\nineNone{\symbol{"126CB}}
\newcommand\oneNfourteen{\symbol{"12591}}
\newcommand\fiveNfourteen{\symbol{"12674}}
\newcommand\oneNthirtyFour{\symbol{"125BA}}
\newcommand\nineNthirtyFour{\symbol{"126D8}}
\newcommand\oneNfortyEight{\symbol{"125D0}}
\newcommand\fiveNfortyEight{\symbol{"12684}}
\newcommand\oneNfortyFive{\symbol{"125CA}}

\newcommand\oneNtwentyTwo{\symbol{"1259A}}

\newfontfamily\curviform{curviform.ttf}
\newcommand\oneAšC{\symbol{"E55D}}
\newcommand\oneUC{\symbol{"E568}}
\newcommand\oneŊešTwoC{\symbol{"E573}}
\newcommand\oneŊešʾuC{\symbol{"E57E}}
\newcommand\oneŠarʾuC{\symbol{"E58D}}
\newcommand\oneBurʾuC{\symbol{"E593}}
\newcommand\oneBarigaC{\symbol{"E599}}
\newcommand\oneEšeThreeC{\symbol{"E5A3}}

\newfontfamily\cuneiformComposite{CuneiformComposite.ttf}
\newcommand\oneŠarTwoC{{\cuneiformComposite 𒊹}}

\usepackage{mathtools}

\usepackage{unicode-math}
\setmathfont[Scale=MatchLowercase, math-style=ISO]{Cambria Math}

\usepackage[colorlinks,allcolors=Periwinkle]{hyperref}
\usepackage[backend=biber,firstinits=true,maxnames=100,style=alphabetic,maxalphanames=4,doi=true,url=false,eprint=true,labelalpha=true,dateusetime=true]{biblatex}
\addbibresource{bibliography.bib}

% Allow breaking in numbers and after lower and upper case letters in bibliography
% URLs, see https://tex.stackexchange.com/a/134281.
% We use fairly low values to avoid unsightly spacing:
% http : / / example . com is not an improvement over http://ex-
% ample.com.  We prefer breaking in numbers and uppercase letters, which are often
% IDs, rather than lowercase letters, which sometimes form meaningful words, or at
% least tokens that are not customarily broken, e.g. the protocol.
\setcounter{biburlnumpenalty}{100}
\setcounter{biburllcpenalty}{500}
\setcounter{biburlucpenalty}{100}

\AtEveryBibitem{\clearlist{language}}  % TODO(egg): Why are we doing this?

\newcommand\UTCdoc[1]{\href{https://www.unicode.org/cgi-bin/GetMatchingDocs.pl?#1}{#1}}

\DeclareFieldFormat{doi}{%
  \newline
  \mkbibacro{DOI}\addcolon\space
    \ifhyperref
      {\href{https://doi.org/#1}{\nolinkurl{#1}}}
      {\nolinkurl{#1}}}
\DeclareFieldFormat{eprint:cnki}{%
  \newline
  \mkbibacro{CNKI}\addcolon\space
    \ifhyperref
      {\href{http://www.cnki.com.cn/Article/CJFDTOTAL-#1.htm}{\nolinkurl{#1}}}
      {\nolinkurl{#1}}}
\DeclareFieldFormat{eprint:utc}{%
  \newline
  \mkbibacro{UTC}\addcolon\space
    \ifhyperref
      {\UTCdoc{#1}}
      {\nolinkurl{#1}}}
\DeclareFieldFormat{eprint}{%
  \newline
  eprint\addcolon\space
    \ifhyperref
      {\href{#1}{\nolinkurl{#1}}}
      {\nolinkurl{#1}}}
\DeclareFieldFormat{isbn}{%
  \newline
  \mkbibacro{ISBN}\addcolon\space#1}
      
\newcommand{\idest}{\emph{i.e.}}
\newcommand{\exempligratia}{\emph{e.g.}}
\newcommand{\sequentes}{\emph{sqq.}}

\usepackage{epigraph}
\renewcommand{\epigraphsize}{\footnotesize\itshape}
\setlength\epigraphwidth{9cm}
\setlength\epigraphrule{0pt}
\epigraphnoindent

\usepackage[dvipsnames]{xcolor}

\usepackage{enumitem}
\renewcommand\labelitemi{---}

\title{Archaic cuneiform numbers}
\author{Robin Leroy, Anshuman Pandey, and Steve Tinney}

\begin{document}

\maketitle

\tableofcontents

\section{Summary}

\section{Background}

[TODO(egg): Restructure this. The internal references are all garbled.]

The Unicode Standard includes some cuneiform numbers: {\xsuxfont 𒁹}–{\xsuxfont 𒑆} 1–9(diš) and {\xsuxfont 𒀸}–{\xsuxfont 𒐇} 1–9(aš), {\xsuxfont 𒌋}–{\xsuxfont 𒐐} 1–5(u), {\xsuxfont 𒐕}–{\xsuxfont 𒐝} 1–9(ŋeš₂), {\xsuxfont 𒐞}–{\xsuxfont 𒐢} 1–5(ŋešʾu), etc., used in the Sumero-Akkadian Cuneiform script (ISO 15924:
Xsux, Script property value long name: Cuneiform).

In the investigation that led to their encoding in Unicode Version 5.0, it was thought appropriate to unify these with the earlier curviform numerals {\anshufont \oneNone}–{\anshufont \nineNone} 1–9($\text{ašᶜ}=N_{1}$), {\anshufont \oneNfourteen}–{\anshufont \fiveNfourteen} 1–5($\text{uᶜ}=N_{14}$), {\anshufont \oneNthirtyFour}–{\anshufont \nineNthirtyFour} 1–9($\text{ŋeš₂ᶜ}=N_{34}$), {\anshufont \oneNfortyEight}–{\anshufont \fiveNfortyEight} 1–5($\text{ŋešʾuᶜ}=N_{48}$), etc.
It has now become apparent that a distinction needs to be made for the adequate representation
of Early Dynastic (ED) texts and scholarship pertaining to them.

In addition, these numerals will be needed for the representation of proto-cuneiform
texts from the earlier archaic period.
The non-numeric signs of proto-cuneiform  (ISO 15924: Pcun) will be the subject of a separate proposal;
we need only note here that the divergence between the approaches to character identity
in modern scholarship requires that proto-cuneiform be disunified from cuneiform:
proto-cuneiform is effectively treated as an undeciphered script.
In contrast, the cuneiform encoding model is semantic,
requiring an understanding of the text to correctly encode it.

The use of the curviform numeric signs is however understood,
as we will discuss in Section \ref{metrologies};
further, the conventions used for archaic numerals are also used when
discussing ED numerals, see Section \ref{transliteration}.
As a result, the same numerals can be used when encoding archaic and ED texts,
and in order to avoid issues ambiguities in representation when converting from
transliteration, these should be unified.
The overall picture of unifications and disunifications would be as follows:
\begin{center}
\begin{tabular}{| l | l | l | l |} \hline
                  & Uruk III \& earlier & ED – Ur III  & OB \& later \\\hline
Non-numeric signs & Future Pcun   & \multicolumn{2}{|c|}{Existing Xsux} \\\hline
Numbers           & This proposal & \parbox[t]{3cm}{This proposal\\+ Existing Xsux} & Existing Xsux\\\hline
\end{tabular}
\end{center}

\section{Metrologies}
\label{metrologies}
\epigraph{{\obfont 𒁾 𒊬𒊑𒉈 𒂵𒁺} \\ {\obfont 𒁾 𒀸 𒊺 𒄥𒋫 𒍠 {\nafont 𒐞} 𒄥𒂠} \\ {\obfont 𒁾 𒁹 𒂆𒋫 𒍠 𒆬 𒌋 𒈠𒈾𒂠} \\
I want to write tablets: the tablet of 1 \emph{gur} of barley to
600 \emph{gur}; the tablet of 1 shekel of silver to 10 minas […]}{Edubbaʾa D}

In order to explain why TODO:$n$ more numerals are needed,
it is useful to first recall why we have so many kinds of cuneiform numerals already.

As is well known\footnote{See, \exempligratia, \cite[Section 22.3.3 ``Non-Decimal Radix Systems'',
\emph{sub} ``\href{https://www.unicode.org/versions/Unicode16.0.0/core-spec/chapter-22/\#G42894}{Cuneiform Numerals}'']{Unicode16}.}
a sexagesimal place value system (SPVS) was used in Meso\-potamia from the late third millenium onwards.
One should bear in mind, however, that other systems were used;
the SPVS was primarily used in calculations,
with results being expressed in non-positional systems \cites[76]{Robson2008}{Robson2022}.
The digits 1–59 of the SPVS have inner structure which is reflected in the encoding: the digits 1–9 are the individual
characters {\xsuxfont 𒁹}–{\xsuxfont 𒑆}, the multiples of ten (10–50) are {\xsuxfont 𒌋}–{\xsuxfont 𒐐},
but the other digits 11–59 are sequences {\xsuxfont 𒌋𒁹}–{\xsuxfont 𒐐𒑆};
in effect the base-sixty digits are themselves written in base ten, with a different set of symbols for the tens place.
This reflects the origin of the sexagesimal place value system;
it derives from a \emph{non-positional} system, hereafter the \emph{cuneiform discrete counting system} \ref{systemSOB},
which had different signs for the units {\xsuxfont 𒁹}–{\xsuxfont 𒑆},
tens {\xsuxfont 𒌋}–{\xsuxfont 𒐐}, sixties {\xsuxfont 𒐕}–{\xsuxfont 𒐝} (with larger wedges
than the units), six hundreds {\xsuxfont 𒐞}–{\xsuxfont 𒐢},
three thousand six hundreds {\xsuxfont 𒊹}–{\xsuxfont 𒐫}, and thirty-six thousands
 {\xsuxfont 𒐬}–{\xsuxfont 𒐱}.
 
The relations between the values of the signs in the cuneiform discrete counting system
may be summarized by the following factor diagram\footnote{These diagrams,
which have become standard in discussions of Mesopotamian metrology, originate with \cite[10]{Friberg1978},
where they are called \emph{step-diagrams}.},
where the number over arrow indicates the multiple
of the preceding sign (right of the arrow) corresponding to the following sign (left).
\begin{equation}
\text{\xsuxfont 𒐬} \xleftarrow{10}\text{\xsuxfont 𒊹} \xleftarrow{6}\text{\xsuxfont 𒐞} \xleftarrow{10}\text{\xsuxfont 𒐕} \xleftarrow{6}\text{\xsuxfont 𒌋}\xleftarrow{10}\text{\xsuxfont 𒁹}
\tag{S\textsubscript{Ur III/OB}}
\label{systemSOB}
\end{equation}
For example, the number $1729=((\textcolor{BrickRed}{2}\times 10 + \textcolor{Dandelion}{8})\times 6 + \textcolor{NavyBlue}{4})\times 10 + \textcolor{Orchid}{9} = 28\times 60 + 49$
would be written {\xsuxfont \textcolor{BrickRed}{𒐟}\textcolor{Dandelion}{𒐜}\textcolor{NavyBlue}{𒐏}\textcolor{Orchid}{𒑄}} in the discrete counting system,
and {\xsuxfont 𒎙𒑄𒐏𒑆} in the sexagesimal place value system.

The discrete counting system was not the only non-positional system in use in the Ur III and Old Babylonian periods; different systems were in use depending on what was being counted or measured.
For instance, field areas were measured using the following system, where for the named
units we have provided the name of the unit in transliterated Sumerian, normalized Old Babylonian Akkadian,
and the approximate metric equivalent \cites[378]{Friberg2007}{Robson2019}:
\begin{equation}
\text{\xsuxfont 𒐬}
\xleftarrow{10}\text{\xsuxfont 𒊹}
\xleftarrow{6}\text{\xsuxfont 𒐴}
\xleftarrow{\ 10\ }\underset{\mathclap{\substack{1~\text{bur₃}\\1~\text{\emph{būrum}}\\6,48~\text{ha}}}}{\text{\xsuxfont 𒌋}}
\xleftarrow{\quad3\quad}\underset{\mathclap{\substack{1~\text{eše₃}\\1~\text{\emph{eblum}}\\2,16~\text{ha}}}}{\text{\xsuxfont 𒑘}}
\xleftarrow{\quad6\quad}\underset{\mathclap{\substack{1~\text{iku}\\1~\text{\emph{ikûm}}\\3600~\text{m}^2}}}{\text{\xsuxfont 𒀸}}
\xleftarrow{\quad2\quad}\underset{\mathclap{\substack{1~\text{\emph{ubûm}}\\1800~\text{m}^2}}}{\text{\xsuxfont 𒀹}}
\xleftarrow{\ 2\ }\text{\xsuxfont 𒑠}
\tag{G\textsubscript{Ur III/OB}}
\label{systemGOB}
\end{equation}

Note that for the range of areas given above\footnote{For
areas smaller than a quarter \emph{ikûm}, an overt unit is used,
with $1~\text{\emph{mūšarum}}$ ($36~\text{m}^2$) written {\xsuxfont 𒁹𒊬}, equal to one hundredth of an \emph{ikûm},
then sexigesimally subdivided in $60~\text{\xsuxfont{𒂆}}$ (shekels).
For areas greater than $3600~\text{\emph{būrū}}$,
the {\xsuxfont 𒊹}- and {\xsuxfont 𒐬}-numerals are reused with a suffix {\xsuxfont 𒃲} (gal, Sumerian: big),
as follows \cites[\pno 295 with notes b and c]{Robson2008}[378]{Friberg2007}{Robson2019}: \[
\underbrace{\text{\xsuxfont 𒐬𒃲}
\xleftarrow{10}\text{\xsuxfont 𒊹𒃲}
\xleftarrow{6}\text{\xsuxfont 𒐬}
\xleftarrow{10}\text{\xsuxfont 𒊹}
\xleftarrow{6}\text{\xsuxfont 𒐴}
\xleftarrow{10}\text{\xsuxfont 𒌋}
\xleftarrow{3}\text{\xsuxfont 𒑘}
\xleftarrow{6}\text{\xsuxfont 𒀸}
\xleftarrow{2}\text{\xsuxfont 𒀹}
\xleftarrow{2}\text{\xsuxfont 𒑠}}_{\text{\footnotesize\xsuxfont 𒃷}}
\xleftarrow{2,5}\underbrace{\text{\xsuxfont 𒌋}
\xleftarrow{10}\text{\xsuxfont 𒁹}}_{\text{\footnotesize\xsuxfont 𒊬}}
\xleftarrow{6}\underbrace{\text{\xsuxfont 𒌋}
\xleftarrow{10}\text{\xsuxfont 𒁹}}_{\text{\footnotesize\xsuxfont 𒂆}}\text.
\]},
this system does not use any symbols separate from the numerals
for the individual units (\emph{ubûm}, \emph{ikûm}, \emph{eblum}, and \emph{būrum}).
As mentioned in \cite{Robson2019}, the whole numeric expression for the area would be followed by the sign {\xsuxfont 𒃷}
functioning as punctuation, but the numerals are tied to the metrology; thus
a surface of $5~\text{\emph{būrū}}$ $1~\text{\emph{eblum}}$ $4~\text{\emph{ikû}}$ ($100~\text{\emph{ikû}}$, $36~\text{ha}$) would be written\footnote{As in the surface of the field of {\xsuxfont 𒀀𒅗𒋡𒆠} (Apisal) reported on \href{https://cdli.mpiwg-berlin.mpg.de/artifacts/102305/reader/99066}{P102305} r. 1.}
{\xsuxfont 𒐐𒑘𒐂𒃷}. Contrast this with systems
where the same numerals are used for different units,
and overt units are used, as in ``88 acres 3 roods 33 perches''.
Note also that the same signs are shared between multiple systems,
with different relations; the ŠAR₂ sign {\xsuxfont 𒊹} is equal to sixty times the U sign {\xsuxfont 𒌋}
in the area system, but to three hundred and sixty times {\xsuxfont 𒌋} in the discrete counting system.

Another such system of note is the one for capacities\footnote{Used
for volumes of grain, but also oil, dairy products, beer, etc., as well as to express the capacity of boats;
volumes of earthworks instead use system \ref{systemGOB} based on a height of one cubit, see\cites[488]{Powell1987}[294]{Robson2008}{Robson2019}.} \cites[376]{Friberg2007}{Robson2019},
\begin{equation}
\underbrace{
%\text{(as in \ref{systemSOB})}
\text{\xsuxfont 𒐬} \xleftarrow{10}\text{\xsuxfont 𒊹} \xleftarrow{6}\text{\xsuxfont 𒐞} \xleftarrow{10}\text{\xsuxfont 𒐕}
\xleftarrow{6}\text{\xsuxfont 𒌋}
\xleftarrow{10}\underset{\mathclap{\substack{1~\text{gur}\\1~\text{\emph{kurrum}}}}}{\text{\xsuxfont 𒀸}}}_{\text{\xsuxfont 𒄥}}
\xleftarrow{\quad5\quad}\underset{\mathclap{\substack{1~\text{bariga}\\1~\text{\emph{parsiktum}}}}}{\text{\xsuxfont 𒁹}}
\xleftarrow{\quad6\quad}\underset{\mathclap{\substack{1~\text{ban₂}\\1~\text{\emph{sūtum}}}}}{\text{\xsuxfont 𒑏}}
\xleftarrow{\ 10\ }\underset{\mathclap{\substack{1~\text{sila₃}\\1~\text{\emph{qûm}}\\1~\text{l}}}}{\text{\xsuxfont 𒁹𒋡}}\text,
\tag{C\textsubscript{Ur III/OB}}
\label{systemC}
\end{equation}
where the numerals for ban₂ are {\xsuxfont 𒑏}, {\xsuxfont 𒑐}, {\xsuxfont 𒑑}, {\xsuxfont 𒑒},
and {\xsuxfont 𒑔}, and those for bariga are {\xsuxfont 𒁹}, {\xsuxfont 𒑖}, {\xsuxfont 𒑗}, and {\xsuxfont 𒐉} (contrast
ordinary {\xsuxfont 𒈫} and {\xsuxfont 𒐈} otherwise used with {\xsuxfont 𒁹}-numerals).
As described in \cite[\pno 585 with notes (b) and (f)]{Huehnergard2011},
the sign GUR {\xsuxfont 𒄥}, while it is used only with volumes in excess of one gur,
is written after the whole expression,
after the overt unit sign {\xsuxfont 𒋡} if present, and after the word for ``grain'' if present, as in
\[\begin{matrix}
\text{\xsuxfont 𒐢𒐝𒌋𒐂}&
\text{\xsuxfont 𒑑}&
\text{\xsuxfont 𒐋}&
\text{\xsuxfont 𒋡}&
\text{\xsuxfont 𒊺}&
\text{{\xsuxfont 𒄥}\footnotemark}\\
3554~\text{gur}&
3~\text{ban₂}&
6&\text{sila₃}&
\text{ of grain.}
\end{matrix}\]
\footnotetext{From \href{https://cdli.mpiwg-berlin.mpg.de/artifacts/309594}{P309594}.}
Observe that while large numbers of gur follow\footnote{A larger unit, the guru₇ (\emph{karûm}, grain heap), is sometimes used instead, with {\xsuxfont 𒀸𒄦}={\xsuxfont 𒊹𒄥} ($1~\text{\emph{karûm}}=3600~\text{kurrū}$). See \cites[415]{Friberg2007}{Robson2019}.}
system \ref{systemSOB},
the use of horizontal (AŠ) numerals for the gur disambiguates from the vertical bariga,
as {\xsuxfont 𒌋𒁹𒄥} would be $10$~gur $1$~bariga, and {\xsuxfont 𒌋𒀸𒄥} would be $11$~gur;
again even with some overt units, most of the numerals
that participate in a metrological system have an interpretation
dependent on that system. To quote \cite[78]{Robson2008}:
``The SPVS temporarily changed the status of numbers from properties of real-world objects to independent entities that could be manipulated without regard to […] metrological system. […] Once the calculation was done, the result was expressed in the most appropriate metrological units and thus re-entered the natural world as a concrete quantity.''

This intertwining of units and numerals explains the large number of already-encoded numeral series:
\begin{itemize}[nosep]
\item {\xsuxfont 𒁹}–{\xsuxfont 𒑆} used in \ref{systemSOB} and the SPVS as well as with overt units;
\item {\xsuxfont 𒌋}–{\xsuxfont 𒐔} used in \ref{systemGOB}, of which {\xsuxfont 𒌋}–{\xsuxfont 𒐐} are also used in \ref{systemSOB} and the SPVS as well as with overt units;
\item {\xsuxfont 𒐕}–{\xsuxfont 𒐝} used in \ref{systemSOB} and the SPVS;
\item {\xsuxfont 𒀸}–{\xsuxfont 𒐇} used in \ref{systemC} as well as in the weight system;
\item {\xsuxfont 𒑏}, {\xsuxfont 𒑐}, {\xsuxfont 𒑑}, {\xsuxfont 𒑒}, {\xsuxfont 𒑔} used in \ref{systemC};
\item {\xsuxfont 𒁹}, {\xsuxfont 𒑖}, {\xsuxfont 𒑗}, {\xsuxfont 𒐉} used in \ref{systemC}—note the overlap with {\xsuxfont 𒁹}–{\xsuxfont 𒑆};
\item {\xsuxfont 𒑘} and 	{\xsuxfont 𒑙} used in \ref{systemGOB}.
\end{itemize}
%Note that these distinctions are needed even though they are not always represented in transliteration;
%the area {\xsuxfont 𒐐𒑘𒐂𒃷} could be transliterated as 5.1.4, \cite[295]{Robson2008}.

\section{Arguments for curviform–cuneiform unification}
As outlined in, \exempligratia, \cite{UTR56}, the cuneiform encoding model is diachronic;
each character may have wildly different glyphs depending on time period and region.
For instance, the sign IM
may resemble {\cuneiformComposite 𒅎} in texts from Early Dynastic IIIa Šuruppag as in the character code
charts,
{\cuneiformComposite 𒉎} later in the third millenium\footnote{Merging with U+1224E {\xsuxfont 𒉎} NI₂.},
{\obfont 𒅎} in Old Babylonian cursive,
{\nafont 𒅎} in Neo-Assyrian, but is always encoded as U+1214E \textsc{cuneiform sign im}.

This encoding model allows for the interoperable representation of editions
of diachronic reference works such as sign lists\footnote{Notably the online edition of \cite{Borger2010} in \cite[Signs]{eBL}, as well as
\cite{OSL}.}
and dictionaries\footnote{Notably the online edition of \cite{Schramm2010} in \cite[Dictionary]{eBL},
as well as \cite{ePSD2}.},
and of composite texts\footnote{For example, there are Neo-Assyrian and Neo-Babylonian
copies parts of the laws of {\xsuxfont 𒄩𒄠𒈬𒊏𒁉}, as well as Old Babylonian copies in both archaizing
and cursive styles. Some sections are known only from those copies. See \cite[110\psqq]{Oelsner2022}.}.
By being compatible with similarly diachronic transliteration practice
(that is, by avoiding distinctions finer than those made in transliteration),
the encoding model also allows for automated conversion of transliterated
corpora to cuneiform,
which has proven useful as a processing step in analyses such as
\cites{Romach24}{JauhiainenJauhiainen24}\footnote{Attendees
may recall the summary given on the third day of UTC \#180, as recorded in \cite{L2/24-159}.
Other readers may refer to \cite[242,148]{RAI69Abstracts}.}.
The diachronic approach is also useful for pedagogical applications\footnote{For instance,
Old Babylonian grammar may be taught in the Neo-Assyrian script, as in \cite{Caplice2002}.}.

In this context, the argument was made in \cite{L2/04-099} as part of ongoing work on the cuneiform
encoding\footnote{At that time scoped to the répertoire of the Ur III period and later, see \cite[1]{L2/03-162},
although many disunifications, such as $\text{\xsuxfont 𒅎}\ne\text{\xsuxfont 𒉎}$, were informed by Early Dynastic distinctions.}
that the curviform numerals, which occasionally appear in the Ur III period
and are used heavily in the Early Dynastic period,
were a stylistic distinction unifiable with the cuneiform digits, and that
an archaizing Ur III font or an Early Dynastic font could have curviform glyphs for
the appropriate characters; some co-occurrence was known and acknowledged,
but considered to be styling rather than plain text.
Although they had been part of the preliminary proposal \cite{L2/03-393R},
they were therefore removed from \cite{L2/04-036} and \cite{L2/04-189}, which both state that
``The distinction between curved numerals and their cuneiform descendants
is treated as glyphic for the purposes of the present proposal;
this issue will need to be revisited in subsequent encoding phases.''

Indeed, some metrological systems from the Early Dynastic period match the
ones previously mentioned.
In particular, the discrete counting system used in the Early Dynastic period
(and earlier in the Uruk period) clearly mirrors system \ref{systemSOB}
\cites[374]{Friberg2007}[127,165]{DamerowEnglund1987}:
\begin{equation}
\text{\curviform \oneŠarʾuC}
\xleftarrow{10}\text{\oneŠarTwoC}
\xleftarrow{6}\text{\curviform \oneŊešʾuC}
\xleftarrow{10}\text{\curviform \oneŊešTwoC}
\xleftarrow{6}{\text{\curviform \oneUC}}
\xleftarrow{10}{\text{\curviform \oneAšC}}\text3
\tag{S}
\label{systemS}
\end{equation}
Likewise the area system used in the Early Dynastic IIIb period mirrors system \ref{systemGOB}
\cites[63]{NissenDamerowEnglund2007}[378]{Friberg2007}{Gombert2016}:
\begin{equation}
\text{\curviform \oneŠarʾuC}
\xleftarrow{10}\text{\oneŠarTwoC}
\xleftarrow{6}\text{\curviform \oneBurʾuC}
\xleftarrow{10}{\text{\curviform \oneUC}}
\xleftarrow{3}{\text{\curviform \oneEšeThreeC}}
\xleftarrow{6}\text{\curviform \oneAšC}\text,
\tag{G\textsubscript{ED IIIb}}
\label{systemGED}
\end{equation}

The reader will have noticed that in system \ref{systemS}, the vertical
{\xsuxfont 𒁹} from \ref{systemSOB} becomes a horizontal {\curviform \oneAšC}.
This is noted in \cite[4]{L2/04-099}
This is however far from the only case of such a reallocation of function

\section{Limited benefits of diachronic encoding for numerals}

[Composite texts dating back to the period where curved numerals are in use
tend to be limited to lexical texts, which do not usually have numbers.
When they do, diachronic encoding is prevented
by diš-aš distincitons anyway.
Administrative texts, which are where numbers are most prominent,
are not composite.]

[Diachronic reference works tend to not include numbers, or when they do,
to treat them specially (for intance, they are shown at the
end of sign lists such as TODO).]

[The overarching goal of having consistent representation for
equivalent numeric expressions from different periods
is quickly foiled by changes in metrology.]

Note that in \cite{Romach24} [TODO(egg): Cite the GitHub repository],
as in many other such analyses, numbers are removed as an early step
in processing; these therefore would not benefit from diachrony in
the encoding of numeric expressions.

\section{Problems with unification: Early metrology}

\section{Problems with unification: Non-numeric usage}
\epigraph{
{\nafont 𒊕 𒉆𒁾𒊬 𒁹 𒀸𒁉 𒅗𒁉 𒐋𒀀𒀭 𒐕𒀀𒀭 𒁀𒁺 𒅗 𒌤𒁉 𒄿𒍪𒌋}\\
{\nafont 𒊑𒌍 𒁾𒊬𒊒𒋾 𒊓𒀭𒋳𒆪 𒅖𒋼 𒋗𒌋 𒊑𒁶𒋗 𒋀𒅆𒋗 𒋗𒍑 𒄿𒆲 𒉌𒂍𒋢 𒋾𒁲𒂊}\\
{The beginning of the scribal art is a single wedge. That one has six pronunciations; it also stands for `sixty'. Do you know its reading?}}{Examenstext A}

\subsection{The case of ŠAR₂}

\section{Compatibility with transliteration}
\label{transliteration}


\section{The necessity of ED–Uruk numeral identification}

\section{Characters not included in this proposal}
\subsection{Missing numerals}
($N_{17}$, $12N_{14}$, etc.) 7(diš \emph{tenû})
\subsection{Stacking patterns}
(… are a mess, vary within Uruk, and are not transliterated/documented by Englund, so let’s not go there for now.)

\section{Acknowledgements}

\printbibliography
\end{document}