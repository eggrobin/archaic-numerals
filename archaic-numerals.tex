\documentclass[10pt, a4paper, twoside]{article}
\usepackage{fontspec}
\setmainfont[
  Mapping=tex-text, 
  Numbers={OldStyle, Proportional}, 
  Ligatures={TeX, Common}, 
  SmallCapsFeatures={Letters=SmallCaps},
  Contextuals=WordFinal,
            ]{Cambria}
\setmonofont[Scale=MatchLowercase]{Consolas}

\newfontfamily\xsuxfont{NotoSansCuneiform-egg.ttf}
\newfontfamily\obfont{Santakku.ttf}
\newfontfamily\nafont{Assurbanipal.ttf}
\newfontfamily\anshufont{Uni12580ProtoCuneiformChartFont.ttf}
\newfontfamily\hantfont{NotoSerifTC-Regular.ttf}

\usepackage[backend=biber,firstinits=true,maxnames=100,style=alphabetic,maxalphanames=4,doi=true,isbn=false,url=false,eprint=true,labelalpha=true]{biblatex}

\usepackage{mathtools}

\usepackage{unicode-math}
\setmathfont[Scale=MatchLowercase, math-style=ISO]{Cambria Math}

\usepackage[colorlinks]{hyperref}

\usepackage{epigraph}
\renewcommand{\epigraphsize}{\footnotesize\itshape}
\setlength\epigraphwidth{9cm}
\setlength\epigraphrule{0pt}
\epigraphnoindent

\usepackage[dvipsnames]{xcolor}

\usepackage{enumitem}
\renewcommand\labelitemi{---}

\title{Archaic cuneiform numbers}
\author{Robin Leroy and Anshuman Pandey}
\newcommand\oneNone{\symbol{"12580}}
\newcommand\nineNone{\symbol{"126CB}}
\newcommand\oneNfourteen{\symbol{"12591}}
\newcommand\fiveNfourteen{\symbol{"12674}}
\newcommand\oneNthirtyFour{\symbol{"125BA}}
\newcommand\nineNthirtyFour{\symbol{"126D8}}
\newcommand\oneNfortyEight{\symbol{"125D0}}
\newcommand\fiveNfortyEight{\symbol{"12684}}
\newcommand\UTCdoc[1]{\href{https://www.unicode.org/cgi-bin/GetMatchingDocs.pl?#1}{#1}}

\begin{document}

\maketitle

The Unicode Standard includes some cuneiform numbers: {\xsuxfont 𒁹}–{\xsuxfont 𒑆} 1–9(diš) and {\xsuxfont 𒀸}–{\xsuxfont 𒐇} 1–9(aš), {\xsuxfont 𒌋}–{\xsuxfont 𒐐} 1–5(u), {\xsuxfont 𒐕}–{\xsuxfont 𒐝} 1–9(ŋeš₂), {\xsuxfont 𒐞}–{\xsuxfont 𒐢} 1–5(ŋešʾu), etc., used in the Sumero-Akkadian Cuneiform script (ISO 15924:
Xsux, Script property value long name: Cuneiform).

In the investigation that led to their encoding in Unicode Version 5.0, it was thought appropriate to unify these with the earlier curviform numerals {\anshufont \oneNone}–{\anshufont \nineNone} 1–9($\text{ašᶜ}=N_{1}$), {\anshufont \oneNfourteen}–{\anshufont \fiveNfourteen} 1–5($\text{uᶜ}=N_{14}$), {\anshufont \oneNthirtyFour}–{\anshufont \nineNthirtyFour} 1–9($\text{ŋeš₂ᶜ}=N_{34}$), {\anshufont \oneNfortyEight}–{\anshufont \fiveNfortyEight} 1–5($\text{ŋešʾuᶜ}=N_{48}$), etc.,
see \UTCdoc{L2/04-099}.
While the curviform numerals sometimes co-occur with the cuneiform ones,
this was analysed as a stylistic distinction which should not be encoded in plain text.
It has now become apparent that a distinction needs to be made for the adequate representation
of Early Dynastic (ED) texts and scholarship pertaining to them.

In addition, these numerals will be needed for the representation of proto-cuneiform
texts from the earlier archaic period.
The non-numeric signs of proto-cuneiform  (ISO 15924: Pcun) will be the subject of a separate proposal;
we need only note here that the divergence between the approaches to character identity
in modern scholarship requires that proto-cuneiform be disunified from cuneiform:
proto-cuneiform is effectively treated as an undeciphered script.
In contrast, the cuneiform encoding model is semantic,
requiring an understanding of the text to correctly encode it.

The use of the curviform numeric signs is however understood,
as we will discuss in Section \ref{metrologies};
further, the conventions used for archaic numerals are also used when
discussing ED numerals, see Section \ref{transliteration}.
As a result, the same numerals can be used when encoding archaic and ED texts,
and in order to avoid issues ambiguities in representation when converting from
transliteration, these should be unified.
The overall picture of unifications and disunifications would be as follows:
\begin{center}
\begin{tabular}{| l | l | l | l |} \hline
                  & Uruk III \& earlier & ED – Ur III  & OB \& later \\\hline
Non-numeric signs & Future Pcun   & \multicolumn{2}{|c|}{Existing Xsux} \\\hline
Numbers           & This proposal & \parbox[t]{3cm}{This proposal\\+ Existing Xsux} & Existing Xsux\\\hline
\end{tabular}
\end{center}

\section{Metrologies}
\label{metrologies}
\epigraph{{\obfont 𒁾 𒊬𒊑𒉈 𒂵𒁺} \\ {\obfont 𒁾 𒀸 𒊺 𒄥𒋫 𒍠 {\nafont 𒐞} 𒄥𒂠} \\ {\obfont 𒁾 𒁹 𒂆𒋫 𒍠 𒆬 𒌋 𒈠𒈾𒂠} \\
I want to write tablets: the tablet of 1 \emph{gur} of barley to
600 \emph{gur}; the tablet of 1 shekel of silver to 10 minas […]}{Edubbaʾa D}

In order to explain why TODO:$n$ more numerals are needed,
it is useful to first recall why we have so many kinds of cuneiform numerals already.

As is well known\footnote{See, \emph{e.g.}, \emph{The Unicode Standard}, Version 16.0, Section 22.3.3 \emph{Non-Decimal Radix Systems},
\emph{sub} ``\href{https://www.unicode.org/versions/Unicode16.0.0/core-spec/chapter-22/\#G42894}{Cuneiform Numerals}''.}
a sexagesimal place value system (SPVS) was used in Meso\-potamia from the late third millenium onwards.
One should bear in mind, however, that other systems were used;
the SPVS was primarily used in calculations,
with results being expressed in non-positional systems.
The digits 1–59 of the SPVS have inner structure which is reflected in the encoding: the digits 1–9 are the individual
characters {\xsuxfont 𒁹}–{\xsuxfont 𒑆}, the multiples of ten (10–50) are {\xsuxfont 𒌋}–{\xsuxfont 𒐐},
but the other digits 11–59 are sequences {\xsuxfont 𒌋𒁹}–{\xsuxfont 𒐐𒑆};
in effect the base-sixty digits are themselves written in base ten, with a different set of symbols for the tens place.
This reflects the origin of the sexagesimal place value system;
it derives from a \emph{non-positional} system, hereafter the \emph{cuneiform discrete counting system} \ref{systemSOB},
which had different signs for the units {\xsuxfont 𒁹}–{\xsuxfont 𒑆},
tens {\xsuxfont 𒌋}–{\xsuxfont 𒐐}, sixties {\xsuxfont 𒐕}–{\xsuxfont 𒐝} (with larger wedges
than the units), six hundreds {\xsuxfont 𒐞}–{\xsuxfont 𒐢},
three thousand six hundreds {\xsuxfont 𒊹}–{\xsuxfont 𒐫}, and thirty-six thousands
 {\xsuxfont 𒐬}–{\xsuxfont 𒐱}.
 
The relations between the values of the signs in the cuneiform discrete counting system may be summarized as follows,
 where the number over arrow indicates the multiple of the preceding sign (right of the arrow) corresponding to the following sign (left).
\begin{equation}
\text{\xsuxfont 𒐬} \xleftarrow{10}\text{\xsuxfont 𒊹} \xleftarrow{6}\text{\xsuxfont 𒐞} \xleftarrow{10}\text{\xsuxfont 𒐕} \xleftarrow{6}\text{\xsuxfont 𒌋}\xleftarrow{10}\text{\xsuxfont 𒁹}
\tag{S\textsubscript{Ur III/OB}}
\label{systemSOB}
\end{equation}
For example, the number $1729=((\textcolor{BrickRed}{2}\times 10 + \textcolor{Dandelion}{8})\times 6 + \textcolor{NavyBlue}{4})\times 10 + \textcolor{Orchid}{9} = 28\times 60 + 49$
would be written {\xsuxfont \textcolor{BrickRed}{𒐟}\textcolor{Dandelion}{𒐜}\textcolor{NavyBlue}{𒐏}\textcolor{Orchid}{𒑄}} in the discrete counting system,
and {\xsuxfont 𒎙𒑄𒐏𒑆} in the sexagesimal place value system.

The discrete counting system was not the only non-positional system in use in the Ur III and Old Babylonian periods; different systems were in use depending on what was being counted or measured.
For instance, field areas were measured using the following system, where for the named
units we have provided the name of the unit in transliterated Sumerian, normalized Old Babylonian Akkadian,
and the approximate metric equivalent:
\begin{equation}
\text{\xsuxfont 𒐬}
\xleftarrow{10}\text{\xsuxfont 𒊹}
\xleftarrow{6}\text{\xsuxfont 𒐴}
\xleftarrow{\ 10\ }\underset{\mathclap{\substack{1~\text{bur₃}\\1~\text{\emph{būrum}}\\6,48~\text{ha}}}}{\text{\xsuxfont 𒌋}}
\xleftarrow{\quad3\quad}\underset{\mathclap{\substack{1~\text{eše₃}\\1~\text{\emph{eblum}}\\2,16~\text{ha}}}}{\text{\xsuxfont 𒑘}}
\xleftarrow{\quad6\quad}\underset{\mathclap{\substack{1~\text{iku}\\1~\text{\emph{ikûm}}\\3600~\text{m}^2}}}{\text{\xsuxfont 𒀸}}
\xleftarrow{\quad2\quad}\underset{\mathclap{\substack{1~\text{\emph{ubûm}}\\1800~\text{m}^2}}}{\text{\xsuxfont 𒀹}}
\xleftarrow{\ 2\ }\text{\xsuxfont 𒑠}
\tag{G\textsubscript{Ur III/OB}}
\label{systemGOB}
\end{equation}

Note that for the range of areas given above\footnote{For
areas smaller than a quarter \emph{ikûm}, an overt unit is used,
with $1~\text{\emph{mūšarum}}$ ($36~\text{m}^2$) written {\xsuxfont 𒁹𒊬}, equal to one hundredth of an \emph{ikûm},
then sexigesimally subdivided in $60~\text{\xsuxfont{𒂆}}$ (shekels).
For areas greater than $3600~\text{\emph{būrū}}$,
the {\xsuxfont 𒊹}- and {\xsuxfont 𒐬}-numerals are reused with a suffix {\xsuxfont 𒃲} (gal, Sumerian: big),
as follows: \[
\underbrace{\text{\xsuxfont 𒐬𒃲}
\xleftarrow{10}\text{\xsuxfont 𒊹𒃲}
\xleftarrow{6}\text{\xsuxfont 𒐬}
\xleftarrow{10}\text{\xsuxfont 𒊹}
\xleftarrow{6}\text{\xsuxfont 𒐴}
\xleftarrow{10}\text{\xsuxfont 𒌋}
\xleftarrow{3}\text{\xsuxfont 𒑘}
\xleftarrow{6}\text{\xsuxfont 𒀸}
\xleftarrow{2}\text{\xsuxfont 𒀹}
\xleftarrow{2}\text{\xsuxfont 𒑠}}_{\text{\footnotesize\xsuxfont 𒃷}}
\xleftarrow{2,5}\underbrace{\text{\xsuxfont 𒌋}
\xleftarrow{10}\text{\xsuxfont 𒁹}}_{\text{\footnotesize\xsuxfont 𒊬}}
\xleftarrow{6}\underbrace{\text{\xsuxfont 𒌋}
\xleftarrow{10}\text{\xsuxfont 𒁹}}_{\text{\footnotesize\xsuxfont 𒂆}}\text.
\]},
this system does not use any symbols separate from the numerals
for the individual units (\emph{ubûm}, \emph{ikûm}, \emph{eblum}, and \emph{būrum}).
The whole numeric expression for the area would be followed by the sign {\xsuxfont 𒃷}
functioning as punctuation, but the numerals are tied to the metrology; thus
a surface of $5~\text{\emph{būrū}}$ $1~\text{\emph{eblum}}$ $4~\text{\emph{ikû}}$ ($100~\text{\emph{ikû}}$, $36~\text{ha}$) would be written\footnote{As in the surface of the field of {\xsuxfont 𒀀𒅗𒋡𒆠} (Apisal) reported on \href{https://cdli.mpiwg-berlin.mpg.de/artifacts/102305/reader/99066}{P102305} r. 1.}
{\xsuxfont 𒐐𒑘𒐂𒃷}. Contrast this with systems
where the same numerals are used for different units,
and overt units are used, as in ``88 acres 3 roods 33 perches''.
Note also that the same signs are shared between multiple systems,
with different relations; the ŠAR₂ sign {\xsuxfont 𒊹} is equal to sixty times the U sign {\xsuxfont 𒌋}
in the area system, but to three hundred and sixty times {\xsuxfont 𒌋} in the discrete counting system.

Another such system of note is the one for volumes,
\begin{equation}
\underbrace{
%\text{(as in \ref{systemSOB})}
\text{\xsuxfont 𒐬} \xleftarrow{10}\text{\xsuxfont 𒊹} \xleftarrow{6}\text{\xsuxfont 𒐞} \xleftarrow{10}\text{\xsuxfont 𒐕}
\xleftarrow{6}\text{\xsuxfont 𒌋}
\xleftarrow{10}\underset{\mathclap{\substack{1~\text{gur}\\1~\text{\emph{kurrum}}}}}{\text{\xsuxfont 𒀸}}}_{\text{\xsuxfont 𒄥}}
\xleftarrow{\quad5\quad}\underset{\mathclap{\substack{1~\text{bariga}\\1~\text{\emph{parsiktum}}}}}{\text{\xsuxfont 𒁹}}
\xleftarrow{\quad6\quad}\underset{\mathclap{\substack{1~\text{ban₂}\\1~\text{\emph{sūtum}}}}}{\text{\xsuxfont 𒑏}}
\xleftarrow{\ 10\ }\underset{\mathclap{\substack{1~\text{sila₃}\\1~\text{\emph{qûm}}\\1~\text{l}}}}{\text{\xsuxfont 𒁹𒋡}}\text,
\tag{C}
\label{systemC}
\end{equation}
where the numerals for ban₂ are {\xsuxfont 𒑏}, {\xsuxfont 𒑐}, {\xsuxfont 𒑑}, {\xsuxfont 𒑒},
and {\xsuxfont 𒑔}, and those for bariga are {\xsuxfont 𒁹}, {\xsuxfont 𒑖}, {\xsuxfont 𒑗}, and {\xsuxfont 𒐉} (contrast
ordinary {\xsuxfont 𒈫} and {\xsuxfont 𒐈} otherwise used with {\xsuxfont 𒁹}-numerals).
Note that while it is used only with volumes in excess of one gur, the sign GUR {\xsuxfont 𒄥} is written after the whole expression,
after the overt unit sign {\xsuxfont 𒋡} if present, and after the word for ``grain'' if present, as in {\xsuxfont \textcolor{BrickRed}{𒐢𒐝𒌋𒐂}\textcolor{Orange}{𒑑}\textcolor{Dandelion}{𒐋}\textcolor{ForestGreen}{𒋡}\textcolor{NavyBlue}{𒊺}\textcolor{Orchid}{𒄥}}
(\textcolor{BrickRed}{$3554$}~\textcolor{Orchid}{gur} \textcolor{Orange}{$3$~ban₂} \textcolor{Dandelion}{$6$}~\textcolor{ForestGreen}{sila₃} of \textcolor{NavyBlue}{grain}\footnote{From \href{https://cdli.mpiwg-berlin.mpg.de/artifacts/309594}{P309594}.}).
Observe that while large numbers of gur follow\footnote{A larger unit, the guru₇ (\emph{karûm}, grain heap), is sometimes used instead, with {\xsuxfont 𒀸𒄦}={\xsuxfont 𒊹𒄥} ($1~\text{\emph{karûm}}=3600~\text{kurrū}$).} system \ref{systemSOB},
the use of horizontal (AŠ) numerals for the gur disambiguates with the vertical bariga,
as {\xsuxfont 𒌋𒁹𒄥} would be $10$~gur $1$~bariga, and {\xsuxfont 𒌋𒀸𒄥} would be $11$~gur;
again even with some overt units, most of the numerals are tied to the metrology.

This intertwining of units and numerals explains the large number of already-encoded numeral series:
\begin{itemize}[nosep]
\item {\xsuxfont 𒁹}–{\xsuxfont 𒑆} used in \ref{systemSOB} and the SPVS as well as with overt units;
\item {\xsuxfont 𒌋}–{\xsuxfont 𒐔} used in \ref{systemGOB}, of which {\xsuxfont 𒌋}–{\xsuxfont 𒐐} are also used in \ref{systemSOB} and the SPVS as well as with overt units;
\item {\xsuxfont 𒐕}–{\xsuxfont 𒐝} used in \ref{systemSOB} and the SPVS;
\item {\xsuxfont 𒀸}–{\xsuxfont 𒐇} used in \ref{systemC} as well as in the weight system;
\item {\xsuxfont 𒑏}, {\xsuxfont 𒑐}, {\xsuxfont 𒑑}, {\xsuxfont 𒑒}, {\xsuxfont 𒑔} used in TODO;
\item {\xsuxfont 𒁹}, {\xsuxfont 𒑖}, {\xsuxfont 𒑗}, {\xsuxfont 𒐉} used in \ref{systemC}—note the overlap with {\xsuxfont 𒁹}–{\xsuxfont 𒑆};
\item {\xsuxfont 𒑘} and 	{\xsuxfont 𒑙} used in \ref{systemGOB}.
\end{itemize}
%Note that these distinctions are needed even though they are not always represented in transliteration;
%the area {\xsuxfont 𒐐𒑘𒐂𒃷} could be transliterated as 5.1.4, \cite[295]{Robson2008}.

\section{Arguments for curviform-cuneiform unification}

\section{Problems with unification: Early metrology}

\section{Problems with unification: Non-numeric usage}
\epigraph{
{\nafont 𒊕 𒉆𒁾𒊬 𒁹 𒀸𒁉 𒅗𒁉 𒐋𒀀𒀭 𒐕𒀀𒀭 𒁀𒁺 𒅗 𒌤𒁉 𒄿𒍪𒌋}\\
{\nafont 𒊑𒌍 𒁾𒊬𒊒𒋾 𒊓𒀭𒋳𒆪 𒅖𒋼 𒋗𒌋 𒊑𒁶𒋗 𒋀𒅆𒋗 𒋗𒍑 𒄿𒆲 𒉌𒂍𒋢 𒋾𒁲𒂊}\\
{The beginning of the scribal art is a single wedge. That one has six pronunciations; it also stands for `sixty'. Do you know its reading?}}{Examenstext A}

\subsection{The case of ŠAR₂}

\section{Compatibility with transliteration}
\label{transliteration}


\section{The necessity of ED–Uruk numeral identification}

\section{Characters not included in this proposal}
\subsection{Missing numerals}
ED {\xsuxfont 𒐴 𒑏𒑐𒑑𒑒𒑔}
\subsection{Stacking patterns}

\end{document}