\documentclass[10pt, a4paper, twoside]{article}
\usepackage{fontspec}
\usepackage{realscripts}
\setmainfont[
  BoldFont=CambriaB,
  BoldItalicFont=CambriaZ,
  Numbers={OldStyle, Proportional}, 
  Ligatures={TeX, Common}, 
  SmallCapsFeatures={Letters=SmallCaps},
  %Contextuals=WordFinal,
            ]{Cambria}
\setmonofont[Scale=MatchLowercase]{Consolas}
\newcommand{\textcsc}[1]{{\addfontfeature{Letters=UppercaseSmallCaps}#1}}

\usepackage[dvipsnames]{xcolor}
\usepackage{accsupp}
\newfontfamily\cuneiformComposite{CuneiformComposite.ttf}
\newfontfamily\originalNoto{NotoSansCuneiform-UN-Swapped.ttf}
\newfontfamily\xsuxfont{NotoSansCuneiform-egg.ttf}
\newfontfamily\obfont{Santakku.ttf}
\newfontfamily\nafont{Assurbanipal.ttf}
\newfontfamily\nbfont{Esagil.ttf}
\newfontfamily\hantfont{NotoSerifTC-Regular.ttf}
%\usepackage{arabluatex}

\newfontfamily\arabicfont[Script=Arabic]{ScheherazadeNew-Regular.ttf}
%\newcommand{\textarabic}[1]{%\BeginAccSupp{method=pdfstringdef,unicode,ActualText={#1}}%
%\bgroup\textdir TRT%
%\arabfont #1%
%\egroup%\EndAccSupp{}
%}

\newfontfamily\proposalfont{Archaic-Cuneiform-Numerals.ttf}

\newcommand\sevenAshTenu{{\symbol{"1246F}}}
\newcommand\eightAshTenu{{\symbol{"12475}}}
\newcommand\nineAshTenu{{\symbol{"12476}}}
\newcommand\oneAshTimesDishTenu{{\symbol{"12477}}}
\newcommand\twoAshTimesDishTenu{{\symbol{"12478}}}
\newcommand\threeAshTimesDishTenu{{\symbol{"12479}}}
\newcommand\fourAshTimesDishTenu{{\symbol{"1247A}}}
\newcommand\fiveAshTimesDishTenu{{\symbol{"1247B}}}
\newcommand\sixAshTimesDishTenu{{\symbol{"1247C}}}
\newcommand\sevenAshTimesDishTenu{{\symbol{"1247D}}}
\newcommand\eightAshTimesDishTenu{{\symbol{"1247E}}}
\newcommand\nineAshTimesDishTenu{{\symbol{"1247F}}}

\newcommand\oneAšC{{\proposalfont\symbol{"12550}}} % 𒀸
\newcommand\twoAšC{{\proposalfont\symbol{"12551}}} % 𒐀
\newcommand\threeAšC{{\proposalfont\symbol{"12552}}} % 𒐁
\newcommand\fourAšC{{\proposalfont\symbol{"12553}}} % 𒐂
\newcommand\fiveAšC{{\proposalfont\symbol{"12554}}} % 𒐃
\newcommand\sixAšC{{\proposalfont\symbol{"12555}}} % 𒐄
\newcommand\sevenAšC{{\proposalfont\symbol{"12556}}} % 𒐅
\newcommand\eightAšC{{\proposalfont\symbol{"12557}}} % 𒐆
\newcommand\nineAšC{{\proposalfont\symbol{"12558}}} % 𒐇
\newcommand\oneDišC{{\proposalfont\symbol{"12559}}}
\newcommand\twoDišC{{\proposalfont\symbol{"1255A}}}
\newcommand\threeDišC{{\proposalfont\symbol{"1255B}}}
\newcommand\fourDišC{{\proposalfont\symbol{"1255C}}}
\newcommand\fiveDišC{{\proposalfont\symbol{"1255D}}}
\newcommand\sixDišC{{\proposalfont\symbol{"1255E}}}
\newcommand\sevenDišC{{\proposalfont\symbol{"1255F}}}
\newcommand\eightDišC{{\proposalfont\symbol{"12560}}}
\newcommand\nineDišC{{\proposalfont\symbol{"12561}}}
\newcommand\oneUC{{\proposalfont\symbol{"12562}}}
\newcommand\twoUC{{\proposalfont\symbol{"12563}}}
\newcommand\threeUC{{\proposalfont\symbol{"12564}}}
\newcommand\fourUC{{\proposalfont\symbol{"12565}}}
\newcommand\fiveUC{{\proposalfont\symbol{"12566}}}
\newcommand\sixUC{{\proposalfont\symbol{"12567}}}
\newcommand\sevenUC{{\proposalfont\symbol{"12568}}}
\newcommand\eightUC{{\proposalfont\symbol{"12569}}}
\newcommand\nineUC{{\proposalfont\symbol{"1256A}}}
\newcommand\oneŊešTwoC{{\proposalfont\symbol{"1256B}}}
\newcommand\twoŊešTwoC{{\proposalfont\symbol{"1256C}}}
\newcommand\threeŊešTwoC{{\proposalfont\symbol{"1256D}}}
\newcommand\fourŊešTwoC{{\proposalfont\symbol{"1256E}}}
\newcommand\fiveŊešTwoC{{\proposalfont\symbol{"1256F}}}
\newcommand\sixŊešTwoC{{\proposalfont\symbol{"12570}}}
\newcommand\sevenŊešTwoC{{\proposalfont\symbol{"12571}}}
\newcommand\eightŊešTwoC{{\proposalfont\symbol{"12572}}}
\newcommand\nineŊešTwoC{{\proposalfont\symbol{"12573}}}
\newcommand\oneŊešʾuC{{\proposalfont\symbol{"12574}}}
\newcommand\twoŊešʾuC{{\proposalfont\symbol{"12575}}}
\newcommand\threeŊešʾuC{{\proposalfont\symbol{"12576}}}
\newcommand\fourŊešʾuC{{\proposalfont\symbol{"12577}}}
\newcommand\fiveŊešʾuC{{\proposalfont\symbol{"12578}}}
\newcommand\oneŠarTwoC{{\proposalfont\symbol{"12579}}}
\newcommand\twoŠarTwoC{{\proposalfont\symbol{"1257A}}}
\newcommand\threeŠarTwoC{{\proposalfont\symbol{"1257B}}}
\newcommand\fourŠarTwoC{{\proposalfont\symbol{"1257C}}}
\newcommand\fiveŠarTwoC{{\proposalfont\symbol{"1257D}}}
\newcommand\sixŠarTwoC{{\proposalfont\symbol{"1257E}}}
\newcommand\seveŠarTwoC{{\proposalfont\symbol{"1257F}}}
\newcommand\eightŠarTwoC{{\proposalfont\symbol{"12580}}}
\newcommand\nineŠarTwoC{{\proposalfont\symbol{"12581}}}
\newcommand\oneŠarʾuC{{\proposalfont\symbol{"12582}}}
\newcommand\twoŠarʾuC{{\proposalfont\symbol{"12583}}}
\newcommand\threeŠarʾuC{{\proposalfont\symbol{"12584}}}
\newcommand\fourŠarʾuC{{\proposalfont\symbol{"12585}}}
\newcommand\fiveŠarʾuC{{\proposalfont\symbol{"12586}}}
\newcommand\oneEighthIkuC{{\proposalfont\symbol{"12587}}}
\newcommand\oneEighthIkuCV{{\proposalfont\symbol{"12588}}}
\newcommand\oneQuarterIkuC{{\proposalfont\symbol{"12589}}}
\newcommand\oneQuarterIkuCV{{\proposalfont\symbol{"1258A}}}
\newcommand\oneHalfIkuCV{{\proposalfont\symbol{"1258B}}}
\newcommand\oneEšeThreeC{{\proposalfont\symbol{"1258C}}}
\newcommand\oneBurʾuC{{\proposalfont\symbol{"1258E}}}
\newcommand\fiveBurʾuC{{\proposalfont\symbol{"12592}}}
\newcommand\oneBanTwoC{{\proposalfont\symbol{"12593}}}
\newcommand\twoBanTwoC{{\proposalfont\symbol{"12594}}}
\newcommand\threeBanTwoC{{\proposalfont\symbol{"12595}}}
\newcommand\fourBanTwoC{{\proposalfont\symbol{"12596}}}
\newcommand\fiveBanTwoC{{\proposalfont\symbol{"12597}}}
\newcommand\oneThirdCV{{\proposalfont\symbol{"12598}}}
\newcommand\twoThirdsCV{{\proposalfont\symbol{"12599}}}
\newcommand\oneNFiftyOne{{\proposalfont\symbol{"1259A}}}
\newcommand\oneNFiftyFour{{\proposalfont\symbol{"125A3}}}
\newcommand\threeNFiftyFour{{\proposalfont\symbol{"125A5}}}
\newcommand\oneNFiftySix{{\proposalfont\symbol{"125A8}}}
\newcommand\twoNFiftySix{{\proposalfont\symbol{"125A9}}}
\newcommand\oneNTwentyFour{{\proposalfont\symbol{"125AA}}}
\newcommand\oneNTwentySix{{\proposalfont\symbol{"125AB}}}
\newcommand\oneNTwentyEight{{\proposalfont\symbol{"125AC}}}
\newcommand\oneNTwentyNineA{{\proposalfont\symbol{"125AD}}}
\newcommand\oneNTwentyNineB{{\proposalfont\symbol{"125AE}}}
\newcommand\oneNThirtyA{{\proposalfont\symbol{"125AF}}}
\newcommand\oneNThirtyC{{\proposalfont\symbol{"125B0}}}
\newcommand\oneNThirtyD{{\proposalfont\symbol{"125B1}}}
\newcommand\oneNThirtyE{{\proposalfont\symbol{"125B2}}}
\newcommand\oneNThirtyOne{{\proposalfont\symbol{"125B3}}}
\newcommand\oneNThirtyTwo{{\proposalfont\symbol{"125B4}}}
\newcommand\oneNThirtyThree{{\proposalfont\symbol{"125B5}}}
\newcommand\oneNThirtyNineA{{\proposalfont\symbol{"125B6}}}
\newcommand\oneNThirtyNineB{{\proposalfont\symbol{"125BA}}}
\newcommand\threeNThirtyFive{{\proposalfont\symbol{"125CE}}}
\newcommand\fourNThirtyFive{{\proposalfont\symbol{"125CF}}}
\newcommand\fiveNThirtyFive{{\proposalfont\symbol{"125D0}}}
\newcommand\oneNSix{{\proposalfont\symbol{"125D1}}}
\newcommand\oneNTwentyOne{{\proposalfont\symbol{"125DA}}}
\newcommand\oneNThirtyEight{{\proposalfont\symbol{"125DF}}}
\newcommand\oneNFiftyTwo{{\proposalfont\symbol{"125E0}}}
\newcommand\twoNFiftyTwo{{\proposalfont\symbol{"125E1}}}
\newcommand\fourNFiftyTwo{{\proposalfont\symbol{"125E3}}}
\newcommand\oneNSixty{{\proposalfont\symbol{"125E9}}}
\newcommand\oneNTwentyFourA{{\proposalfont\symbol{"125EA}}}
\newcommand\oneNForty{{\proposalfont\symbol{"125EB}}}
\newcommand\fourNForty{{\proposalfont\symbol{"125EE}}}
\newcommand\oneNThree{{\proposalfont\symbol{"125EF}}}
\newcommand\twoNThree{{\proposalfont\symbol{"125F0}}}
\newcommand\oneNEighteen{{\proposalfont\symbol{"125F4}}}
\newcommand\twoNEighteen{{\proposalfont\symbol{"125F5}}}
\newcommand\oneNFortyFiveA{{\proposalfont\symbol{"125FD}}}
\newcommand\oneNTwentyFourB{{\proposalfont\symbol{"125FE}}}
\newcommand\oneNTwentySixB{{\proposalfont\symbol{"125FF}}}
\newcommand\oneNTwentyEightB{{\proposalfont\symbol{"12600}}}
\newcommand\oneNTwentyNineAB{{\proposalfont\symbol{"12601}}}
\newcommand\oneNFortyOne{{\proposalfont\symbol{"12602}}}
\newcommand\oneNFour{{\proposalfont\symbol{"12606}}}
\newcommand\oneNNineteen{{\proposalfont\symbol{"1260B}}}
\newcommand\sevenNNineteen{{\proposalfont\symbol{"12611}}}
\newcommand\oneNFortySix{{\proposalfont\symbol{"12614}}}
\newcommand\oneNThirtySix{{\proposalfont\symbol{"12616}}}
\newcommand\twoNThirtySix{{\proposalfont\symbol{"12617}}}
\newcommand\oneNFortyNine{{\proposalfont\symbol{"1261F}}}
\newcommand\fourNFortyNine{{\proposalfont\symbol{"12622}}}
\newcommand\oneNTwentyFive{{\proposalfont\symbol{"12623}}}
\newcommand\oneNTwentySeven{{\proposalfont\symbol{"12624}}}
\newcommand\oneNTwentyEightC{{\proposalfont\symbol{"12625}}}
\newcommand\oneNTwentyNineAC{{\proposalfont\symbol{"12626}}}
\newcommand\oneNThirtyAC{{\proposalfont\symbol{"12627}}}
\newcommand\oneNThirtyCC{{\proposalfont\symbol{"12628}}}
\newcommand\oneNFortyTwoA{{\proposalfont\symbol{"12629}}}
\newcommand\oneNFortyTwoB{{\proposalfont\symbol{"1262D}}}
\newcommand\oneNFive{{\proposalfont\symbol{"12631}}}
\newcommand\fourNFive{{\proposalfont\symbol{"12634}}}
\newcommand\oneNTwenty{{\proposalfont\symbol{"12636}}}
\newcommand\eightNTwenty{{\proposalfont\symbol{"1263D}}}
\newcommand\oneNFortySeven{{\proposalfont\symbol{"1263F}}}
\newcommand\oneNThirtySeven{{\proposalfont\symbol{"12641}}}
\newcommand\twoNThirtySeven{{\proposalfont\symbol{"12642}}}
\newcommand\oneNTwo{{\proposalfont\symbol{"125BE}}}
\newcommand\sixNTwo{{\proposalfont\symbol{"125C3}}}
\newcommand\oneNFifteen{{\proposalfont\symbol{"125C7}}}
\newcommand\oneNThirtyFive{\proposalfont\symbol{"125CC}}
\newcommand\oneNNine{{\proposalfont\symbol{"12643}}}
\newcommand\oneNEleven{{\proposalfont\symbol{"12644}}}
\newcommand\oneNTwelve{{\proposalfont\symbol{"12645}}}
\newcommand\oneNSevenB{{\proposalfont\symbol{"12649}}}
\newcommand\oneNOneF{{\proposalfont\symbol{"1264C}}}
\newcommand\oneNEightFlat{{\proposalfont\symbol{"12655}}}
\newcommand\oneNFourteenFlat{{\proposalfont\symbol{"12656}}}
\newcommand\oneNThirtyFourFlat{{\proposalfont\symbol{"1265F}}}
\newcommand\oneNFortyFiveFlat{{\proposalfont\symbol{"12668}}}
\newcommand\oneNTwentyTwoFlat{{\proposalfont\symbol{"1266A}}}
\newcommand\oneNFiftyOneFlat{{\proposalfont\symbol{"1266C}}}
\newcommand\oneNThirtyFourFlatTenu{{\proposalfont\symbol{"12675}}}
\newcommand\oneNFourFlat{{\proposalfont\symbol{"12676}}}
\newcommand\twoNFourFlat{{\proposalfont\symbol{"12677}}}
\newcommand\fourNFourFlat{{\proposalfont\symbol{"12679}}}
\newcommand\oneNNineteenFlat{{\proposalfont\symbol{"1267B}}}
\newcommand\twoNNineteenFlat{{\proposalfont\symbol{"1267C}}}
\newcommand\oneNFortySixFlat{{\proposalfont\symbol{"12684}}}
\newcommand\twoNFortySixFlat{{\proposalfont\symbol{"12685}}}
\newcommand\oneNThirtySixFlat{{\proposalfont\symbol{"12686}}}

\newcommand{\nhphantom}[1]{\sbox0{#1}\hspace{-\the\wd0}}

\usepackage{afterpage}
\usepackage{float}
\usepackage{caption}

\usepackage{tikz-cd}
\usetikzlibrary{babel}

\usepackage{polyglossia}
\setdefaultlanguage[variant=british]{english}
\setotherlanguages{greek,german,russian,french,arabic}
\usepackage{luabidi}
% We use the english/american quote style, i.e., outer double quotes and inner
% single quotes, but british typographic rules (punctuation after the quotation
% marks).
\usepackage[style=english/american]{csquotes}

\usepackage{mathtools}
\usepackage{empheq}

\usepackage{unicode-math}
\setmathfont[Scale=MatchLowercase, math-style=ISO]{Cambria Math}
\setmathfontface\unifrak{UnifrakturMaguntia.ttf}[Scale=MatchLowercase]

\usepackage[colorlinks,allcolors=Periwinkle]{hyperref}
\usepackage[backend=biber,giveninits=true,maxnames=100,style=alphabetic,maxalphanames=4,doi=true,url=false,eprint=true,labelalpha=true,dateusetime=true]{biblatex}
\addbibresource{artefacts.bib}
\addbibresource{artefacts2.bib}
\addbibresource{bibliography.bib}
\addbibresource{unicode.bib}
\DeclareSourcemap{
  \maps{
    \map{
      \step[fieldsource=keywords, match=\regexp{reference}, final]
      \step[fieldsource=entrykey, final]
      \step[fieldset=shorthand, origfieldval]
    }
    \map{
      \step[fieldsource=keywords, match=\regexp{unicode}, final]
      \step[fieldsource=eprint, final]
      \step[fieldset=shorthand, origfieldval]
    }
    \map{
      \pertype{artwork}
      \step[fieldsource=eprint, final]
      \step[fieldset=shorthand, origfieldval]
    }
  }
}

% Allow breaking in numbers and after lower and upper case letters in bibliography
% URLs, see https://tex.stackexchange.com/a/134281.
% We use fairly low values to avoid unsightly spacing:
% http : / / example . com is not an improvement over http://ex-
% ample.com.  We prefer breaking in numbers and uppercase letters, which are often
% IDs, rather than lowercase letters, which sometimes form meaningful words, or at
% least tokens that are not customarily broken, e.g. the protocol.
\setcounter{biburlnumpenalty}{100}
\setcounter{biburllcpenalty}{500}
\setcounter{biburlucpenalty}{100}
\renewcommand\UrlFont{}

\AtEveryBibitem{\clearlist{language}}  % TODO(egg): Why are we doing this?

\DeclareBibliographyAlias{artwork}{report}

\newcommand\UTCdoc[1]{\href{https://www.unicode.org/cgi-bin/GetMatchingDocs.pl?#1}{#1}}
\newcommand\WGtwodoc[1]{\href{http://www.unicode.org/cgi-bin/GetMatchingWG2Docs.pl?#1}{#1}}

\DeclareFieldFormat{doi}{%
  \newline
  \mkbibacro{DOI}\addcolon\space
    \ifhyperref
      {\href{https://doi.org/#1}{#1}}
      {#1}}
\DeclareFieldFormat{eprint:cnki}{%
  \newline
  \mkbibacro{CNKI}\addcolon\space
    \ifhyperref
      {\href{http://www.cnki.com.cn/Article/CJFDTOTAL-#1.htm}{#1}}
      {#1}}
\DeclareFieldFormat{eprint:utc}{%
  \newline
  \mkbibacro{UTC}\addcolon\space
    \ifhyperref
      {\UTCdoc{#1}}
      {#1}}
\DeclareFieldFormat{eprint:wg2}{%
  \newline
  \mkbibacro{ISO/IEC JTC~1/SC~2/WG~2}\addcolon\space
    \ifhyperref
      {\WGtwodoc{#1}}
      {#1}}
\DeclareFieldFormat{eprint:cdli}{%
  \newline
  \mkbibacro{CDLI}\addcolon\space
    \ifhyperref
      {\href{https://cdli.ucla.edu/#1}{#1}}
      {#1}}
\DeclareFieldFormat{eprint:ebda}{%
  \newline
  Eb\textsc{da}\addcolon\space
    \ifhyperref
      {\href{http://ebda.cnr.it/tablet/view/#1}{#1}}
      {#1}}
\DeclareFieldFormat{eprint:oracc}{%
  \newline
  \mkbibacro{ORACC}\addcolon\space
    \ifhyperref
      {\href{http://oracc.org/#1}{#1}}
      {#1}}
\DeclareFieldFormat{eprint:etcsl}{%
  \newline
  \mkbibacro{ETCSL}
    \ifhyperref
      {transliteration: \href{https://etcsl.orinst.ox.ac.uk/cgi-bin/etcsl.cgi?text=c.#1\&display=Crit\&charenc=gtilde}{c.#1};
       translation: \href{https://etcsl.orinst.ox.ac.uk/cgi-bin/etcsl.cgi?text=t.#1\&display=Crit\&charenc=gtilde}{t.#1}}
      {transliteration: {c.#1}
       translation: {t.#1}}}
\DeclareFieldFormat{eprint:louvre}{%
  \newline
  Louvre Collections\addcolon\space
    \ifhyperref
      {\href{https://collections.louvre.fr/#1}{#1}}
      {#1}}
\DeclareFieldFormat{eprint}{%
  \newline
    \ifhyperref
      {\href{#1}{\nolinkurl{#1}}}
      {\nolinkurl{#1}}}
\DeclareFieldFormat{isbn}{%
  \newline
  \mkbibacro{ISBN}\addcolon\space#1}
\renewcommand{\relateddelim}{\newunitpunct}
      
\newcommand{\idest}{\emph{i.e.}}
\newcommand{\exempligratia}{\emph{e.g.}}
\newcommand{\confer}{\emph{cf.}}
\newcommand{\obverse}{obv.}
\newcommand{\reverse}{\IfFontFeatureActiveTF{Numbers=Tabular}{\rlap{rev.}\hphantom{obv.}}{rev.}}
\newcommand{\recto}{\emph{\IfFontFeatureActiveTF{Numbers=Tabular}{\rlap{recto}\hphantom{verso}}{recto}}}
\newcommand{\verso}{\emph{verso}}
\newcommand{\withnote}{n.}
\newcommand{\withnotes}{nn.}

\usepackage{multirow}

\usepackage{epigraph}
\renewcommand{\epigraphsize}{\footnotesize}
\renewcommand{\textflush}{flushepinormal}
%\setlength\epigraphrule{0pt}
\epigraphnoindent
\usepackage{multicol}

\usepackage{enumitem}
\renewcommand\labelitemi{---}
\renewcommand\labelitemii{---}

\usepackage[style=iso]{datetime2}

\hyphenation{cunei-form}
\usepackage{microtype}
\tikzcdset{
arrow style=math font,
}

\renewcommand{\topfraction}{0.9}

\usepackage{luacolor}
\usepackage{lua-ul}
\LuaULSetHighLightColor{yellow}
\newcommand{\removed}[1]{\highLight{\strikeThrough{#1}}}
\newcommand{\changed}[1]{\highLight{#1}}

\usepackage{pdfpages}

\usepackage{fvextra}
\usepackage[color=Periwinkle]{attachfile2}
\usepackage{fancyhdr}
\pagestyle{fancy}
\newcommand{\thisDocumentNumber}{L2/24-\changed{\phantom{888} {\xsuxfont 𒉡 𒀠𒌀}}}
\newcommand{\thisDocumentTitle}{Twelve cuneiform \emph{tenû} numerals}
\fancyfoot{}
\fancyhead{}
\fancyhead[LE,RO]{\thepage}
\fancyhead[CO]{\thisDocumentNumber}
\fancyhead[CE]{\thisDocumentTitle}
\renewcommand{\headrulewidth}{0.4pt}

\title{\thisDocumentTitle}
\author{Robin Leroy and Steve Tinney}

\begin{document}

\maketitle
\begin{tikzpicture}[overlay, remember picture]
  \path (current page.north east) ++(-1,-1) node[below left] {\Large\thisDocumentNumber};
\end{tikzpicture}

\tableofcontents

\section{Summary}

This document proposes filling the Cuneiform Numbers and Punctuation block with
twelve cuneiform numerals used in the third millennium.
Three of those are additional numerals in the AŠ (or DIŠ) \emph{tenû} series,
$7${\xsuxfont 𒀹}\footnote{We follow \cites{Gori2023}{Gori2024} and use unit numerals
rather than sign names in transliterations to indicate the type of numeral.
Contrary to Gori, we write the multiplicity of the sign rather than its value,
as in ATF; thus $3${\xsuxfont 𒌋} for both $3$(bur₃) and $3$(u),
rather than $30${\xsuxfont 𒌋} for the latter.}--$9${\xsuxfont 𒀹},
where $1\text{\xsuxfont 𒀹}=\text{\xsuxfont 𒀹}$ through
$6\text{\xsuxfont 𒀹}=\text{\xsuxfont 𒑎}$ are already encoded.
Their glyphic range and usage are discussed in §\ref{dištenû}.
The other proposed characters constitute a new series of numerals,
formed by {\xsuxfont 𒀹} numerals crossing an {\xsuxfont 𒀸} wedge.
They are discussed in §\ref{aš×dištenû}.

These characters are extensively used in Early Dynastic administrative corpus,
which is published online\footnote{See, \exempligratia, the transcription of \cite{P220703} in
\url{https://build-oracc.museum.upenn.edu/epsd2/P220703/cuneify}.
Note that as of this writing, that page uses the private use area for the characters proposed
in this document, and uses provisionally assigned code points for the characters proposed in \cite{L2/24-210R},
neither of which are portable—web fonts are used for both—\confer{}
\url{https://build-oracc.museum.upenn.edu/epsd2/P131747/cuneify}
for the Ur~III \cite{P131747}, which only uses assigned code points.}
using Unicode cuneiform as part of the \cite{ePSD2} project.
They are also used in publications discussing third millennium administrative texts.
Their absence from the Standard can be explained by the initial scope going back only
to the Ur~III period, and by the explicit exclusion of numbers from the scope of
the Early Dynastic extension; see \cites{L2/12-208}[\pno~19\psq{} \withnote~17]{L2/24-210R}.

\section{Proposed changes to the Standard}
\label{proposal}
\subsection{Core specification text}

No change is needed in the core specification.
\subsection{Code charts}
The code charts for the affected block,
including the character names list with proposed informative aliases, cross references, and informative notes,
are shown on the following pages.
The chart incorporates the annotations proposed in \cite{L2/24-239},
as amended by the names list editor.
A plain text file containing the
\textattachfile{tenu-numerals-ucd/NamesList.txt}{NamesList.txt} lines is
attached to this document.
\vspace{0pt plus 0.5fill}
\begin{center}{}\end{center}
\vfill
\includepdf[pages=-]{tenu-numerals-chart.pdf}
\subsection{Properties}
\label{properties}
Add to the respective UCD files the lines given in this section.
These are available as plain text files attached to this document.
Changes to derived files are not listed.
\fvset{fontsize=\scriptsize,breaklines,breaksymbol=\tiny\ensuremath{\textcolor{red}{\hookrightarrow}}}
\subsubsection{Name, General\_Category, Numeric\_Value, etc.}
Attached: \textattachfile{tenu-numerals-ucd/UnicodeData.txt}{UnicodeData.txt}.
\VerbatimInput{tenu-numerals-ucd/UnicodeData.txt}
\subsubsection{Line\_Break}
Attached: \textattachfile{tenu-numerals-ucd/LineBreak.txt}{LineBreak.txt}.
\VerbatimInput{tenu-numerals-ucd/LineBreak.txt}
\subsubsection{Script}
Attached: \textattachfile{tenu-numerals-ucd/Scripts.txt}{Scripts.txt}.
\VerbatimInput{tenu-numerals-ucd/Scripts.txt}

\section{DIŠ \emph{tenû} numerals}\label{dištenû}
This section discusses the following proposed characters:
\begin{itemize}[nosep]
  \item U+\textcsc{1246F} {\cuneiformComposite\sevenAshTenu} \textcsc{CUNEIFORM NUMERIC SIGN SEVEN ASH TENU}
  \item U+\textcsc{12475} {\cuneiformComposite\eightAshTenu} \textcsc{CUNEIFORM NUMERIC SIGN EIGHT ASH TENU}
  \item U+\textcsc{12476} {\cuneiformComposite\nineAshTenu} \textcsc{CUNEIFORM NUMERIC SIGN NINE ASH TENU}
\end{itemize}
\subsection{Name}
The existing numerals in the {\xsuxfont 𒀹} series are
named U+12039 {\cuneiformComposite 𒀹} \textcsc{CUNEIFORM SIGN ASH ZIDA TENU} for the first one
and U+\textcsc{1244A}–U+\textcsc{1244E} {\cuneiformComposite 𒑊}–{\cuneiformComposite 𒑎} \textcsc{CUNEIFORM NUMERIC SIGN $n$ ASH TENU} for the others.

Some\footnote{Besides \emph{tenû},
the terms \emph{gunû} ``speckled'' (with wedges),
\emph{nutillû} ``unfinished'', and \emph{šeššig} (filled with {\xsuxfont 𒊺} ŠE) are used.
Contrast however the use of \textcsc{CROSSING} rather than
\emph{gilimû}, \textcsc{OPPOSING} rather than \emph{igi-gubû}, or
\textcsc{SQUARED} rather than \emph{limmubi igi-gubû}.} technical terms used in cuneiform character names are derived from the structural descriptions of
cuneiform signs by Akkadian-speaking scribes in late second and first millennium lexical texts.
In particular, the word \emph{tenû} \cites[66\psq]{Gong1993}[32\psqq]{Gong2000}[12\psq]{Gong2003}
is used to describe slanted signs or parts of signs: thus {\xsuxfont 𒋙} is described as {\xsuxfont 𒁇} \emph{tenû} in \cite[\reverse~1~46′]{P365233}\footnote{Note that
while the third millennium {\xsuxfont\addfontfeature{StylisticSet=1} 𒋙} and {\xsuxfont 𒁇} are related by a 45° rotation,
in the Neo-Assyrian style used by this list, these signs look like {\nafont 𒋙} and {\nafont 𒁇},
so that only one wedge is slanted, as noted in \cites[66]{Gong1993}[34]{Gong2000}[12]{Gong2003}.},
{\xsuxfont 𒃸} as {\xsuxfont 𒃷} \emph{tenû} in \cites[\reverse~2~47]{P391514}
and as {\xsuxfont 𒂠} \emph{tenû} [\obverse~2~80]{P467315},
{\xsuxfont 𒄪} as {\xsuxfont 𒄩} \emph{tenû} in \cite[2~33]{P391514},
{\xsuxfont 𒆿} as {\xsuxfont 𒆸} (containing) {\xsuxfont 𒀸} \emph{tenû} in \cite[\obverse 16′]{P365267}\footnote{These
descriptions also spell out the names of the component signs;
as today, they are named after one of their values:
{\xsuxfont 𒃷} \emph{tenû} is written
{\nafont 𒂵𒈾𒋼𒉡𒌑} \emph{ga-na te-nu-u₂} after the value gan₂,
{\xsuxfont 𒂠} \emph{tenû} as {\nafont 𒊺𒋼𒉡𒌑} \emph{še te-nu-u} after the value še₃.
As today, these names are not unique, see \cites[52\psqq]{Gong2000}[17\psq]{Gong2003}, with,
\exempligratia, {\xsuxfont 𒃻} being known today as NINDA, GAR (its character name), and NIG₂,
and by the scribes as {\nafont 𒎏𒁕𒆪} \emph{nin-da-ku}, {\nbfont 𒂵𒊏𒆪} \emph{ga-ra-ku},
and {\nafont 𒉌𒅅} \emph{ni-ig}.}.
In most cases, the direction of the slant not explicitly specified.
The terms \emph{kaba tenû} and \emph{zida tenû}, from Sumerian {\xsuxfont 𒆏} gab₂ ``left'' and {\xsuxfont 𒍣} zid ``right'' respectively,
are used in \cite{P345960}, which contrasts {\xsuxfont 𒀺} described as \emph{kaba tenû} and {\xsuxfont 𒀹} described as \emph{zida tenû}.

In modern transliteration, {\xsuxfont 𒀹} numerals are described as {\xsuxfont 𒀸} \emph{tenû} (ATF: \texttt{asz@t}) or {\xsuxfont 𒁹} \emph{tenû} (ATF: \texttt{disz@t}),
the latter being the norm in \cite{CDLI} transliterations\footnote{For an example of a transliteration using aš tenû, see
\cite[§5.1.8 \reverse~1~4]{Greco2021}; note that only the HTML version uses aš tenû, the PDF uses diš.}.
Informative aliases using \emph{diš tenû} have been recommended for the existing characters in \cite{L2/24-239}.
The proposed names use \textcsc{ASH TENU} for consistency with the already-encoded characters.

\subsection{Ur III usage}
As described in \cite[135]{KWU} (see Figure~\ref{fig-KWU-135}),
slanted signs are used in Ur III economic texts primarily in
subtractive notation with {\xsuxfont 𒇲}\footnote{As noted in \cite[\pno~25 \withnote~40]{L2/24-210R},
the sign {\xsuxfont 𒇲} (lal, ``minus'')
is often ligated with the following numerals,
with the subtrahend placed under a sometimes considerably enlarged {\xsuxfont 𒇲},
similar to the layout of the radical in modern mathematical notation,
see, \exempligratia, \cite[\href{http://oracc.org/epsd2/P020092.69}{\reverse~3~1};
\href{http://oracc.org/epsd2/P020092.70}{2}]{P020092}.
The font used in this document ligates or kerns {\xsuxfont 𒀹} subtrahends,
but does not enlarge the {\xsuxfont 𒇲}.} lal\footnote{Also transliterated la₂, as in \cite{CDLI}.
In the transliterated Ur III corpus on \cite{CDLI}, out of 3304 occurrences of \texttt{(disz@t)},
1971 are in {\xsuxfont 𒇲$n$𒀹} \texttt{la2 $n$(disz@t)}.},
as well as for ordinals\footnote{1583 out of 3304 occurrences are {\xsuxfont $n$𒀹𒄰}
\texttt{$n$(disz@t)-kam}, including 647 after {\xsuxfont 𒇲}}
and for ages of animals in years\footnote{203 occurrences of
\texttt{gu4}, \texttt{ab2}, \texttt{ansze}, or \texttt{dur3} \texttt{$n$(disz@t)}}.
\begin{figure}[htb]
  \begin{center}
  \includegraphics[width=0.75\textwidth]{kwu-p-135.png}
  \end{center}
  \caption{\cite[135]{KWU} \label{fig-KWU-135}}
\end{figure}
\begin{figure}[htb]
  \begin{center}
  \includegraphics[width=0.75\textwidth]{kwu-p-132.png}
  \end{center}
  \caption{\cite[132]{KWU} \label{fig-KWU-132}}
\end{figure}

Accounts of animals giving their ages in years
rarely go beyond three-year old animals.
Subtractive notation, which appears in the ED IIIa period \cite[77]{Robson2008},
is used to compactly express numbers close to a larger round number, \exempligratia,
{\xsuxfont 𒌋𒇲𒀹} $10-1$ instead of {\xsuxfont 𒐎} for $9$,
{\xsuxfont 𒌍𒇲𒑊} $30-2$ instead of {\xsuxfont 𒎙𒐍} for $28$,
or {\xsuxfont 𒐕𒇲𒀹} $60-1$ instead of {\xsuxfont 𒐐𒐎} for $59$;
\confer{} IX instead of VIIII in Roman numerals.
It is therefore usually limited to small
subtrahends\footnote{Of the 1971 Ur III occurrences of
\texttt{lal $n$(disz@t)}, 1930 are with $n\leq2$,
of which 1823 with $n=1$.}.
Larger subtrahends do occur for quantities close to a much larger unit;
however in Ur III, they are often written using {\xsuxfont 𒁹} numerals, as in
\cite[\obverse~2~15]{P109346} \smash{\xsuxfont 𒐉𒂆𒇲𒐌𒊺𒆬𒄀} ``$4$~shekels minus $7$~grains of gold'',
a weight which would otherwise be written
\smash{\xsuxfont 𒐈𒑛𒂆𒐐𒐈𒊺} ``$3+\frac{2}{3}$~shekels and $53$~grains'',
as $180\text{\xsuxfont 𒊺}=1\text{\xsuxfont 𒂆}$. See also Figure~\ref{fig-KWU-132}.

Ordinals with {\xsuxfont 𒀹} numerals are also typically limited to small numbers
or subtractive notation:
many of the attestations of {\xsuxfont $n$𒀹𒄰} ``$n$th''
are in year names\footnote{430 occurrences of \texttt{$n$(disz@t)-kam}
are on lines starting with \texttt{mu}, of which 308 are in {\xsuxfont 𒌋𒇲𒀹}.}, such as
{\xsuxfont 𒈬 𒃸𒄯𒆠 𒀀𒁺 𒑊𒄰𒀸 𒁀𒅆𒌨}
mu kar₂-ḫar\textsuperscript{ki} a-ra₂ $2${\xsuxfont 𒀹}-kam-aš ba-ḫul
``year Karhar was destroyed for the second time'' (31st year of Šulgi’s reign),
{\xsuxfont 𒈬 𒋛𒈬𒊒𒌝𒆠 𒀀𒁺 𒑋𒄰𒀸 𒁀𒅆𒌨} mu si-mu-ru-um\textsuperscript{ki} a-ra₂ $3${\xsuxfont 𒀹}-kam-aš ba-ḫul
``year Simurrum was destroyed for the third time'' (32nd year of Šulgi’s reign), or
{\xsuxfont 𒈬 𒋛𒈬𒊒𒌝𒆠 𒅇 𒇻𒇻𒁍𒌝 𒀀𒁺 𒌋𒇲𒀹𒄰𒀸 𒁀𒅆𒌨}
mu si-mu-ru-um\textsuperscript{ki} u₃ lu-lu-bu-um\textsuperscript{ki} a-ra₂ $1${\xsuxfont 𒌋} lal $1${\xsuxfont 𒀹}-kam-aš
ba-ḫul ``year Simurrum and Lullubum were destroyed for the ninth time'' (44th year of Šulgi’s reign).
Larger ordinals are frequent, in particular for the day of the month, but these are written with
{\xsuxfont 𒁹} numerals, thus {\xsuxfont 𒌓𒐌𒄰} for ``the 7th day'' or {\xsuxfont 𒌓𒎙𒐍𒄰} for ``the 28th day''.

The rarity of the higher {\xsuxfont 𒀹} numerals in the Ur III corpus
likely explains the absence of $7${\xsuxfont 𒀹}--$9${\xsuxfont 𒀹}
from the répertoire of Unicode Version 5.0,
which was aiming to encode a répertoire appropriate for the Ur III period
and later.

\subsection{Early Dynastic usage}
The situation is different in the Early Dynastic corpus.
As described in \cite{L2/24-210R}, {\xsuxfont 𒀹} numerals are used in many
Early Dynastic metrological systems, and in particular in the Early Dynastic IIIb
length system \cites[466]{Powell1987}{Lecompte2016}[289\psq]{Lecompte2020}{Robson2022}[23\psq]{L2/24-210R}
\newlength{\rodRopeWidth}
\settowidth{\rodRopeWidth}{$\underbrace{
  \text{\oneŊešʾuC}
  \xleftarrow{10}\text{\oneŊešTwoC}
  \xleftarrow{6}\text{\oneUC}}_{\text{\xsuxfont 𒃻𒁺}}=
  \underbrace{\text{\oneAšC}
  \xleftarrow{2}\text{\oneBanTwoC}}_{\text{\xsuxfont 𒂠}}$}
\begin{equation}
\begin{tikzcd}[row sep=tiny]
  \hspace*{\rodRopeWidth}
  \mathllap{\text{\xsuxfont 𒐕}
  \xleftarrow{\ \ \ 6\ \ \ }
  {\text{\xsuxfont 𒌋}}
  \xleftarrow{\ \ \ 2\ \ \ }\text{\xsuxfont 𒈦}}&
  \\
  &\arrow[lu, "10" above right, start anchor={west}, end anchor={east}]
  \arrow[ld, "10", start anchor={west}, end anchor={east}]\smash{\underset{\substack{\text{gi}\\\text{reed\vphantom{g}}\\3~\text{m}}}{\underbrace{\text{\xsuxfont 𒀹}}_{\text{\xsuxfont 𒄀}}}}
  \xleftarrow{\ \ 6\ \ }\smash{
    \underset{\mathclap{\substack{\text{kuš₃\vphantom{g}}\\\text{cubit\vphantom{g}}\\50~\text{cm}}}}
    {\underbrace{\text{\xsuxfont 𒀹}}_{\text{\xsuxfont 𒌑\vphantom{𒄀}}\mathrlap{*}}}}
    \xleftarrow{\ \ 3\ \ }\smash{
      \underset{\mathclap{\substack{\text{šu-du₃-a\vphantom{g}}\\\text{double-hand\vphantom{g}}\\17~\text{cm}}}}
      {\underbrace{\text{\xsuxfont 𒀹}}_{\mathclap{\text{\xsuxfont 𒋗𒆕𒀀\vphantom{𒄀}}*}}}}
    \xleftarrow{\ 10\ }\smash{
      \underset{\mathclap{\substack{\text{šu-si\vphantom{g}}\\\text{finger}\\17~\text{mm}}}}
      {\underbrace{\text{\xsuxfont 𒀹}}_{\mathclap{\text{\xsuxfont 𒋗𒋛\vphantom{𒄀}}*}}}}
  \\
  \underset{\substack{\text{\textsuperscript*{ninda}nindaₓ(DU)}\\\text{rod}\\6~\text{m}}}{\underbrace{
  \text{\oneŊešʾuC}
  \xleftarrow{10}\text{\oneŊešTwoC}
  \xleftarrow{6}\text{\oneUC}}_{\text{\xsuxfont 𒃻𒁺}}}=
  \underset{\substack{\text{eše₂}\\\text{rope}\\60~\text{m}}}
  {\underbrace{\text{\oneAšC}
  \xleftarrow{2}\text{\oneBanTwoC}}_{\text{\xsuxfont 𒂠}}}&
\end{tikzcd}
\tag{$L_{\text{ED IIIb}}$}
\label{systemLED}
\end{equation}
where, as in \cite{L2/24-210R}, $*$ indicates prefix units.

While this system has a unit $1~\text{\xsuxfont 𒃻𒁺}=2~\text{\xsuxfont 𒄀}$,
lengths above $1~\text{\xsuxfont 𒄀}$ are only expressed in {\xsuxfont 𒂠},
or equivalently in tens of {\xsuxfont 𒃻𒁺}, and in
half-{\xsuxfont 𒂠} equal to $10~\text{\xsuxfont 𒄀}$.
We can therefore expect $7$--$9~\text{\xsuxfont 𒄀}$ to occur, expressed using
{\xsuxfont 𒀹} numerals.
Indeed, 37 texts in the transliterated ED IIIb corpus on \cite{CDLI}
contain undamaged attestations of either {\xsuxfont \sevenAshTenu 𒄀} or
{\xsuxfont \eightAshTenu 𒄀}\footnote{Of those, 34 have {\xsuxfont \sevenAshTenu 𒄀} and 9 have {\xsuxfont \eightAshTenu 𒄀}.};
some of these attestations are shown in Figures~\ref{figSevenReedsFirst}--\ref{figEightReedsLast}.
However, {\xsuxfont \nineAshTenu 𒄀} is not attested, since instead subtractive notation is used, as in
{\xsuxfont\threeŊešTwoC\fourUC\oneBanTwoC 𒇲𒀹𒄀} in \cite[\obverse~3~3]{P020129},
{\xsuxfont\oneŊešTwoC\fiveUC 𒃻𒁺𒇲𒀹𒄀𒌑𒑋} in \cite[\reverse~2~2]{P221272},
or {\xsuxfont 𒌋𒇲𒀹𒄀} in \cite[\obverse~3~8]{P020304}.

A similar situation occurs in some systems of capacity with {\xsuxfont 𒀹} numerals counting
{\xsuxfont 𒋡} sila₃, so that {\xsuxfont \sevenAshTenu 𒋡} and {\xsuxfont \eightAshTenu 𒋡} are
attested, see Figures~\ref{figSevenSila3} and~\ref{figEightSila3}.

The use of {\xsuxfont 𒀹} numerals for ordinals, especially for days,
is more prevalent in the Early Dynastic period
than in the Ur III period, and the use of subtractive notation is less frequent\footnote{Although
also attested, see, \exempligratia, \cite[\reverse~3~6]{P221346} {\xsuxfont 𒌓𒌋𒇲𒀹𒄰},
\cite[\reverse~2~1]{P221006} {\xsuxfont 𒀉𒌓𒌋𒇲𒀸𒄰}}.
in these numbers. We therefore find attestations of
{\xsuxfont\sevenAshTenu}--{\xsuxfont\nineAshTenu} in ``$n$th day'',
some of which are shown in Figures~\ref{figSeventhDay}--\ref{figNinthDay}.

In Ebla, the {\xsuxfont 𒀹} numerals are primarily used in subtractive notation,
see \cite[\pno~88 \withnote~298; \pno~120 \withnote~465; \pno~167 \withnote~739; \pno~180 \withnote~801]{Gori2024}.
However, contrary to Ur III, {\xsuxfont 𒀹} numerals remain used for large subtrahends,
thus \cite[\pno~101 \withnote~355]{Gori2024} cites occurrences of {\fourUC\xsuxfont 𒇲𒑌} for $36$ and
{\xsuxfont\oneAšC 𒇲𒑎𒈪𒀜}\footnote{Recall that {\xsuxfont 𒈪𒀜} \emph{mi-at} is Eblaite for ``hundred'',
see \cites[33]{Archi2015}[27]{L2/24-210R}.} for $94$.
In particular, \cite[129\psq]{Gori2024} cites occurrences of {\xsuxfont 𒇲$9$𒀹} in Ebla,
shown in Figure~\ref{figEblaLalNine}.

\begin{figure}[H]
  \begin{center}
  \resizebox{.75\textwidth}{!}{
  \includegraphics[height=17mm]{P221254 o 3 7 copy.png}
  \includegraphics[height=17mm]{P221254 o 3 7.png}
  }
  \end{center}
  \caption[]{{\xsuxfont 𒐕𒎙\footnotemark\sevenAshTenu{}𒄀𒊕} ``$501$~m (first) width'' (of a field) in \cite[\obverse~3~7]{P221254} from ED~IIIb Ŋirsu.
  Left: Copy from \cite{AllottedelaFuÿe1908}.
  Right: \cite{CDLI} photograph.}\label{figSevenReedsFirst}
\end{figure}
\footnotetext{The {\xsuxfont 𒌋} numeral here has trapezoidal stylus impressions,
rather than the right-angled triangle typical of Ur III and later.
While \cite[106]{Gori2024} distinguishes rhomboidal impressions from cuneiform ones,
there is no contrast, and one finds a continuous glyphic range of trapezia of various
shapes between the triangular impressions and the rhomboidal ones.
All should be encoded {\xsuxfont 𒌋}, and Early Dynastic fonts should use a
trapezoidal or rhomboidal glyph as stylistically appropriate.
Mechanically, the quadrilateral impressions are made with a stylus rotated counterclockwise
compared to normal wedges, so that a fourth side is impressed by a fourth
face of the stylus opposite the right face: three edges of the left face are impressed.}
\begin{figure}[H]
  \begin{center}
    \resizebox{.75\textwidth}{!}{
  \includegraphics[height=17mm]{P221266 o 1 1 copy.png}
  \includegraphics[height=17mm]{P221266 o 1 1.png}
  }
  \end{center}
  \caption{{\xsuxfont \sevenAshTenu{}𒄀 𒂊 𒆹} ``$21$~m of reed-bed dyke'' (attributed to {\xsuxfont 𒌨𒁮} the farmer) in \cite[\obverse~1~1]{P221266} from ED~IIIb Ŋirsu. Left: Copy from \cite{AllottedelaFuÿe1908}.
  Right: \cite{Louvre} photograph.}\label{figSevenReedsLast}
\end{figure}
\begin{figure}[H]
  \begin{center}
    \resizebox{.75\textwidth}{!}{
  \includegraphics[height=17mm]{P020303 o 2 2 copy.png}
  \includegraphics[height=17mm]{P020303 o 2 2.png}
  }
  \end{center}
  \caption[]{{\xsuxfont\threeŊešTwoC\fourUC 𒃻𒁺\eightAshTenu 𒄀 𒃴\footnotemark{}𒁉 𒌑𒑌} ``$1344$~m, its height $2$~m'' (dimensions of a dyke on the river {\xsuxfont 𒊊𒌉𒁕𒈿𒀀𒀭}) in \cite[\obverse~2~2]{P020303} from ED~IIIb Ŋirsu. Left: Copy from \cite{Marzahn1991}.
  Right: \cite{CDLI} photograph.}
\end{figure}
{This sign contains some hatching ({\xsuxfont 𒃴}×{\xsuxfont 𒆜} or {\xsuxfont 𒃴}×{\xsuxfont 𒊺}).
  A contrast is made in \cite{aBZL} between SUKUD (containnig {\xsuxfont 𒊺}) and GALAM
  It is unclear at this time whether this should be addressed in the encoding.}
\begin{figure}[H]
  \begin{center}
    \resizebox{.75\textwidth}{!}{
  \includegraphics[height=17mm]{P221254 o 1 2 copy.png}
  \includegraphics[height=17mm]{P221254 o 1 2.png}
  }
  \end{center}
  \caption{{\xsuxfont 𒐕𒌋\eightAshTenu{}𒄀𒊕𒁲} ``$444$~m equal widths'' (of a field) in \cite[\obverse~1~2]{P221254}.
  Left: Copy from \cite{AllottedelaFuÿe1908}. 
  Right: \cite{CDLI} photograph.}\label{figEightReedsLast}
\end{figure}
\begin{figure}[H]
  \begin{center}
    \resizebox{.75\textwidth}{!}{
  \includegraphics[height=17mm]{P020182 r 3 5-7 copy.png}
  \includegraphics[height=17mm]{P020182 r 3 5-7.png}
  }
  \end{center}
  \caption{{\xsuxfont \oneAšC 𒃻𒌉𒁕 \sevenAshTenu 𒋡 𒉌} / {\xsuxfont 𒀹𒋡 𒂶} / {\xsuxfont\sevenAshTenu 𒋡 𒅗𒈝} ``$1$~niŋbanda $7$ sila of butter, $1$ sila of cream, $7$ sila of dates''
  in \cite[\reverse~3~5--7]{P020182} from ED~IIIb Ŋirsu.
  Left: Copy from \cite{VS14}.
  Right: \cite{CDLI} photograph.}\label{figSevenSila3}
\end{figure}
\begin{figure}[H]
  \begin{center}
    \resizebox{.75\textwidth}{!}{
  \includegraphics[height=17mm]{P221730 r 2 5-6 copy.png}
  \includegraphics[height=17mm]{P221730 r 2 5-6.png}
  }
  \end{center}
  \caption{{\xsuxfont \eightAshTenu 𒋡 𒉌} / {\xsuxfont\eightAshTenu 𒋡 𒅗𒈝} ``$8$ sila of butter, $8$ sila of dates''
  in \cite[\reverse~2~5--6]{P221730} from ED~IIIb Ŋirsu.
  Left: Copy from \cite{Никольский1908}.
  Right: \cite{CDLI} photograph.}\label{figEightSila3}
\end{figure}
\begin{figure}[H]
  \begin{center}
    \resizebox{.75\textwidth}{!}{
  \includegraphics[height=17mm]{P220703 r 2 7 copy.png}
  \includegraphics[height=17mm]{P220703 r 2 7.png}
  }
  \end{center}
  \caption{{\xsuxfont 𒌓\sevenAshTenu{}𒄰} ``seventh day'' in \cite[\reverse~2~7]{P220703} from ED~IIIb Ŋirsu.
  Left: Copy from \cite{AllottedelaFuÿe1908}. 
  Right: \cite{Louvre} photograph.}\label{figSeventhDay}
\end{figure}
\begin{figure}[H]
  \begin{center}
    \resizebox{.75\textwidth}{!}{
  \includegraphics[height=17mm]{P221590 o 2 3 copy.png}
  \includegraphics[height=17mm]{P221590 o 2 3.png}
  }
  \end{center}
  \caption{{\xsuxfont 𒌓\sevenAshTenu{}𒉌𒆷} ``seventh day passed'' in \cite[\obverse~2~3]{P221590} from ED~IIIb Nippur.
  Left: Copy from \cite{Westenholz1975}. 
  Right: \cite{CDLI} photograph.}
\end{figure}
\begin{figure}[H]
  \begin{center}
    \resizebox{.75\textwidth}{!}{
  \includegraphics[height=17mm]{P220703 r 3 1 copy.png}
  \includegraphics[height=17mm]{P220703 r 3 1.png}
  }
  \end{center}
  \caption{{\xsuxfont 𒌓\eightAshTenu{}𒄰} ``eighth day'' in \cite[\reverse~3~1]{P220703}.
  Left: Copy from \cite{AllottedelaFuÿe1908}. 
  Right: \cite{Louvre} photograph.}\label{figEighthDay}
\end{figure}
\begin{figure}[H]
  \begin{center}
    \resizebox{.75\textwidth}{!}{
  \includegraphics[height=17mm]{P452986 r 1 1.png}
  }
  \end{center}
  \caption{{\xsuxfont 𒌗𒌓\nineAshTenu{}𒉌𒆷} ``ninth day passed'' in \cite[\obverse~1~1]{P452986} (ED~IIIa).
  \cite{CDLI} photograph.}
\end{figure}
\begin{figure}[H]
  \begin{center}
    \resizebox{.75\textwidth}{!}{
  \includegraphics[height=17mm]{P222129 o 1 2 copy.png}
  \includegraphics[height=17mm]{P222129 o 1 2.png}
  }
  \end{center}
  \caption{{\xsuxfont 𒌓\nineAshTenu{}𒆛} ``ninth day'' in \cite[\obverse~1~2]{P222129} from ED~IIIa Šuruppag.
  Left: Copy from \cite{Martin2001}. 
  Right: \cite{CDLI} photograph.}\label{figNinthDay}
\end{figure}
\begin{figure}[H]
  \begin{center}
    \resizebox{.75\textwidth}{!}{
  \includegraphics[height=17mm]{ARET 7 90 r 1 1.png}
  \includegraphics[height=17mm]{ARET 1 29 v 3 2.png}
  }
  \end{center}
  \caption{
    Left: {\xsuxfont\oneUC 𒇲\nineAshTenu{} 𒈠𒈾 𒌓𒆬} ``$9$~minas and $51$~shekels of silver'' in
    \cite[\href{http://ebda.cnr.it/tablet/view/1227\#47825}{\recto~1~1}]{P241283};
    right: {\xsuxfont \twoAšC 𒇲\nineAshTenu{} 𒈠𒈾 𒌓𒆬} ``$1$~mina and $51$~shekels of silver''
    in \cite[\href{http://ebda.cnr.it/tablet/view/30\#6246}{\verso~3~2}]{P241325}, both from Ebla.
    Photographs from \cite{EbDA}.}\label{figEblaLalNine}
\end{figure}

\subsection{Glyphs}
As illustrated in the above figures, the angle of the {\xsuxfont 𒀹}
varies, and is not always faithfully reproduced in copies.
The representative glyphs retain the same angle used for the already-encoded numerals.

The stacking patterns for the proposed characters do not vary among the attestations cited above.
Note that stacking patterns are known to vary for other numerals in this series;
for instance, {\xsuxfont 𒑌} and {\xsuxfont 𒑍} sometimes appear with all wedges in a row in ED~IIIa
tablets, as in \cites{P010787}{P010896}{P010928}.
As discussed in \cite[45\psqq]{L2/24-210R},
the disunification of variant stacking patterns poses problems when producing
cuneiform text from transliterated corpora, as the stacking patterns are not normally
indicated in transliteration, and the default stacking pattern varies over time:
{\xsuxfont 𒐌} in Ur~III,
{\xsuxfont 𒑂} in Neo-Assyrian.
While {\xsuxfont 𒀸}, {\xsuxfont 𒁹}, and {\xsuxfont 𒌋} numerals needed to have their
stacking patterns disunified for compatibility with \cite{MZL}, this practice
should not be extended to Early Dynastic stacking patterns of
{\xsuxfont 𒀸}, {\xsuxfont 𒁹}, and {\xsuxfont 𒌋} numerals,
nor to {\xsuxfont 𒀹} numerals.

\section{AŠ×(DIŠ \emph{tenû}) numerals}\label{aš×dištenû}
This section discusses the following proposed characters:
\begin{itemize}[nosep]
  \item U+\textcsc{12477} {\cuneiformComposite\oneAshTimesDishTenu} \textcsc{CUNEIFORM NUMERIC SIGN ASH TIMES ONE DISH TENU}
  \item U+\textcsc{12478} {\cuneiformComposite\twoAshTimesDishTenu} \textcsc{CUNEIFORM NUMERIC SIGN ASH TIMES TWO DISH TENU}
  \item U+\textcsc{12479} {\cuneiformComposite\threeAshTimesDishTenu} \textcsc{CUNEIFORM NUMERIC SIGN ASH TIMES THREE DISH TENU}
  \item U+\textcsc{1247A} {\cuneiformComposite\fourAshTimesDishTenu} \textcsc{CUNEIFORM NUMERIC SIGN ASH TIMES FOUR DISH TENU}
  \item U+\textcsc{1247B} {\cuneiformComposite\fiveAshTimesDishTenu} \textcsc{CUNEIFORM NUMERIC SIGN ASH TIMES FIVE DISH TENU}
  \item U+\textcsc{1247C} {\cuneiformComposite\sixAshTimesDishTenu} \textcsc{CUNEIFORM NUMERIC SIGN ASH TIMES SIX DISH TENU}
  \item U+\textcsc{1247D} {\cuneiformComposite\sevenAshTimesDishTenu} \textcsc{CUNEIFORM NUMERIC SIGN ASH TIMES SEVEN DISH TENU}
  \item U+\textcsc{1247E} {\cuneiformComposite\eightAshTimesDishTenu} \textcsc{CUNEIFORM NUMERIC SIGN ASH TIMES EIGHT DISH TENU}
  \item U+\textcsc{1247F} {\cuneiformComposite\nineAshTimesDishTenu} \textcsc{CUNEIFORM NUMERIC SIGN ASH TIMES NINE DISH TENU}
\end{itemize}
\subsection{Name}
As indicated by their name, these signs consist of a horizontal wedge (AŠ) with an overlaid {\xsuxfont 𒀹} numeral.
Their ATF name is \texttt{$n$(|ASZ×DISZ@t|)}, as ATF numerals are of the form
\texttt{$n$($\langle\text{\textnormal{name}}\rangle$)}.
Since we have no such restriction in Unicode character names, we move the number before the \textcsc{DISH TENU}
to better describe their structure.
These numerals are not described in terms of AŠ \emph{tenû}, so we follow
\cite{CDLI} and \cite{OSL} terminology instead of attempting consistency with the names of the {\xsuxfont 𒀹} series.
Two characters already have \textcsc{DISH TENU} in their names:
U+\textcsc{12483} {\cuneiformComposite 𒒃} \textcsc{CUNEIFORM SIGN BAD TIMES DISH TENU} and
U+\textcsc{12543} {\cuneiformComposite 𒕃} \textcsc{CUNEIFORM SIGN ZU5 TIMES THREE DISH TENU}.

\subsection{Usage}
These numerals are used in the Early Dynastic IIIb period
to indicate regnal years.
They are extremely well attested, with 1482 artefacts
containing \texttt{(|ASZxDISZ@t|)} in the current transliterated \cite{CDLI} corpus.
Almost all attestations are from Ŋirsu, and most of them are in regnal years of
{\xsuxfont 𒌷𒅗𒄀𒈾} (Irikaginak\footnote{variously transliterated
iri-inim-gi-na, uru-ka-gi-na, etc., see \cite[\pno~72 \withnote~158]{SallabergerSchrakamp2015}
and literature referenced therein.}) and his predecessor {\xsuxfont 𒈗𒀭𒁕} (Lugalanda), but
their use is also attested in regnal years of earlier rulers in the first dynasty of Lagaš:
72 tablets dated to the reign of {\xsuxfont 𒂗𒇷𒋻𒍣} (Enentarzid),
\cite{P247594} possibly\footnote{Dated instead to the reign of {\xsuxfont 𒌷𒅗𒄀𒈾} by \cite[70]{SallabergerSchrakamp2015}.}
dated to the reign of {\xsuxfont 𒂗𒀭𒈾𒁺} (Enanatum) the second,
\cite{P222224} to the reign of {\xsuxfont 𒂗𒋼𒈨𒈾} (Enmetenak\footnote{Transliteration: en-mete-na, sometimes en-te-me-na.}), % [\obverse~2~4]
and \cite{P221783} from Lagaš to the reign of {\xsuxfont 𒂗𒀭𒈾𒁺} the first.

Where attested\footnote{The length of the reign of {\xsuxfont 𒈗𒀭𒁕} (6 years and 1 month)
and the dearth of documents dated to the reign of {\xsuxfont 𒌷𒅗𒄀𒈾} after his defeat by
{\xsuxfont 𒈗𒍠𒄀𒋛} mean that these are quite rare; see
\cite[\pno~71;\pno~74 \withnote~176]{SallabergerSchrakamp2015}.},
regnal years beyond the ninth are written differently:
{\xsuxfont 𒌋} for the 10th year of {\xsuxfont 𒌷𒅗𒄀𒈾} in \cite{P222640},
and with subtractive subtraction for
the 17th\footnote{This text mentions {\xsuxfont 𒂗𒇷𒋻𒍣} as temple administrator.
See \cite[69]{SallabergerSchrakamp2015} for its attribution to the reign of {\xsuxfont 𒂗𒋼𒈨𒈾}.}
written {\xsuxfont 𒎙𒇲𒑋} in \cite[\reverse~4~12]{P221483} and
the 19th year of {\xsuxfont 𒂗𒋼𒈨𒈾} written {\xsuxfont 𒎙𒇲𒑊} in \cites[\reverse~3~3]{P221413}[\reverse~3~3]{P222223}.
The numeral series therefore stops at {\xsuxfont 𒀸}×$9${\xsuxfont 𒀹}.
Figures~\ref{figArrears}--\ref{figRegnalNine} show these numerals used in ancient and modern text.

\begin{figure}[H]
  \begin{center}
  \includegraphics[width=.75\textwidth]{P220930 o copy.png}\\
  \includegraphics[width=.75\textwidth]{P220930 o.png}
  \end{center}
  \caption{Obverse of \cite{P220930}, showing {\xsuxfont 𒇲𒀀 𒅎𒅎𒈠𒄰 \oneAshTimesDishTenu}
  ``arrears of the year before last 1 (of the reign of {\xsuxfont 𒌷𒅗𒄀𒈾})'',
  {\xsuxfont 𒇲𒀀 𒅎𒈠𒄰 \twoAshTimesDishTenu}
  ``arrears of last year 2'',
  {\xsuxfont 𒇲𒀀 𒈬𒀀𒄰 \threeAshTimesDishTenu}
  ``arrears of this year 3''.
  The arrears in question consist of fish and turtles.
  Top: Copy from \cite{AllottedelaFuÿe1908}. 
  Bottom: \cite{CDLI} photograph.}\label{figArrears}
\end{figure}

\begin{figure}[H]
  \begin{center}
  \includegraphics[width=0.75\textwidth]{eng88-p166-n37.png}
  \caption{Discussion of {\xsuxfont\oneAshTimesDishTenu} notation for year names in
  \cite[\pno~166 \withnote~37]{Englund1988}, referring to \cite{P220930}.
  See Figure~\ref{figArrears}.}
  \end{center}
\end{figure}

\begin{figure}[H]
  \begin{center}
    \resizebox{.75\textwidth}{!}{
  \includegraphics[height=17mm]{P020133 r copy.png}
  \includegraphics[height=17mm]{P020133 r 1 2 sqq.png}
  }
  \end{center}
  \caption{{\xsuxfont 𒋢 𒀲 𒌑𒀸} / {\xsuxfont 𒂗𒇷𒋻𒍣} / {\xsuxfont 𒉺𒋼𒋛} / {\xsuxfont 𒉢𒁓𒆷𒆠𒅗} /
  {\xsuxfont \oneAshTimesDishTenu\twoAshTimesDishTenu\threeAshTimesDishTenu\fourAshTimesDishTenu
  𒌨𒌨𒀀 𒂊𒃻} ``Donkey skins property of Enentarzid ensik of Lagaš.
  (Years) 1 2 3 4 are all put together.''
  in \cite[\reverse~1~2~\psqq]{P020133}.
  Left: Copy from \cite{VS14}. 
  Right: \cite{CDLI} photograph.}
\end{figure}

\begin{figure}[H]
  \begin{center}
    \resizebox{.75\textwidth}{!}{
  \includegraphics[height=17mm]{P221169 r 3 2 copy.png}
  \includegraphics[height=17mm]{P221169 r 3 2.png}
  }
  \end{center}
  \caption{{\xsuxfont 𒆬 𒁇𒂀𒁀 \fiveAshTimesDishTenu\sixAshTimesDishTenu 𒄰}
  ``Silver payment 5th and 6th (years of Lugalanda)'' in \cite[\reverse~3~2]{P221169},
  dated to the 1st year of {\xsuxfont 𒌷𒅗𒄀𒈾}.
  Left: Copy from \cite{AllottedelaFuÿe1908}. 
  Right: \cite{CDLI} photograph.}
\end{figure}

\begin{figure}[H]
  \begin{center}
    \resizebox{.75\textwidth}{!}{
  \includegraphics[height=17mm]{P222006 o 2 3 copy.png}
  \includegraphics[height=17mm]{P222006 o 2 3.png}
  }
  \end{center}
  \caption{{\xsuxfont \sevenAshTimesDishTenu} (of {{\xsuxfont 𒌷𒅗𒄀𒈾}}) in \cite[\obverse~2~3]{P222006}.
  Left: Copy from \cite{Никольский1908}. 
  Right: \cite{CDLI} photograph.}
\end{figure}

\begin{figure}[H]
  \begin{center}
    \resizebox{.75\textwidth}{!}{
  \includegraphics[height=17mm]{P386436 o 2 4 sqq copy.png}
  \includegraphics[height=17mm]{P386436 o 2 4 sqq.png}
  }
  \end{center}
  \caption{{\xsuxfont 𒌷𒅗𒄀𒈾} / {\xsuxfont 𒈗} / {\xsuxfont 𒉢𒁓𒆷𒆠} / {\xsuxfont \sevenAshTimesDishTenu}
  ``Irikaginak king of Lagaš, year 7'' in \cite[\obverse~2~4~\psqq]{P386436}.
  Left: Copy from \cite{Cripps2010}. 
  Right: \cite{CDLI} photograph.}
\end{figure}

\begin{figure}[H]
  \begin{center}
    \resizebox{.75\textwidth}{!}{
  \includegraphics[height=17mm]{P221075 o 3 6 copy.png}
  \includegraphics[height=17mm]{P221075 o 3 6.png}
  }
  \end{center}
  \caption{{\xsuxfont 𒁀𒁺 \sevenAshTimesDishTenu}
  ``carried off, (year) 7 (of Lugalanda)'' in \cite[\reverse~3~6]{P221075},
  dated to the 1st year of {\xsuxfont 𒌷𒅗𒄀𒈾}.
  Left: Copy from \cite{AllottedelaFuÿe1908}. 
  Right: \cite{CDLI} photograph.}\label{figP221075}
\end{figure}

\begin{figure}[H]
  \begin{center}
    \resizebox{.75\textwidth}{!}{
  \includegraphics[height=17mm]{P221034 r 2 5 copy.png}
  \includegraphics[height=17mm]{P221034 r 2 5.png}
  }
  \end{center}
  \caption{{\xsuxfont 𒉌𒁺 \sevenAshTimesDishTenu}
  ``delivered, (year) 7 (of Lugalanda)'' in \cite[\reverse~2~5]{P221034},
  dated to the 1st year of {\xsuxfont 𒌷𒅗𒄀𒈾}.
  Left: Copy from \cite{AllottedelaFuÿe1908}. 
  Right: \cite{CDLI} photograph.}\label{figP221034}
\end{figure}

\begin{figure}[H]
  \begin{center}
    \resizebox{.75\textwidth}{!}{
  \includegraphics[height=17mm]{P222224 o 2 4 copy.png}
  \includegraphics[height=17mm]{P222224 o 2 4.png}
  }
  \end{center}
  \caption{{\xsuxfont 𒀊 \eightAshTimesDishTenu}
  (referencing the year 8 of {\xsuxfont 𒂗𒋼𒈨𒈾}) in
  \cite[\reverse~2~4]{P222224}.
  Left: Copy from \cite{CrosThureau-Dangin1910}. 
  Right: \cite{Louvre} photograph.}
\end{figure}

\begin{figure}[H]
  \begin{center}
    \resizebox{.75\textwidth}{!}{
  \includegraphics[height=17mm]{P221906 r 2 copy.png}
  \includegraphics[height=17mm]{P221906 r 2.png}
  }
  \end{center}
  \caption{{\xsuxfont \nineAshTimesDishTenu}
  (of {\xsuxfont 𒌷𒅗𒄀𒈾}) in
  \cite[\reverse~2]{P221906}.
  Left: Copy from \cite{Никольский1908}. 
  Right: \cite{CDLI} photograph.}\label{figRegnalNine}
\end{figure}

\subsection{Glyphs}
The {\xsuxfont 𒀹} wedges in these numerals are
consistently grouped for numbers above {\xsuxfont \threeAshTimesDishTenu}:
{\xsuxfont \fourAshTimesDishTenu} 2-2, {\xsuxfont \fiveAshTimesDishTenu} 3-2,
{\xsuxfont \sixAshTimesDishTenu} 3-3, {\xsuxfont \sevenAshTimesDishTenu} 4-3,
{\xsuxfont \eightAshTimesDishTenu} 4-4, and {\xsuxfont \nineAshTimesDishTenu} 3-3-3.
The representative glyphs follow this grouping.
{\xsuxfont \sevenAshTimesDishTenu} is sometimes grouped more finely 2-2-3,
as in \cites[\obverse~3~6]{P221075}[\reverse~2~5]{P221034}.
Grouping distinctions are not marked in transliteration (and often even lost in copy, as in
Figure~\ref{figP221034}), are not contrastive, and should not be represented in encoding.
The {\xsuxfont 𒀹} wedges are not stacked until the reign of {\xsuxfont 𒈗𒍠𒄀𒋛},
the notation for whose regnal years is discussed in §\ref{Lugalzagesi}.

\subsection{Later usage}\label{Lugalzagesi}
{\settowidth{\epigraphwidth}{\epigraphsize\xsuxfont 𒂕~𒉢𒁓𒆷𒆠\hspace{.5em}𒂕~𒉢𒁓𒆷𒆠\hspace{.5em}𒂕~𒉢𒁓𒆷𒆠}
\epigraph{
{\begin{multicols}{3}
{\xsuxfont 𒇽~𒄑𒆵𒆠𒆤\\
𒂕~𒉢𒁓𒆷𒆠\\
𒁀𒅆𒌨𒀀𒋫\\
𒉆𒁖\\
𒀭𒎏𒄈𒋢𒁕\\
𒂊𒁕𒀝𒅗𒀭}\\
{[…]}\\
{\xsuxfont 𒈗𒍠𒄀𒋛\\
𒉺𒋼𒋛\\
𒄑𒆵𒆠𒅗\\
𒀭𒊏𒉌\\
𒀭𒊺𒉀𒆤\\
𒉆𒁖𒁉\\
𒄘𒈾\\\hspace*{\fill}𒃶𒅍𒅍}
\end{multicols}}
Having raided Lagaš, the leader of Umma surely committed a sin against Ninŋirsu!
[…]
May Nisaba, the personal god of Lugalzagesi, the ruler of Umma, take the responsibility for the punishment!}
{\cite[\reverse~2~10 \psqq]{P222618}, translation \cite{ETCSRI}}}

A different notation of regnal years is used during the reign of {\xsuxfont 𒈗𒍠𒄀𒋛},
sometimes involving numerals of the form {\xsuxfont 𒁁}×{\xsuxfont 𒀹}.
These numerals are less comprehensively attested, and their interpretation
is sometimes still unclear, see Figure~\ref{figEng88-p144-n17}.
The co-occurrence and likely
contrast of {\xsuxfont 𒁁}×{\xsuxfont 𒀹} and {\xsuxfont 𒀸}×{\xsuxfont 𒀹}
in \cite{P221534} may preclude treating the former as a stylistic variant of the former.
Note that
U+\textcsc{12483} {\cuneiformComposite 𒒃} \textcsc{CUNEIFORM SIGN BAD TIMES DISH TENU}
is already encoded as a non-numeric character, and should be used for $1(\text{\xsuxfont 𒁁}×\text{\xsuxfont 𒀹})$
if needed.
These numerals are not being proposed at this time.

\begin{figure}[H]
  \begin{center}
  \includegraphics[width=0.75\textwidth]{eng88-p144-n17.png}
  \caption{Discussion of late presargonic dates in \cite[\pno~144 \withnote~11]{Englund1988}.
  Note that the reference to \cite{P221534} should read BIN~8, 116, rather than 117.}\label{figEng88-p144-n17}
  \end{center}
\end{figure}

\section*{Acknowledgements}
\addcontentsline{toc}{section}{Acknowledgements}
\changed{TODO(egg): Acknowledge reviewers.}
% Peter Constable and Karljürgen Feuerherm provided useful feedback on the wording.
Robin Leroy authored the bulk of the text.
Erica Scarpa suggested several useful references.
% Rick McGowan suggested including a note in the character names list to clarify the identity of shrunk numerals in the code charts.
Steve Tinney provided essential assistance on the reading of the Sumerian texts, suggested useful references,
and provided valuable feedback on early drafts of the document.
% Ken Whistler gave important advice on matters of encodability, roadmapping, code point choice, and names list editing.

The Neo-Assyrian font is \emph{Assurbanipal} and the Neo-Babylonian font is \emph{Esagil},
fonts created by Sylvie Vanséveren,
available on the Hethitologie Portal Mainz \cite{Vanséveren2021}.
The \emph{CuneiformComposite} font by Steve Tinney is used 
for the reference glyphs of already-encoded cuneiform;
the proposed reference glyphs were produced by Robin Leroy based on
\emph{CuneiformComposite}.
A modified version of \emph{Noto Sans Cuneiform}, by Monotype Imaging,
is used for most of the cuneiform text in this document; it incorporates glyphs
by Steve Tinney for the characters proposed in this document.
The font used for the characters proposed in \cite{L2/24-210R}
is the one used in that proposal, by Robin Leroy, Anshuman Pandey, and Steve Tinney.
Arabic text is set in \emph{Scheherazade New} by SIL International;
monospace text is set in \emph{Consolas} by Luc(as) de Groot;
the remainder of the text is set in \emph{Cambria} and \emph{Cambria Math} by Monotype Imaging and Tiro Typeworks.
%{\hspace*{\fill}\changed{\xsuxfont 𒉡~𒀠𒌀}}
%\vfill
\hspace*{\fill}{\xsuxfont 𒀭𒊺𒉀 𒍠𒊩}

\nocite{DCCLT}
\printbibheading[heading=bibintoc]
\AtNextBibliography{\addfontfeatures{Numbers=Tabular}}
\printbibliography[heading=subbibintoc,title={Artefacts},type=artwork]
%\AtNextBibliography{\addfontfeatures{Numbers=Tabular}}
\printbibliography[heading=subbibintoc,title={ISO and Unicode documents},nottype=artwork,keyword=unicode]
\printbibliography[heading=subbibintoc,title={Online corpora and related projects},nottype=artwork,keyword=reference]
\printbibliography[heading=subbibintoc,title={Other documents},nottype=artwork,notkeyword=unicode,notkeyword=reference]
%\includepdf[pages=-2]{summary-form.pdf}

\end{document}